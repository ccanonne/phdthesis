\chapter*{Introduction} % Not a numbered chapter
\addcontentsline{toc}{chapter}{Introduction} % Puts your introduction in your table of contents even though we have used the asterisk in the \chapter command above.

\epigraph{``The thing can be done,'' said the Butcher, ``I think.\\
The thing must be done, I am sure.\\
The thing shall be done! Bring me paper and ink,\\
The best there is time to procure.''}{Lewis Carroll, \textit{The Hunting of the Snark}}

\section*{Our contribution}

\section*{Organization of the dissertation}

In~\cref{chap:preliminaries}, we lay down the necessary notation and definitions that will be used throughout this thesis, and state some results from the literature that we shall need afterwards. We will also prove there several simple results that will be relied upon in the other chapters.~\cref{chap:unified:ub} then will be concerned with general frameworks to obtain algorithmic \emph{upper bounds} on distribution testing questions. In more detail,~\cref{sec:shaperestrictions} describes a unified approach for testing membership in classes of distributions, based on~\cite{CDGR:16} and particularly relevant for classes of \emph{shape-restricted} distributions; while~\cref{sec:fourier} contains a different approach for this question, based on~\cite{CDS:17} and well-suited for those classes of distributions which enjoy ``nice'' Fourier spectra.

In~\cref{chap:unified:lb}, we complement these algorithmic frameworks by describing new general approaches to obtaining information-theoretic \emph{lower bounds} in distribution testing.~\cref{sec:learningreductions}, based on~\cite{CDGR:16}, describes a reduction technique which allows to lift hardness of testing a sub-property $\property'\subseteq\property$ to that of testing $\property$ itself, modulo a mild learnability condition on the latter. As a corollary, we obtain new (as well as previously known) lower bounds for many distribution classes, in a clean and unified way.~\cref{sec:communication} then provides a different 
