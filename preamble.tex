\makeatletter
\@ifundefined{theorem}{%
  % Theorems (each with its own style, all same counter). Cf. http://ftp.math.purdue.edu/mirrors/ctan.org/macros/latex/contrib/hyperref/doc/manual.pdf, p.17
  \theoremstyle{plain} %% Style
  	\newtheorem{theorem}{Theorem}[section]
  	\newtheorem*{theorem*}{Theorem} % Unnumbered
  	\newaliascnt{coro}{theorem}
  	  \newtheorem{corollary}[coro]{Corollary}
  	\aliascntresetthe{coro}
  	\newaliascnt{lem}{theorem}
  		\newtheorem{lemma}[lem]{Lemma}
  	\aliascntresetthe{lem}
  	\newaliascnt{clm}{theorem}
  		\newtheorem{claim}[clm]{Claim}
	\aliascntresetthe{clm}
	\newaliascnt{fact}{theorem}
 	 	\newtheorem{fact}[theorem]{Fact}
	\aliascntresetthe{fact}
  	\newtheorem*{unnumberedfact}{Fact}
  \newaliascnt{prop}{theorem}
  		\newtheorem{proposition}[prop]{Proposition}
	\aliascntresetthe{prop}
	\newaliascnt{conj}{theorem}
  		\newtheorem{conjecture}[conj]{Conjecture}
	\aliascntresetthe{conj}
 	 \newtheorem{problem}[theorem]{Problem}
  \theoremstyle{remark} %% Style
  	\newtheorem{remark}[theorem]{Remark}
  	\newtheorem{question}[theorem]{Question}
  	\newtheorem*{notation}{Notation}
 	 \newtheorem{example}[theorem]{Example}
  \theoremstyle{definition} %% Style
  	\newaliascnt{defn}{theorem}
 		 \newtheorem{definition}[defn]{Definition}
 	 \aliascntresetthe{defn}
 	 \newtheorem{observation}[theorem]{Observation}
 	 
 	 \newtheorem{openquestion}{Open Problem}
}{}
\makeatother
\crefname{claim}{Claim}{Claims}
\newenvironment{proofof}[1]{\begin{proof}[Proof of {#1}]}{\end{proof}}

%% \email{} command
\providecommand{\email}[1]{\href{mailto:#1}{\nolinkurl{#1}\xspace}}

%% Remarks and notes
\ifnum\withcolors=1
  \newcommand{\new}[1]{{\color{red} {#1}}} % new
  \newcommand{\newer}[1]{{\color{blue} {#1}}} % even newer
  \newcommand{\newest}[1]{{\color{orange} {#1}}} % even even newer
  \newcommand{\newerest}[1]{{\color{blue!10!black!40!green} {#1}}} % you get the idea.
  \newcommand{\ccolor}[1]{{\color{RubineRed}#1}} % Clement
\else
  \newcommand{\new}[1]{{{#1}}}
  \newcommand{\newer}[1]{{{#1}}}
  \newcommand{\newest}[1]{{{#1}}}
  \newcommand{\newerest}[1]{{{#1}}}
  \newcommand{\ccolor}[1]{{#1}}
\fi

\ifnum\withnotes=1
  \newcommand{\cnote}[1]{\par\ccolor{\textbf{C: }\sf #1}} % Clement
  \newcommand{\todonote}[2][]{\todo[size=\scriptsize,color=red!40,#1]{#2}}  
	\newcommand{\questionnote}[2][]{\todo[size=\scriptsize,color=blue!30]{#2}}
	\newcommand{\todonotedone}[2][]{\todo[size=\scriptsize,color=green!40]{$\checkmark$ #2}}
	\newcommand{\todonoteinline}[2][]{\todo[inline,size=\scriptsize,color=orange!40,#1]{#2}}  
  \newcommand{\marginnote}[1]{\todo[color=white,linecolor=black]{{#1}}}
\else
  \newcommand{\cnote}[1]{}
  \newcommand{\todonote}[2][]{\ignore{#2}}
	\newcommand{\questionnote}[2][]{\ignore{#2}}
	\newcommand{\todonotedone}[2][]{\ignore{#2}}
	\newcommand{\todonoteinline}[2][]{\ignore{#2}}
  \newcommand{\marginnote}[1]{\ignore{#1}}
\fi
\newcommand{\ignore}[1]{\leavevmode\unskip} % eat unnecessary spaces before
\newcommand{\cmargin}[1]{\questionnote{\ccolor{#1}}} % Clement

% Shortcuts
\newcommand{\eps}{\ensuremath{\varepsilon}\xspace}
\newcommand{\Algo}{\ensuremath{\mathcal{A}}\xspace} % Algorithm A
\newcommand{\Tester}{\ensuremath{\mathcal{T}}\xspace} % Testing algorithm T
\newcommand{\Learner}{\ensuremath{\mathcal{L}}\xspace} % Learning algorithm L
\newcommand{\property}{\ensuremath{\mathcal{P}}\xspace} % Property P
\newcommand{\class}{\ensuremath{\mathcal{C}}\xspace} % Class C
\newcommand{\eqdef}{\stackrel{\rm def}{=}}
% \newcommand{\eqdef}{\coloneqq}
\newcommand{\eqlaw}{\stackrel{\mathcal{L}}{=}}
\newcommand{\accept}{\textsf{accept}\xspace}
\newcommand{\fail}{\textsf{fail}\xspace}
\newcommand{\reject}{\textsf{reject}\xspace}
\newcommand{\opt}{{\textsc{opt}}\xspace}
\newcommand{\domain}[1][{\Omega}]{\ensuremath{#1}\xspace} % Domain of a distribution (default notation)
\newcommand{\distribs}[1]{\Delta\!\left(#1\right)} % Domain of a distribution (default notation)
\newcommand{\yes}{\textsf{yes}\xspace}
\newcommand{\no}{\textsf{no}\xspace}
\newcommand{\dyes}{\ensuremath{\cal Y}\xspace}
\newcommand{\dno}{\ensuremath{\cal N}\xspace}

% Complexity
\newcommand{\littleO}[1]{{o\mleft( #1 \mright)}}
\newcommand{\bigO}[1]{{O\mleft( #1 \mright)}}
\newcommand{\bigOSmall}[1]{{O\big( #1 \big)}}
\newcommand{\bigTheta}[1]{{\Theta\mleft( #1 \mright)}}
\newcommand{\bigOmega}[1]{{\Omega\mleft( #1 \mright)}}
\newcommand{\bigOmegaSmall}[1]{{\Omega\big( #1 \big)}}
\newcommand{\tildeO}[1]{\tilde{O}\mleft( #1 \mright)}
\newcommand{\tildeTheta}[1]{\operatorname{\tilde{\Theta}}\mleft( #1 \mright)}
\newcommand{\tildeOmega}[1]{\operatorname{\tilde{\Omega}}\mleft( #1 \mright)}
\providecommand{\poly}{\operatorname*{poly}}

% Influence
\newcommand{\totinf}[1][f]{{\mathbf{Inf}[#1]}}
\newcommand{\infl}[2][f]{{\mathbf{Inf}_{#1}(#2)}}
\newcommand{\infldeg}[3][f]{{\mathbf{Inf}_{#1}^{#2}(#3)}}

% Sets and indicators
\newcommand{\setOfSuchThat}[2]{ \left\{\; #1 \;\colon\; #2\; \right\} } 			% sets such as "{ elems | condition }"
\newcommand{\indicSet}[1]{\mathds{1}_{#1}}                                              % indicator function
\newcommand{\indic}[1]{\indicSet{\left\{#1\right\}}}                                             % indicator function
\newcommand{\disjunion}{\sqcup}%\amalg,\coprod, \dotcup...

% Distance
\newcommand{\dtv}{\operatorname{d}_{\textrm{TV}}}
\newcommand{\hellinger}[2]{{\operatorname{d_{\textrm{H}}}\!\left({#1, #2}\right)}}
\newcommand{\kolmogorov}[2]{{\operatorname{d_{\textrm{K}}}\!\left({#1, #2}\right)}}
\newcommand{\totalvardistrestr}[3][]{{\dtv^{#1}\!\left({#2, #3}\right)}}
\newcommand{\totalvardist}[2]{\totalvardistrestr[]{#1}{#2}}
\newcommand{\chisquarerestr}[3][]{{\operatorname{d}^{#1}_{\chi^2}\!\left({#2 \mid\mid #3}\right)}}
\newcommand{\chisquare}[2]{\chisquarerestr[]{#1}{#2}}
\newcommand{\distop}{\operatorname{dist}}
\newcommand{\dist}[2]{\distop\mleft({#1, #2}\mright)}

% Restriction (functions, sequences, etc.)
\newcommand\restr[2]{{% we make the whole thing an ordinary symbol
  \left.\kern-\nulldelimiterspace % automatically resize the bar with \right
  #1 % the function
  \vphantom{\big|} % pretend it's a little taller at normal size
  \right|_{#2} % this is the delimiter
  }}

% Probability
\newcommand{\proba}{\Pr}
\newcommand{\probaOf}[1]{\proba\!\left[\, #1\, \right]}
\newcommand{\probaCond}[2]{\proba\!\left[\, #1 \;\middle\vert\; #2\, \right]}
\newcommand{\probaDistrOf}[2]{\proba_{#1}\left[\, #2\, \right]}

% Support of a distribution/function
\newcommand{\supp}[1]{\operatorname{supp}\!\left(#1\right)}

% Expectation & variance
\newcommand{\expect}[1]{\mathbb{E}\!\left[#1\right]}
\newcommand{\expectCond}[2]{\mathbb{E}\!\left[\, #1 \;\middle\vert\; #2\, \right]}
\newcommand{\shortexpect}{\mathbb{E}}
\newcommand{\var}{\operatorname{Var}}

% Distributions
\newcommand{\uniform}{\ensuremath{\mathbf{u}}}
\newcommand{\uniformOn}[1]{\ensuremath{\uniform\!\left( #1 \right) }}
\newcommand{\geom}[1]{\ensuremath{\operatorname{Geom}\!\left( #1 \right)}}
\newcommand{\bernoulli}[1]{\ensuremath{\operatorname{Bern}\!\left( #1 \right)}}
\newcommand{\bern}[2]{\ensuremath{\operatorname{Bern}^{#1}\!\left( #2 \right)}}
\newcommand{\binomial}[2]{\ensuremath{\operatorname{Bin}\!\left( #1, #2 \right)}}
\newcommand{\poisson}[1]{\ensuremath{\operatorname{Poisson}\!\left( #1 \right) }}
\newcommand{\gaussian}[2]{\ensuremath{ \mathcal{N}\!\left(#1,#2\right) }}
\newcommand{\gaussianpdf}[2]{\ensuremath{ g_{#1,#2}}}
\newcommand{\betadistr}[2]{\ensuremath{ \operatorname{Beta}\!\left( #1, #2 \right) }}

% Norms
\newcommand{\norm}[1]{\lVert#1{\rVert}}
\newcommand{\normone}[1]{{\norm{#1}}_1}
\newcommand{\normtwo}[1]{{\norm{#1}}_2}
\newcommand{\norminf}[1]{{\norm{#1}}_\infty}
\newcommand{\abs}[1]{\left\lvert #1 \right\rvert}
\newcommand{\dabs}[1]{\lvert #1 \rvert}
\newcommand{\dotprod}[2]{ \left\langle #1,\xspace #2 \right\rangle } 			% <a,b>
\newcommand{\ip}[2]{\dotprod{#1}{#2}} 			% shortcut

\newcommand{\vect}[1]{\mathbf{#1}} 			% shortcut

% Ceiling and floor
\newcommand{\clg}[1]{\left\lceil #1 \right\rceil}
\newcommand{\flr}[1]{\left\lfloor #1 \right\rfloor}

% Common sets
\newcommand{\R}{\ensuremath{\mathbb{R}}\xspace}
\newcommand{\C}{\ensuremath{\mathbb{C}}\xspace}
\newcommand{\Q}{\ensuremath{\mathbb{Q}}\xspace}
\newcommand{\Z}{\ensuremath{\mathbb{Z}}\xspace}
\newcommand{\N}{\ensuremath{\mathbb{N}}\xspace}
\newcommand{\cont}[1]{\ensuremath{\mathcal{C}^{#1}}}

% Oracles and variants
\newcommand{\ICOND}{{\sf INTCOND}\xspace}
\newcommand{\EVAL}{{\sf EVAL}\xspace}
\newcommand{\CDFEVAL}{{\sf CEVAL}\xspace}
\newcommand{\STAT}{{\sf STAT}\xspace}
\newcommand{\SAMP}{{\sf SAMP}\xspace}
\newcommand{\COND}{{\sf COND}\xspace}
\newcommand{\PCOND}{{\sf PAIRCOND}\xspace}
\newcommand{\ORACLE}{{\sf ORACLE}\xspace}

%% Terminology
\newcommand{\pdfsamp}{dual\xspace}
\newcommand{\cdfsamp}{cumulative dual\xspace}
\newcommand{\Pdfsamp}{\expandafter\capitalisewords\expandafter{\pdfsamp}}
\newcommand{\Cdfsamp}{\expandafter\capitalisewords\expandafter{\cdfsamp}}

% L_p norms
\newcommand{\lp}[1][1]{\ell_{#1}}

% Convolution
\DeclareMathOperator{\convolution}{\ast}

%% Terminology
\newcommand{\h}{\ensuremath{\mathbf{h}}} % hypothesis
\newcommand{\p}{\ensuremath{\mathbf{p}}} % distribution (main symbol)
\newcommand{\q}{\ensuremath{\mathbf{q}}} % distribution (other symbol)
\newcommand{\D}{\p}  % distribution (alternative main symbol)
%\newcommand{\D}{\ensuremath{D}}  % distribution (alternative main symbol)
\newcommand{\distrD}{\ensuremath{\mathcal{D}}}
\newcommand{\birge}[2][\D]{\Phi_{#2}(#1)}
\newcommand{\fourier}[1]{\widehat{#1}}

\newcommand{\bracketing}[3][\operatorname{d_{\textrm{H}}}]{\mathcal{N}_{[ ]}(#2,#3,#1)}

% Sign
\DeclareMathOperator{\sign}{sgn}


% Boolean
\newcommand{\bool}{\{0,1\}}
\newcommand{\junta}[2][n]{\mathcal{J}_{#2}^{(#1)}}

%% Roman numerals
\makeatletter
\newcommand{\rom}[1]{\romannumeral #1}
\newcommand{\Rom}[1]{\expandafter\@slowromancap\romannumeral #1@}
\newcommand{\century}[2][th]{\Rom{#2}\textsuperscript{#1}}
\makeatother

% Distribution classes
\newcommand{\classmon}[1][n]{\ensuremath{\mathcal{M}_{#1}}\xspace}
\newcommand{\classtmo}[1][t]{\ensuremath{\classmon[n,#1]}\xspace}
\newcommand{\classuni}{\classtmo[1]}
\newcommand{\classcvx}[1][n]{\ensuremath{\mathcal{K}_{#1}^+}\xspace}
\newcommand{\classcve}[1][n]{\ensuremath{\mathcal{K}_{#1}^-}\xspace}
\newcommand{\classmhr}[1][n]{\ensuremath{\mathcal{MHR}_{#1}}\xspace}
\newcommand{\classpbd}[1][n]{\ensuremath{\mathcal{PBD}_{#1}}\xspace}
\newcommand{\classbin}[1][n]{\ensuremath{\mathcal{BIN}_{#1}}\xspace}
\newcommand{\classpmd}[2][n]{\ensuremath{\mathcal{PMD}_{#1,#2}}\xspace}
\newcommand{\classksiirv}[2][n]{\ensuremath{\mathcal{SIIRV}_{#1,#2}}\xspace}
\newcommand{\classlogconcave}[1][n]{\ensuremath{\mathcal{LCV}_{#1}}\xspace}
\newcommand{\classpoly}[1][t,d]{\ensuremath{\mathcal{P}_{n,#1}}\xspace}
\newcommand{\classhist}[1][t]{\ensuremath{\mathcal{H}_{n,#1}}\xspace}
\newcommand{\estimdist}[1][\class]{\textsc{ProjectionDist}_{#1}}

% More macros
\newcommand{\bitset}{\{0,1\}} 
\newcommand{\F}{\mathbb{F}} 
\newcommand{\GF}{\mathsf{GF}} 
\newcommand{\LTC}{\mathsf{LTC}} 
\newcommand{\LDC}{\mathsf{LDC}}
\newcommand{\rLDC}{\mathsf{relaxed}\text{-}\LDC}
\newcommand{\decrad}{\delta_\textsf{radius}}
\newcommand{\field}[1]{\mathbb{F}_{#1}}
\newcommand{\detDT}[2]{D_{#1}(#2)} % Deterministic DT complexity (with round adaptivity k)
\newcommand{\randDT}[2]{R_{#1}(#2)} % Randomized, two-sided DT complexity (with round adaptivity k)
\newcommand{\randDTos}[2]{R^{\textsf{os}}_{#1}(#2)} % Randomized, one-sided DT complexity (with round adaptivity k)
\newcommand{\randLDT}[2]{R^{\oplus}_{#1}(#2)} % Randomized, two-sided LDT complexity (with round adaptivity k)
\newcommand{\randLDTos}[2]{R^{\oplus,\textsf{os}}_{#1}(#2)} % Randomized, one-sided LDT complexity (with round adaptivity k)

% For conditional sampling article
\newcommand{\good}{comparable\xspace}
\newcommand{\bad}{bad\xspace}
\newcommand{\unknown}{\textsf{unknown}\xspace}
\newcommand{\high}{{\textsf{high}}\xspace}
\newcommand{\low}{{\textsf{low}}\xspace}
\newcommand{\T}{{\rm T}}
\DeclareMathOperator{\rank}{rank}
\DeclareMathOperator{\wt}{sum}

% For communication complexity article
\newcommand{\kfunctional}[4]{K(#1,#2,#3,#4)}
\newcommand{\kf}[1]{\kappa_{#1}}
\newcommand{\pdistfunc}[1][\p]{\operatorname{dist}_{#1}}
\newcommand{\pweightfunc}[1][\p]{\operatorname{weight}_{#1}}
\newcommand{\pdist}[3][\p]{\pdistfunc[#1]\mleft({#2, #3}\mright)}
\newcommand{\pweight}[2][\p]{\pweightfunc[#1]\mleft({#2}\mright)}
\newcommand{\SMP}{\ensuremath{\mathsf{SMP}}}
\newcommand{\code}{\ensuremath{\mathcal{C}}}
\newcommand{\EQ}[1][n]{\textsc{Eq}_{#1}}

% For the Fourier testing article
\newcommand{\modulo}[1]{\llbracket #1\rrbracket}


% For tikz style
\makeatletter
\pgfdeclaredecoration{penciline}{initial}{
    \state{initial}[width=+\pgfdecoratedinputsegmentremainingdistance,auto corner on length=1mm,]{
        \pgfpathcurveto%
        {% From
            \pgfqpoint{\pgfdecoratedinputsegmentremainingdistance}
                            {\pgfdecorationsegmentamplitude}
        }
        {%  Control 1
        \pgfmathrand
        \pgfpointadd{\pgfqpoint{\pgfdecoratedinputsegmentremainingdistance}{0pt}}
                        {\pgfqpoint{-\pgfdecorationsegmentaspect\pgfdecoratedinputsegmentremainingdistance}%
                                        {\pgfmathresult\pgfdecorationsegmentamplitude}
                        }
        }
        {%TO 
        \pgfpointadd{\pgfpointdecoratedinputsegmentlast}{\pgfpoint{1pt}{1pt}}
        }
    }
    \state{final}{}
}
\makeatother

% %%%%%%%%%%%%%%%%%%%%%%%%%%%%%%%%%%%%%%%%%%%%%%%%%%%%%%%%%%%%%%%%
% Add author and title info to PDF (and handles multiple authors)
% %%%%%%%%%%%%%%%%%%%%%%%%%%%%%%%%%%%%%%%%%%%%%%%%%%%%%%%%%%%%%%%%
% \makeatletter
%   \AtBeginDocument{
%   \begingroup
%   \toks@={}%
%   \toksdef\toks@@=2 %
%   \toks@@={}%
%   \long\def\@ReturnFiFi#1#2\fi\fi{\fi\fi#1}%
%   \def\scan@author#1#2 \and#3\@nil{%
%   \ifx\\#3\\%
%     \ifcase#1 %
%       \toks@={#2}%
%     \else
%       \ifnum#1>1 %
%         \toks@=\expandafter{%
%           \the\expandafter\toks@\expandafter,\expandafter\space
%           \the\toks@@
%         }%
%       \fi
%       \toks@=\expandafter{\the\toks@\space and #2}%
%     \fi
%     \else
%       \ifcase#1 %
%         \toks@={#2}%
%         \@ReturnFiFi{%
%           \scan@author1#3\@nil
%         }%
%       \else
%         \ifnum#1>1 %
%           \toks@=\expandafter{%
%             \the\expandafter\toks@\expandafter,\expandafter\space
%             \the\toks@@
%           }%
%       \fi
%       \toks@@={#2}%
%       \@ReturnFiFi{%
%         \scan@author2#3\@nil
%       }%
%     \fi
%   \fi
%   }%
%   \expandafter\expandafter\expandafter\scan@author
%   \expandafter\expandafter\expandafter0%
%   \expandafter\@author\space\and\@nil
%   \edef\x{\endgroup
%   \noexpand\hypersetup{pdfauthor={\the\toks@}}%
%   }%
%   \x
%   }
% \makeatother
