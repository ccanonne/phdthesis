% This is the abstract of my dissertation.

\thispagestyle{empty} % No page number in entire abstract
\begin{center}
  ABSTRACT

  \@title

  \@author
\end{center}

\epigraph{Recently there has been a lot of glorious hullabaloo about Big Data and how it is going to revolutionize the way we work, play, eat and sleep.}{Rocco A. Servedio}

In order to study the real world, scientists (and computer scientists) develop simplified models that attempt to capture the essential features of the observed system.
Understanding the power and limitations of these models, when they apply or fail to fully capture the situation at hand, is therefore of uttermost importance.

In this thesis, we investigate the role of some of these models in property testing of probability distributions (\emph{distribution testing}), as well as in related areas. We introduce natural extensions of the standard
model (which only allows access to independent draws from the underlying distribution), in order to circumvent some of its limitations or draw new insights about the
problems they aim at capturing. Our results are organized in three main directions:

\begin{enumerate}[(i)]
  \item We provide systematic approaches to tackle distribution testing questions. Specifically, we provide two general algorithmic frameworks that apply to a wide range of properties, and yield efficient and near-optimal results for many of them. We complement these by introducing two methodologies to prove information-theoretic lower bounds in distribution testing, which enable us to derive hardness results in a clean and unified way.
  
  \item We introduce and investigate two new models of access to the unknown distributions, which both generalize the standard sampling model in different ways and allow testing algorithms to achieve significantly better efficiency. Our study of the power and limitations of algorithms in these models shows how these could lead to faster algorithms in practical situations, and  yields a better understanding of the underlying bottlenecks in the standard sampling setting.

  \item We then leave the field of distribution testing to explore areas adjacent to property testing. We define a new algorithmic primitive of \emph{sampling correction}, which in some sense lies in between distribution learning and testing and aims to capture settings where data originates from imperfect or noisy sources. Our work sets out to model these situations in a rigorous and abstracted way, in order to enable the development of systematic methods to address these issues.
\end{enumerate}
