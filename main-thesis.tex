% Dissertation Template for Columbia University Ph.D. programs
% By Charles McNamara, 2016
% It's probably a good idea to review the university guidelines just so you know what you want your dissertation to look like. You can read about those guidelines at this site: http://gsas.columbia.edu/content/formatting-guidelines.

\documentclass[letterpaper,10pt]{memoir} % The memoir class is great for longer works that use separate chapters. The Dissertation Office recommends 10-pt Arial or 12-pt Times New Roman. I use 12-pt for readability.
\def\withcolors{1}
\def\withnotes{1}

\def\withannoyingspacing{0}

% Below are some LaTeX packages to include to make sure that your Unicode characters render correctly. This is especially important if your dissertation includes polytonic Greek!
 \DisemulatePackage{setspace} % You need to use this package for "true MS Word" double-spacing.
 \usepackage{setspace} % Allows you to set different spacing (double, etc.) throughout your writing.
% Here is some stuff on the bibliography. You want to keep your bibliography file in the same directory as this file.

\usepackage[american]{babel} % Enables hyphenation and date formats according to American conventions. Change "american" to "british" (or another value) if you work outside the US.

\usepackage[T1]{fontenc}
\usepackage[utf8]{inputenc}

%% Eye-candy
\usepackage{lmodern}
\usepackage{xspace}                                     % Smart spacing with \xspace
\usepackage[protrusion=true,expansion=true]{microtype}  % Improve font rendering

% Striking out text
% \usepackage[normalem]{ulem}

% Extra fonts
\usepackage[mathscr]{euscript}

%% Math
\usepackage{amsfonts,amsmath,amssymb, amsthm, mathtools}
\usepackage{thm-restate}
\usepackage{dsfont} % For the indicator symbol
\usepackage{stmaryrd} % for [[ ]] (\llbracket,\rrbracket)

% Algorithm environment
\usepackage{algorithmicx,algpseudocode,algorithm}

% Colors (with names)
\usepackage[usenames,dvipsnames,table]{xcolor}

% Quotes: \blockquote command
\usepackage{csquotes}

% Relative sizes for text
\usepackage{relsize}

% Bibliography
%\usepackage[numbers]{natbib}

% Required for the table of results
\usepackage{multirow}
\usepackage{chngpage} % allows for temporary adjustment of side margins

% For the commands such as \capitalisewords
\usepackage{mfirstuc}

% Graphics
\usepackage{tikz}
\usetikzlibrary{arrows}
\usetikzlibrary{calc,decorations.pathmorphing,patterns}

%%\usepackage{showidx} % To debug; does not play well with hyperref

% References and links
\usepackage[colorlinks,citecolor=blue,bookmarks=true,linktocpage]{hyperref}
\usepackage{aliascnt}
\usepackage[numbered]{bookmark}
\usepackage[capitalise,nameinlink]{cleveref}

% Titling
\usepackage{titling}

% Compressed lists
\usepackage[shortlabels]{enumitem}
  \setitemize{noitemsep,topsep=3pt,parsep=2pt,partopsep=2pt} % Uncomment for compact item lists
  \setenumerate{itemsep=1pt,topsep=2pt,parsep=2pt,partopsep=2pt}
  \setdescription{itemsep=1pt}
  
% Package for todo notes.
\ifnum\withnotes=1
  \usepackage[colorinlistoftodos,textsize=scriptsize]{todonotes}
\fi

% Verbatim inputs and code
\usepackage{verbatim}
\usepackage{bigfoot} % To use {verbatim} in footnotes

% Resizable parentheses that work (without the space between \left(#1\right)
\usepackage{mleftright} % \mleft( #1 \mright)

% Plots
\usepackage{pgfplots}
\pgfplotsset{compat=1.13}

\makeatletter
\@ifundefined{theorem}{%
  % Theorems (each with its own style, all same counter). Cf. http://ftp.math.purdue.edu/mirrors/ctan.org/macros/latex/contrib/hyperref/doc/manual.pdf, p.17
  \theoremstyle{plain} %% Style
  	\newtheorem{theorem}{Theorem}[section]
  	\newtheorem*{theorem*}{Theorem} % Unnumbered
  	\newaliascnt{coro}{theorem}
  	  \newtheorem{corollary}[coro]{Corollary}
  	\aliascntresetthe{coro}
  	\newaliascnt{lem}{theorem}
  		\newtheorem{lemma}[lem]{Lemma}
  	\aliascntresetthe{lem}
  	\newaliascnt{clm}{theorem}
  		\newtheorem{claim}[clm]{Claim}
	\aliascntresetthe{clm}
	\newaliascnt{fact}{theorem}
 	 	\newtheorem{fact}[theorem]{Fact}
	\aliascntresetthe{fact}
  	\newtheorem*{unnumberedfact}{Fact}
  \newaliascnt{prop}{theorem}
  		\newtheorem{proposition}[prop]{Proposition}
	\aliascntresetthe{prop}
	\newaliascnt{conj}{theorem}
  		\newtheorem{conjecture}[conj]{Conjecture}
	\aliascntresetthe{conj}
 	 \newtheorem{problem}[theorem]{Problem}
  \theoremstyle{remark} %% Style
  	\newtheorem{remark}[theorem]{Remark}
  	\newtheorem{question}[theorem]{Question}
  	\newtheorem*{notation}{Notation}
 	 \newtheorem{example}[theorem]{Example}
  \theoremstyle{definition} %% Style
  	\newaliascnt{defn}{theorem}
 		 \newtheorem{definition}[defn]{Definition}
 	 \aliascntresetthe{defn}
 	 \newtheorem{observation}[theorem]{Observation}
 	 
 	 \newtheorem{openquestion}{Open Problem}
}{}
\makeatother
\crefname{claim}{Claim}{Claims}
\newenvironment{proofof}[1]{\begin{proof}[Proof of {#1}]}{\end{proof}}

%% \email{} command
\providecommand{\email}[1]{\href{mailto:#1}{\nolinkurl{#1}\xspace}}

%% Remarks and notes
\ifnum\withcolors=1
  \newcommand{\new}[1]{{\color{red} {#1}}} % new
  \newcommand{\newer}[1]{{\color{blue} {#1}}} % even newer
  \newcommand{\newest}[1]{{\color{orange} {#1}}} % even even newer
  \newcommand{\newerest}[1]{{\color{blue!10!black!40!green} {#1}}} % you get the idea.
  \newcommand{\ccolor}[1]{{\color{RubineRed}#1}} % Clement
\else
  \newcommand{\new}[1]{{{#1}}}
  \newcommand{\newer}[1]{{{#1}}}
  \newcommand{\newest}[1]{{{#1}}}
  \newcommand{\newerest}[1]{{{#1}}}
  \newcommand{\ccolor}[1]{{#1}}
\fi

\ifnum\withnotes=1
  \newcommand{\cnote}[1]{\par\ccolor{\textbf{C: }\sf #1}} % Clement
  \newcommand{\todonote}[2][]{\todo[size=\scriptsize,color=red!40,#1]{#2}}  
	\newcommand{\questionnote}[2][]{\todo[size=\scriptsize,color=blue!30]{#2}}
	\newcommand{\todonotedone}[2][]{\todo[size=\scriptsize,color=green!40]{$\checkmark$ #2}}
	\newcommand{\todonoteinline}[2][]{\todo[inline,size=\scriptsize,color=orange!40,#1]{#2}}  
  \newcommand{\marginnote}[1]{\todo[color=white,linecolor=black]{{#1}}}
\else
  \newcommand{\cnote}[1]{}
  \newcommand{\todonote}[2][]{\ignore{#2}}
	\newcommand{\questionnote}[2][]{\ignore{#2}}
	\newcommand{\todonotedone}[2][]{\ignore{#2}}
	\newcommand{\todonoteinline}[2][]{\ignore{#2}}
  \newcommand{\marginnote}[1]{\ignore{#1}}
\fi
\newcommand{\ignore}[1]{\leavevmode\unskip} % eat unnecessary spaces before
\newcommand{\cmargin}[1]{\questionnote{\ccolor{#1}}} % Clement

% Shortcuts
\newcommand{\eps}{\ensuremath{\varepsilon}\xspace}
\newcommand{\Algo}{\ensuremath{\mathcal{A}}\xspace} % Algorithm A
\newcommand{\Tester}{\ensuremath{\mathcal{T}}\xspace} % Testing algorithm T
\newcommand{\Learner}{\ensuremath{\mathcal{L}}\xspace} % Learning algorithm L
\newcommand{\property}{\ensuremath{\mathcal{P}}\xspace} % Property P
\newcommand{\class}{\ensuremath{\mathcal{C}}\xspace} % Class C
\newcommand{\eqdef}{\stackrel{\rm def}{=}}
% \newcommand{\eqdef}{\coloneqq}
\newcommand{\eqlaw}{\stackrel{\mathcal{L}}{=}}
\newcommand{\accept}{\textsf{accept}\xspace}
\newcommand{\fail}{\textsf{fail}\xspace}
\newcommand{\reject}{\textsf{reject}\xspace}
\newcommand{\opt}{{\textsc{opt}}\xspace}
\newcommand{\domain}[1][{\Omega}]{\ensuremath{#1}\xspace} % Domain of a distribution (default notation)
\newcommand{\distribs}[1]{\Delta\!\left(#1\right)} % Domain of a distribution (default notation)
\newcommand{\yes}{\textsf{yes}\xspace}
\newcommand{\no}{\textsf{no}\xspace}
\newcommand{\dyes}{\ensuremath{\cal Y}\xspace}
\newcommand{\dno}{\ensuremath{\cal N}\xspace}

% Complexity
\newcommand{\littleO}[1]{{o\mleft( #1 \mright)}}
\newcommand{\bigO}[1]{{O\mleft( #1 \mright)}}
\newcommand{\bigOSmall}[1]{{O\big( #1 \big)}}
\newcommand{\bigTheta}[1]{{\Theta\mleft( #1 \mright)}}
\newcommand{\bigOmega}[1]{{\Omega\mleft( #1 \mright)}}
\newcommand{\bigOmegaSmall}[1]{{\Omega\big( #1 \big)}}
\newcommand{\tildeO}[1]{\tilde{O}\mleft( #1 \mright)}
\newcommand{\tildeTheta}[1]{\operatorname{\tilde{\Theta}}\mleft( #1 \mright)}
\newcommand{\tildeOmega}[1]{\operatorname{\tilde{\Omega}}\mleft( #1 \mright)}
\providecommand{\poly}{\operatorname*{poly}}

% Influence
\newcommand{\totinf}[1][f]{{\mathbf{Inf}[#1]}}
\newcommand{\infl}[2][f]{{\mathbf{Inf}_{#1}(#2)}}
\newcommand{\infldeg}[3][f]{{\mathbf{Inf}_{#1}^{#2}(#3)}}

% Sets and indicators
\newcommand{\setOfSuchThat}[2]{ \left\{\; #1 \;\colon\; #2\; \right\} } 			% sets such as "{ elems | condition }"
\newcommand{\indicSet}[1]{\mathds{1}_{#1}}                                              % indicator function
\newcommand{\indic}[1]{\indicSet{\left\{#1\right\}}}                                             % indicator function
\newcommand{\disjunion}{\sqcup}%\amalg,\coprod, \dotcup...

% Distance
\newcommand{\dtv}{\operatorname{d}_{\textrm{TV}}}
\newcommand{\hellinger}[2]{{\operatorname{d_{\textrm{H}}}\!\left({#1, #2}\right)}}
\newcommand{\kolmogorov}[2]{{\operatorname{d_{\textrm{K}}}\!\left({#1, #2}\right)}}
\newcommand{\totalvardistrestr}[3][]{{\dtv^{#1}\!\left({#2, #3}\right)}}
\newcommand{\totalvardist}[2]{\totalvardistrestr[]{#1}{#2}}
\newcommand{\chisquarerestr}[3][]{{\operatorname{d}^{#1}_{\chi^2}\!\left({#2 \mid\mid #3}\right)}}
\newcommand{\chisquare}[2]{\chisquarerestr[]{#1}{#2}}
\newcommand{\distop}{\operatorname{dist}}
\newcommand{\dist}[2]{\distop\mleft({#1, #2}\mright)}

% Restriction (functions, sequences, etc.)
\newcommand\restr[2]{{% we make the whole thing an ordinary symbol
  \left.\kern-\nulldelimiterspace % automatically resize the bar with \right
  #1 % the function
  \vphantom{\big|} % pretend it's a little taller at normal size
  \right|_{#2} % this is the delimiter
  }}

% Probability
\newcommand{\proba}{\Pr}
\newcommand{\probaOf}[1]{\proba\!\left[\, #1\, \right]}
\newcommand{\probaCond}[2]{\proba\!\left[\, #1 \;\middle\vert\; #2\, \right]}
\newcommand{\probaDistrOf}[2]{\proba_{#1}\left[\, #2\, \right]}

% Support of a distribution/function
\newcommand{\supp}[1]{\operatorname{supp}\!\left(#1\right)}

% Expectation & variance
\newcommand{\expect}[1]{\mathbb{E}\!\left[#1\right]}
\newcommand{\expectCond}[2]{\mathbb{E}\!\left[\, #1 \;\middle\vert\; #2\, \right]}
\newcommand{\shortexpect}{\mathbb{E}}
\newcommand{\var}{\operatorname{Var}}

% Distributions
\newcommand{\uniform}{\ensuremath{\mathcal{U}}}
\newcommand{\uniformOn}[1]{\ensuremath{\uniform\!\left( #1 \right) }}
\newcommand{\geom}[1]{\ensuremath{\operatorname{Geom}\!\left( #1 \right)}}
\newcommand{\bernoulli}[1]{\ensuremath{\operatorname{Bern}\!\left( #1 \right)}}
\newcommand{\bern}[2]{\ensuremath{\operatorname{Bern}^{#1}\!\left( #2 \right)}}
\newcommand{\binomial}[2]{\ensuremath{\operatorname{Bin}\!\left( #1, #2 \right)}}
\newcommand{\poisson}[1]{\ensuremath{\operatorname{Poisson}\!\left( #1 \right) }}
\newcommand{\gaussian}[2]{\ensuremath{ \mathcal{N}\!\left(#1,#2\right) }}
\newcommand{\gaussianpdf}[2]{\ensuremath{ g_{#1,#2}}}
\newcommand{\betadistr}[2]{\ensuremath{ \operatorname{Beta}\!\left( #1, #2 \right) }}

% Norms
\newcommand{\norm}[1]{\lVert#1{\rVert}}
\newcommand{\normone}[1]{{\norm{#1}}_1}
\newcommand{\normtwo}[1]{{\norm{#1}}_2}
\newcommand{\norminf}[1]{{\norm{#1}}_\infty}
\newcommand{\abs}[1]{\left\lvert #1 \right\rvert}
\newcommand{\dabs}[1]{\lvert #1 \rvert}
\newcommand{\dotprod}[2]{ \left\langle #1,\xspace #2 \right\rangle } 			% <a,b>
\newcommand{\ip}[2]{\dotprod{#1}{#2}} 			% shortcut

\newcommand{\vect}[1]{\mathbf{#1}} 			% shortcut

% Ceiling and floor
\newcommand{\clg}[1]{\left\lceil #1 \right\rceil}
\newcommand{\flr}[1]{\left\lfloor #1 \right\rfloor}

% Common sets
\newcommand{\R}{\ensuremath{\mathbb{R}}\xspace}
\newcommand{\C}{\ensuremath{\mathbb{C}}\xspace}
\newcommand{\Q}{\ensuremath{\mathbb{Q}}\xspace}
\newcommand{\Z}{\ensuremath{\mathbb{Z}}\xspace}
\newcommand{\N}{\ensuremath{\mathbb{N}}\xspace}
\newcommand{\cont}[1]{\ensuremath{\mathcal{C}^{#1}}}

% Oracles and variants
\newcommand{\ICOND}{{\sf INTCOND}\xspace}
\newcommand{\EVAL}{{\sf EVAL}\xspace}
\newcommand{\CDFEVAL}{{\sf CEVAL}\xspace}
\newcommand{\STAT}{{\sf STAT}\xspace}
\newcommand{\SAMP}{{\sf SAMP}\xspace}
\newcommand{\COND}{{\sf COND}\xspace}
\newcommand{\PCOND}{{\sf PAIRCOND}\xspace}
\newcommand{\ORACLE}{{\sf ORACLE}\xspace}

%% Terminology
\newcommand{\pdfsamp}{dual\xspace}
\newcommand{\cdfsamp}{cumulative dual\xspace}
\newcommand{\Pdfsamp}{\expandafter\capitalisewords\expandafter{\pdfsamp}}
\newcommand{\Cdfsamp}{\expandafter\capitalisewords\expandafter{\cdfsamp}}

% L_p norms
\newcommand{\lp}[1][1]{\ell_{#1}}

% Convolution
\DeclareMathOperator{\convolution}{\ast}

%% Terminology
\newcommand{\h}{\ensuremath{\mathbf{h}}} % hypothesis
\newcommand{\p}{\ensuremath{\mathbf{p}}} % distribution (main symbol)
\newcommand{\q}{\ensuremath{\mathbf{p}}} % distribution (other symbol)
\newcommand{\D}{\p}  % distribution (alternative main symbol)
%\newcommand{\D}{\ensuremath{D}}  % distribution (alternative main symbol)
\newcommand{\distrD}{\ensuremath{\mathcal{D}}}
\newcommand{\birge}[2][\D]{\Phi_{#2}(#1)}
\newcommand{\fourier}[1]{\widehat{#1}}

\newcommand{\bracketing}[3][\operatorname{d_{\textrm{H}}}]{\mathcal{N}_{[ ]}(#2,#3,#1)}

% Sign
\DeclareMathOperator{\sign}{sgn}


% Boolean
\newcommand{\bool}{\{0,1\}}
\newcommand{\junta}[2][n]{\mathcal{J}_{#2}^{(#1)}}

%% Roman numerals
\makeatletter
\newcommand{\rom}[1]{\romannumeral #1}
\newcommand{\Rom}[1]{\expandafter\@slowromancap\romannumeral #1@}
\newcommand{\century}[2][th]{\Rom{#2}\textsuperscript{#1}}
\makeatother

% Distribution classes
\newcommand{\classmon}[1][n]{\ensuremath{\mathcal{M}_{#1}}\xspace}
\newcommand{\classtmo}[1][t]{\ensuremath{\classmon[n,#1]}\xspace}
\newcommand{\classuni}{\classtmo[1]}
\newcommand{\classcvx}[1][n]{\ensuremath{\mathcal{K}_{#1}^+}\xspace}
\newcommand{\classcve}[1][n]{\ensuremath{\mathcal{K}_{#1}^-}\xspace}
\newcommand{\classmhr}[1][n]{\ensuremath{\mathcal{MHR}_{#1}}\xspace}
\newcommand{\classpbd}[1][n]{\ensuremath{\mathcal{PBD}_{#1}}\xspace}
\newcommand{\classbin}[1][n]{\ensuremath{\mathcal{BIN}_{#1}}\xspace}
\newcommand{\classpmd}[2][n]{\ensuremath{\mathcal{PMD}_{#1,#2}}\xspace}
\newcommand{\classksiirv}[2][n]{\ensuremath{\mathcal{SIIRV}_{#1,#2}}\xspace}
\newcommand{\classlogconcave}[1][n]{\ensuremath{\mathcal{LCV}_{#1}}\xspace}
\newcommand{\classpoly}[1][t,d]{\ensuremath{\mathcal{P}_{n,#1}}\xspace}
\newcommand{\classhist}[1][t]{\ensuremath{\mathcal{H}_{n,#1}}\xspace}
\newcommand{\estimdist}[1][\class]{\textsc{ProjectionDist}_{#1}}

% More macros
\newcommand{\bitset}{\{0,1\}} 
\newcommand{\F}{\mathbb{F}} 
\newcommand{\GF}{\mathsf{GF}} 
\newcommand{\LTC}{\mathsf{LTC}} 
\newcommand{\LDC}{\mathsf{LDC}}
\newcommand{\rLDC}{\mathsf{relaxed}\text{-}\LDC}
\newcommand{\decrad}{\delta_\textsf{radius}}
\newcommand{\field}[1]{\mathbb{F}_{#1}}
\newcommand{\detDT}[2]{D_{#1}(#2)} % Deterministic DT complexity (with round adaptivity k)
\newcommand{\randDT}[2]{R_{#1}(#2)} % Randomized, two-sided DT complexity (with round adaptivity k)
\newcommand{\randDTos}[2]{R^{\textsf{os}}_{#1}(#2)} % Randomized, one-sided DT complexity (with round adaptivity k)
\newcommand{\randLDT}[2]{R^{\oplus}_{#1}(#2)} % Randomized, two-sided LDT complexity (with round adaptivity k)
\newcommand{\randLDTos}[2]{R^{\oplus,\textsf{os}}_{#1}(#2)} % Randomized, one-sided LDT complexity (with round adaptivity k)

% For conditional sampling article
\newcommand{\good}{comparable\xspace}
\newcommand{\bad}{bad\xspace}
\newcommand{\unknown}{\textsf{unknown}\xspace}
\newcommand{\high}{{\textsf{high}}\xspace}
\newcommand{\low}{{\textsf{low}}\xspace}
\newcommand{\T}{{\rm T}}
\DeclareMathOperator{\rank}{rank}
\DeclareMathOperator{\wt}{sum}

% For communication complexity article
\newcommand{\kfunctional}[4]{K(#1,#2,#3,#4)}
\newcommand{\kf}[1]{\kappa_{#1}}
\newcommand{\pdistfunc}[1][\p]{\operatorname{dist}_{#1}}
\newcommand{\pweightfunc}[1][\p]{\operatorname{weight}_{#1}}
\newcommand{\pdist}[3][\p]{\pdistfunc[#1]\mleft({#2, #3}\mright)}
\newcommand{\pweight}[2][\p]{\pweightfunc[#1]\mleft({#2}\mright)}
\newcommand{\SMP}{\ensuremath{\mathsf{SMP}}}
\newcommand{\code}{\ensuremath{\mathcal{C}}}
\newcommand{\EQ}[1][n]{\textsc{Eq}_{#1}}


% %%%%%%%%%%%%%%%%%%%%%%%%%%%%%%%%%%%%%%%%%%%%%%%%%%%%%%%%%%%%%%%%
% Add author and title info to PDF (and handles multiple authors)
% %%%%%%%%%%%%%%%%%%%%%%%%%%%%%%%%%%%%%%%%%%%%%%%%%%%%%%%%%%%%%%%%
% \makeatletter
%   \AtBeginDocument{
%   \begingroup
%   \toks@={}%
%   \toksdef\toks@@=2 %
%   \toks@@={}%
%   \long\def\@ReturnFiFi#1#2\fi\fi{\fi\fi#1}%
%   \def\scan@author#1#2 \and#3\@nil{%
%   \ifx\\#3\\%
%     \ifcase#1 %
%       \toks@={#2}%
%     \else
%       \ifnum#1>1 %
%         \toks@=\expandafter{%
%           \the\expandafter\toks@\expandafter,\expandafter\space
%           \the\toks@@
%         }%
%       \fi
%       \toks@=\expandafter{\the\toks@\space and #2}%
%     \fi
%     \else
%       \ifcase#1 %
%         \toks@={#2}%
%         \@ReturnFiFi{%
%           \scan@author1#3\@nil
%         }%
%       \else
%         \ifnum#1>1 %
%           \toks@=\expandafter{%
%             \the\expandafter\toks@\expandafter,\expandafter\space
%             \the\toks@@
%           }%
%       \fi
%       \toks@@={#2}%
%       \@ReturnFiFi{%
%         \scan@author2#3\@nil
%       }%
%     \fi
%   \fi
%   }%
%   \expandafter\expandafter\expandafter\scan@author
%   \expandafter\expandafter\expandafter0%
%   \expandafter\@author\space\and\@nil
%   \edef\x{\endgroup
%   \noexpand\hypersetup{pdfauthor={\the\toks@}}%
%   }%
%   \x
%   }
% \makeatother


%\usepackage{utopia} % Nice font. I like it.
\usepackage{times} % Stupid font. I like it too.

\title{Property Testing and Probability Distributions: New Insights from New Models}
\author{Cl\'{e}ment L. Canonne}
\date{2017}



\usepackage[backend=bibtex,style=authoryear-icomp]{biblatex} % Use biblatex
\addbibresource{./bibliography.bib} % Your bibliography file.

% Here you can set margins and other page formatting
\setlrmarginsandblock{3.175cm}{3.175cm}{*} % Left and right margin -- the dissertation office requires at least 1-inch margins
\setulmarginsandblock{2.54cm}{2.54cm}{*}  % Upper and lower margin -- same thing, at least 1-inch margins
\checkandfixthelayout % A function of the memoir class that finds the right number of lines per page and apparently tidies up the formatting in other mysterious ways...?
\fixpdflayout

 \begin{document}
\chapterstyle{chappell} % Nice formatting for chapter headings. Check out the documentation for the memoir class for other options.
\footnotesep\baselineskip % Footnotes need to have a space between each one for Columbia's Dissertation Office. 
\ifnum\withannoyingspacing=1
  \setstretch{2} % Set line spacing to "true MS Word" double-spacing
\else
  \setstretch{1.5}
\fi
\pagestyle{plain} % Put page numbers at the bottom-center for the whole dissertation. Columbia's dissertation office requires that the numbers appear at this location on the page throughout the document.


\makeatletter
% This is the title page of my dissertation.
% You may need to change the text on this page for your particular department. Consult the Dissertation Office Formatting Guidelines.
\thispagestyle{empty} % No page number
\begin{center}
  \SingleSpace

  \vspace*{1in}

  \@title

  \bigskip % Get space between title and author name

  \@author

  \vspace{5in}

  Submitted in partial fulfillment of the\\
  requirements for the degree of\\
  Doctor of Philosophy\\
  in the Graduate School of Arts and Sciences

  \bigskip
  \bigskip

  COLUMBIA UNIVERSITY

  \bigskip % Get space between Columbia and the year

  \@date
\end{center}

% This is the copyright page of my dissertation.
% You might also consider using a Creative Commons license. Ask your adviser or the university's Copyright Office.

\thispagestyle{empty} % No page number
\null\vfill % Pushes Copyright to the bottom of the page.
\begin{center}
  \SingleSpace %Keep the Copyright single-spaced
© \@date~\@author\\
This work is licensed under the Creative Commons Attribution-ShareAlike 4.0 International License. To view a copy of this license, visit \url{http://creativecommons.org/licenses/by-sa/4.0/} or send a letter to Creative Commons, PO Box 1866, Mountain View, CA 94042, USA.
\end{center}
  %% Here's the copyright page. I use a Creative Commons license.
% This is the abstract of my dissertation.

\pagestyle{empty} % No page number in entire abstract
\begin{center}
  ABSTRACT

  \@title

  \@author
\end{center}

\epigraph{Recently there has been a lot of glorious hullabaloo about Big Data and how it is going to revolutionize the way we work, play, eat and sleep.}{Rocco A. Servedio}

In order to study the real world, scientists (and computer scientists) develop simplified models that attempt to capture the essential features of the observed system.
Understanding the power and limitations of these models, when they apply or fail to fully capture the situation at hand, is therefore of uttermost importance.

In this thesis, we investigate the role of some of these models in property testing of probability distributions (\emph{distribution testing}), as well as in related areas. We introduce natural extensions of the standard
model (which only allows access to independent draws from the underlying distribution), in order to circumvent some of its limitations or draw new insights about the
problems they aim at capturing. Our results are organized in three main directions:

\begin{enumerate}[(i)]
  \item We provide systematic approaches to tackle distribution testing questions. Specifically, we provide two general algorithmic frameworks that apply to a wide range of properties, and yield efficient and near-optimal results for many of them. We complement these by introducing two methodologies to prove information-theoretic lower bounds in distribution testing, which enable us to derive hardness results in a clean and unified way.
  
  \item We introduce and investigate two new models of access to the unknown distributions, which both generalize the standard sampling model in different ways and allow testing algorithms to achieve significantly better efficiency. Our study of the power and limitations of algorithms in these models shows how these could lead to faster algorithms in practical situations, and  yields a better understanding of the underlying bottlenecks in the standard sampling setting.

  \item We then leave the field of distribution testing to explore areas adjacent to property testing. We define a new algorithmic primitive of \emph{sampling correction}, which in some sense lies in between distribution learning and testing and aims to capture settings where data originates from imperfect or noisy sources. Our work sets out to model these situations in a rigorous and abstracted way, in order to enable the development of systematic methods to address these issues.
\end{enumerate}

\makeatother

\frontmatter
\tableofcontents % You need to put your ToC after frontmatter so that it will get lowercase Roman numerals.

%%% A list of graphs and illustrations should go here if you use any.

% This is the acknowledgments page of my

\cleartorecto % A memoir-class command for moving the acknowledgments to a recto page, not verso.
\chapter{Acknowledgments} % For the heading on the page, also registers in the table of contents
\thispagestyle{plain} %This page should have numbers.

We are but dwarfs seating on the shoulders of bigger dwarfs.

% This is the dedication page of my dissertation.

\cleartorecto % A memoir-class command for moving the dedication to a recto page, not verso.
\thispagestyle{plain} % This page should be numbered

\begin{center}
  To Nandini, who told me what a wombat was.
\end{center}


%%% Preface if you have one



%%%%%%%%%%%%%%%%%%%%%%%%%%%%%%%%%%%%%%%%%%%%%%%%%%%%%%%%%%%%%%%%%%%%%%%%%%%%%%%%%%%%%%%%%%%%%%%%%%%% 
%%%%%%%%%%%%%%%%%%%%%%%%%%%%%%%%%%%%%%%%%%%%%%%%%%%%%%%%%%%%%%%%%%%%%%%%%%%%%%%%%%%%%%%%%%%%%%%%%%%% 
% What follows is the main text of your dissertation. You can comment out lines if you want to exclude them from your document for drafts. Everything after \mainmatter will get Arabic numbers centered on the bottom of the page.
\mainmatter

\chapter*{Introduction} % Not a numbered chapter
\addcontentsline{toc}{chapter}{Introduction} % Puts your introduction in your table of contents even though we have used the asterisk in the \chapter command above.

\epigraph{``The thing can be done,'' said the Butcher, ``I think.\\
The thing must be done, I am sure.\\
The thing shall be done! Bring me paper and ink,\\
The best there is time to procure.''}{Lewis Carroll, \textit{The Hunting of the Snark}}

%%%%%%%%%%%%%%%%%%%%%%%%%%%%%%%%%%%%%%%%%%%%%%%%%%%%%%%%%%%%%%%%%%%%%%%%%%%%%%%%%%%%%%%%%%%%%%%%%%%%%%%%%%%%%%%%%%%%%%%%%%%%%%%%%%%%%%%%%%

This dissertation revolves around discrete probability distributions: the wild and empirical ones, found in the ``real world''  wherever data can be found; or the familiar and abstract ones, which underly our (idealized) models of that very same world and let us reason about it. The practical details of the situations in which these distributions show up will not be of too much concern for us: instead, we will take their presence as a given, seeing them as an abstract source of data, values -- ``samples.''

And indeed, inferring \emph{information} from the probability distribution underlying available data is a fundamental problem in Statistics and data analysis, with applications and ramifications in countless other fields. One may want to approximate that distribution in its entirety; or, less ambitiously, to check whether it is consistent with a prespecified model; one may even only want to approximate some simple parameters such as its mean or first few moments. But this decades-old inference question, regardless of its specific variant, has undergone a significant shift these past few years: the amount of data to analyze has grown huge, and our distributions now are often over a \emph{very large} domain. So huge and so large, in fact, that the seasoned and well-studied methods from Statistics and learning theory are no longer practical; and one has to look for faster, more sample-efficient techniques and algorithms.

We may not be able to obtain these in general. But in many situations, we are only interested in figuring out only some very specific information about our probability distribution: we made an assumption or formulated a hypothesis, and want to check whether we were right. To get this \emph{one bit} of information, and this bit only, it may just be possible to overcome the formidable complexity of the task. Understanding when it is, and how, is precisely what the field of distribution testing is about.

Distribution testing, as first explicitly introduced in~\cite{BFRSW:00}, is a branch of property testing~\cite{RS:96,GGR:98}: in the latter, access to an unknown ``huge object'' is presented to an algorithm \textit{via} the ability to perform local ``inspections.'' By making only a small number of such queries to the object, the algorithm must determine with high probability whether the object exhibits some prespecified property of interest, or is \emph{far} from every object with the property. (For a more detailed presentation and overview of the field of property testing, the reader is referred to~\cite{Fischer:01,Ron:08,Ron:10,Goldreich:10,Gol:17,BY:17}.)

In distribution testing, this ``huge object'' is an unknown probability distribution (or a collection thereof) over some known discrete domain $\domain$; and the type of access granted to this distribution is (usually) access to independent samples drawn from that distribution. The question now becomes to bound the number of samples required to test a given statistical property -- as a function of the domain size and the ``farness'' parameter. (In particular, the running time of the algorithm is usually only a secondary concern, even though obtaining time-efficient testers is an explicit goal in many works.) Distribution testing has been a very active area over the past fifteen years, with a flurry\footnote{I immensely enjoy the word ``flurry.''}{} of variants and exciting developments: we refer the reader of this thesis to the surveys~\cite{Rubinfeld:12:Survey,Canonne:15:Survey}, and the book~\cite{Gol:17}, for a more complete picture; and will focus afterwards on our narrow contribution to this field.

%%%%%%%%%%%%%%%%%%%%%%%%%%%%%%%%%%%%%%%%%%%%%%%%%%%%%%%%%%%%%%%%%%%%%%%%%%%%%%%%%%%%%%%%%%%%%%%%%%%%%%%%%%%%%%%%%%%%%%%%%%%%%%%%%%%%%%%%%%
\section*{Our contributions}
Before delving into the specific and technical details, we provide a high-level overview of our contributions. As we shall see, they are organized in three main axes:

\subsection*{Beyond the standard distribution testing \emph{techniques}}  As aforementioned, distribution testing has been been the focus of a significant body of works over recent years, culminating in a full understanding of the complexity for a large number of testing questions. However, while many of these questions have seen their sample complexity fully resolved, these advances have for a large part been the result of distinct, \textit{ad hoc} techniques tailored to the specific problems they were meant to solve. Overall, we still lack \emph{general} tools to tackle distribution testing questions -- both to establish (algorithmic) upper and (information-theoretic) lower bounds.

The first contribution of this thesis is to establish both general algorithmic frameworks (``Swiss Army knives'') and lower bound techniques (``easy reductions'') to attack these questions. Our results are widely applicable, and yield optimal or near-optimal bounds for a variety of (arguably) fundamental testing questions. In this sense, our work can be viewed as building up a user-friendly toolbox for distribution testing, which should come in handy to anyone in the field.

\subsection*{Beyond the \emph{standard} distribution testing techniques} One of the takeaway messages of the aforementioned recent flurry of results in distribution testing is that achieving a sublinear sample complexity with regard to the domain size \emph{is} achievable for most properties of interest. Another takeaway, however, is that this sublinear sample complexity has to be \emph{polynomial} in this domain size $n$, i.e. of the form $n^{\bigOmega{1}}$ -- which, is many real-world settings, turns out to still be prohibitively high. Thus, it is natural to consider natural extensions of the standard ``sample-only'' model, where algorithms now get to have a stronger type of \emph{access} to the unknown probability distribution -- and see if this additional power allows them to achieve a significant better sample complexity. 

In this thesis, we introduce and study two such generalizations (along with some of their variants), which we argue can be implemented in practical situations. The main message is that, whenever these new models are applicable, one can perform much better than in the standard sampling model -- sometimes with a sample complexity \emph{independent} of the domain size. Moreover, such stronger models can also help us in understand what exactly makes these question ``hard'' in the standard sampling model in the first place, and therefore hopefully guide implementations even of ``standard'' testing algorithms by adapting them on a case-by-case basis.\cmargin{Is it too much?}

\subsection*{Beyond the standard distribution \emph{testing} techniques} So far, we stayed within the real of distribution testing: focusing on a specific property of distributions, how to decide whether the unknown one we have access to indeed satisfies this property. This, however, may not be the end goal: for instance, what if after running such a test, we knew the distribution is \emph{close} to having that property -- yet are not guaranteed it does \emph{exactly}? What if a subsequent algorithm, or application, requires such a guarantee? To handle such questions, we introduce the notion of a \emph{sampling corrector}, which (broadly speaking) acts as a filter between a source of imperfect samples and an algorithm to provide access to ``corrected samples'' -- whose distribution is close to the original one, but now does satisfy the property of interest. We further explore this new paradigm of simple correction (and its weaker variant of sampling \emph{improvement}), and study its connections to distribution testing and learning -- showing two-way implications that may prove fruitful in establishing new upper and lower bound in either direction. 

%%%%%%%%%%%%%%%%%%%%%%%%%%%%%%%%%%%%%%%%%%%%%%%%%%%%%%%%%%%%%%%%%%%%%%%%%%%%%%%%%%%%%%%%%%%%%%%%%%%%%%%%%%%%%%%%%%%%%%%%%%%%%%%%%%%%%%%%%%
\section*{Organization of the dissertation}

In~\cref{chap:preliminaries}, we lay down the necessary notation and definitions that will be used throughout this thesis, and state some results from the literature that we shall need afterwards. We will also prove there several simple results that will be relied upon in the other chapters, and more generally set up the board and pieces.~\cref{chap:unified:ub} then will be concerned with general strategies to play the game; or, put differently, with unified frameworks to obtain algorithmic \emph{upper bounds} on distribution testing questions. In more detail,~\cref{sec:shaperestrictions} describes a unified approach for testing membership in classes of distributions, particularly relevant for classes of \emph{shape-restricted} distributions; while~\cref{sec:fourier} contains a different approach for this question, well-suited for those classes of distributions which enjoy ``nice'' Fourier spectra. The first is based on joint work with Ilias Diakonikolas, Themis Gouleakis, and Ronitt Rubinfeld~\cite{CDGR:16}, and the second on the paper~\cite{CDS:17} with Ilias Diakonikolas and Alistair Stewart.

In~\cref{chap:unified:lb}, we complement these algorithmic frameworks by describing new general approaches to obtaining information-theoretic \emph{lower bounds} in distribution testing.~\cref{sec:learningreductions}, based on~\cite{CDGR:16}, describes a reduction technique which allows to lift hardness of testing a sub-property $\property'\subseteq\property$ to that of testing $\property$ itself, modulo a mild learnability condition on the latter. As a corollary, we obtain new (as well as previously known) lower bounds for many distribution classes, in a clean and unified way.~\cref{sec:communication} (based on the paper~\cite{BCG:17} with Eric Blais and Tom Gur) then provides another framework  to easily establish distribution testing lower bounds, this time by carrying over lower bounds from \emph{communication complexity}. We show how this reduction from communication complexity, besides enabling us to easily derive lower bounds for a variety of distribution testing questions, can also shed light on existing results, leading to an unexpected connection between distribution testing and the seemingly unrelated field of interpolation theory.

In these two chapters, we were concerned with the ``standard'' setting of distribution testing, which only assumes access to independent samples; and developed general methods to tackle questions in this setting. In~\cref{chap:newmodels}, we take a different path: instead of finding new strategies to play the game, we change the \emph{rules} themselves -- granting the testing algorithms a more powerful type of access to the unknown distribution. Based on a work with Dana Ron and Rocco Servedio~\cite{CRS:15},~\cref{sec:conditional} introduces and studies the \emph{conditional sampling model}, in which the algorithm can get samples from the underlying probability distribution conditioned on subsets of events of its choosing. In~\cref{sec:extended}, we define and study two different settings, the \emph{dual access} and \emph{cumulative dual access} models, in which one can both draw independent samples from the distribution and query the value on any point of the domain of either its probability mass function or cumulative distribution function. (This is based on the paper~\cite{CR:14}, with Ronitt Rubinfeld.) Both sections thus consider testing algorithms that are at least as powerful as those from the standard sampling setting; the question is to quantify \emph{how much} more powerful these algorithms can be, and what limitations remain.

Finally, in~\cref{chap:correction} we venture out of property testing to explore a different -- albeit related -- paradigm: that of distribution \emph{correcting}. Changing now the \emph{goal} of the game, we introduce the notion of sampling corrector: granted access to independent samples from a probability distribution only \emph{close} to having some property $\property$ of interest, one must provide access to samples from a ``corrected'' distribution which, while still being close to the original distribution, does satisfy $\property$. We prove general results on this new algorithmic primitive, and study its relation to both distribution learning and testing; before focusing specifically on correction of a well-studied property of distributions, monotonicity. This last chapter contains material from~\cite{CGR:16}, joint work with Themis Gouleakis and Ronitt Rubinfeld.

\chapter{Testing Classes of Distributions: Upper Bounds from Swiss Army Knives}\label{chap:unified:ub}

\epigraph{``Should we meet with a Jubjub, that desperate bird,\\
We shall need all our strength for the job!''}{Lewis Carroll, \textit{The Hunting of the Snark}}

%%%%%%%%%%%%%%%%%%%%%%%%%%%%%%%%%%%%%%%%%%%%%%%%%%%
\newcounter{IRL}
\renewcommand{\theIRL}{\textsf{Succinctness}}
\newcommand*{\inlineref}[1]{\refstepcounter{IRL}({\theIRL})\label{#1}}
\newcommand{\RightComment}[1]{\Comment{\parbox[t]{.30\linewidth}{\small#1}}} % For algorithms
%%%%%%%%%%%%%%%%%%%%%%%%%%%%%%%%%%%%%%%%%%%%%%%%%%%

In this chapter, we focus on obtaining algorithmic upper bounds in distribution testing. Our goal, however, departs from most of the previous literature, in that it is not to solve a specific testing problem by coming up with a bespoke\cmargin{Idiomatic? Tailor-made?}{} algorithm for that task. We take instead a different path, and set out to provide \emph{general} algorithms applicable across-the-board to a variety of problems \emph{at once}. Marginally more formally, here is the objective we will address:
\begin{problem}
Design general-purpose testing algorithms, that when applied to a property \property would have (tight, or near-tight, or not absolutely laughable) sample complexity $q(\eps, \tau)$ as long as \property satisfies some ``structural assumption'' $\mathcal{S}_\tau$ parameterized by $\tau$.
\end{problem}
\noindent For instance, one could think of $\mathcal{S}_\tau$ as ``every $\D\in\property$ is close to some piecewise-constant distribution on $1/\tau$ intervals'' (as is e.g. the case for monotone distributions, with $\tau=\frac{1}{\poly(\log n, 1/\eps)}$ by~\cref{theorem:Birge:obl:decomp}); or, in another vein, ``all $\D\in\property$ have $\norm{\D}_{17/4} \leq \tau$'' (technically, one \emph{could} think of that one).  

We make significant progress in this direction by providing two unified frameworks for the question of testing various properties of probability distributions. First, we describe in~\cref{sec:shaperestrictions} a meta-algorithm to test membership in any distribution class, particularly well-suited to any class (including monotone, log-concave, $t$-modal, piecewise-polynomial, and Poisson Binomial distributions) which satisfies a ``shape constraint.'' (Broadly speaking, whenever any distribution in the class is well-approximated, in a strong $\lp[2]$-type sense, by a piecewise-constant distribution on relatively few pieces.)

In~\cref{sec:fourier}, we detail our second general technique, based on an entirely different type of structural assumption. Namely, this approach now leverages purported properties of the \emph{Fourier transform} of the distributions, and performs particularly well for those classes containing distributions with sparse Fourier transform -- such as, for instance, the classes of Poisson Binomial distributions and SIIRVs.

Our two frameworks yield near-sample-optimal and computationally efficient testers for a wide range of distribution families; for most of these, we provide the first non-trivial tester in the literature.

%%%%%%%%%%%%%%%%%%%%%%%%%%%%%%%%%%%%%%%%%%%%%%%%%%%%%%%%%%%%%%%%%%%%%%%%%%%%%%%%%%%%%%%%%%%%%%%%%%%%%%%%%%%%%%%%%%%%%%%%%%%%%%%%%%%%%%%%%%%%%%%%%%%%%%%

\section{The Shape Restrictions Knife}\label{sec:shaperestrictions}
\subsection{Introduction}\label{sec:introduction:shaperestrictions}
In many situations, it is natural to assume that this data exhibits some simple structure because of known properties of the origin of the data, and in fact these assumptions are crucial in making the problem tractable. Such assumptions translate as constraints on the probability distribution -- e.g., it is supposed to be Gaussian, or to meet a smoothness or ``fat tail'' condition (see e.g.,~\cite{Mandelbrot:63:FatTail,Hougaard:86:StableDistribs,PhysRevLett:95}).

As a result, the problem of deciding whether a distribution possesses such a structural property has been widely investigated both in theory and practice, in the context of \emph{shape restricted inference}~\cite{BBBB:72,SS:01} and \emph{model selection}~\cite{MP:03}. Here, it is guaranteed or thought that the unknown distribution satisfies a shape constraint, such as having a monotone or log-concave probability density function~\cite{SN:99,BB:05,Wal:09,Diakonikolas:CRC}.
% From a different perspective, a recent line of work in Theoretical Computer Science, originating from the papers of Batu et al.~\cite{BFRSW:00,BFFKRW:01,GRexp:00} has also been tackling similar questions in the setting of property testing (see~\cite{Ron:08,Ron:10,Rubinfeld:12:Survey,Canonne:15:Survey} for surveys on this field). This very active area has seen a spate of results and breakthroughs over the past decade, culminating in very efficient (both sample and time-wise) algorithms for a wide range of distribution testing problems~\cite{BDKR:05,GMV:06,Alon:2007,DDSVV:13,CDVV:14,AD:15,DKN:15}. In many cases, this led to a tight characterization of the number of samples required for these tasks as well as the development of new tools and techniques, drawing connections to learning and information theory~\cite{ValiantValiant:10lb,VV:11:stoc,VV:14,DK:16}.
%
% In this paper, we focus on the following general property testing problem: given a class (property) of distributions \property and sample access to an \emph{arbitrary} distribution $\D$, one must distinguish between the case that \textsf{(a)} $\D\in\property$, versus \textsf{(b)} $\normone{\D-\D^\prime} >\eps$ for all $\D^\prime\in\property$ (i.e., $\D$ is either in the class, or far from it). While many of the previous works have focused on the testing of specific properties of distributions or obtained algorithms and lower bounds on a case-by-case basis, an emerging trend in distribution testing is to design general frameworks that can be applied to \emph{several} property testing problems~\cite{Valiant:11,VV:11:stoc, DKN:15, DKN:15:FOCS}. This direction, the testing analogue of a similar movement in distribution learning~\cite{CDSS:13,CDSS:14:NIPS,CDSS:14,ADLS:15}, aims at abstracting the minimal assumptions that are shared by a large variety of problems, and giving algorithms that can be used for any of these problems. 

In this work, we consider this decision question from the Theoretical Computer Science viewpoint, namely in the context of distribution testing. We provide a unified framework for the question of testing various ``shape restricted'' properties of probability distributions -- more specifically, we describe a generic technique to obtain upper bounds on the sample complexity of this question, which applies to a broad range of structured classes. Our technique yields sample near-optimal and computationally efficient testers for a wide range of distribution families. Conversely, we also develop a general approach to prove lower bounds on these sample complexities, and use it to derive tight or nearly tight bounds for many of these classes. (This lower bound approach will be covered in the next chapter,~\cref{sec:learningreductions}.)

\subparagraph{Related work} Batu et al.~\cite{BKR:04} initiated the study of efficient property testers for monotonicity and obtained (nearly) matching upper and lower bounds for this problem; while~\cite{AD:15} later considered testing the class of Poisson Binomial Distributions, and settled the sample complexity of this problem (up to the precise dependence on $\eps$). Indyk, Levi, and Rubinfeld~\cite{ILR:12}, focusing on distributions that are piecewise constant on $t$ intervals (``$t$-histograms'') described a $\tilde{O}(\sqrt{tn}/\eps^5)$-sample algorithm for testing membership to this class. Another body of work by~\cite{BDKR:05},~\cite{BKR:04}, and~\cite{DDSVV:13} shows how assumptions on the shape of the distributions can lead to significantly more efficient algorithms. They describe such improvements in the case of identity and closeness testing as well as for entropy estimation, under monotonicity or $k$-modality constraints. Specifically, Batu et al. show in~\cite{BKR:04} how to obtain a $O\big({\log^3 n/\eps^3}\big)$-sample tester for closeness in this setting, in stark contrast to the $\Omega\big({{n}^{2/3}}\big)$ general lower bound. Daskalakis et al.~\cite{DDSVV:13} later gave ${O}(\sqrt{\log n})$ and ${O}({\log^{2/3} n})$-sample testing algorithms for testing respectively identity and closeness of monotone distributions, and obtained similar results for $k$-modal distributions. Finally, we briefly mention two related results, due respectively to~\cite{BDKR:05} and~\cite{DDS:12}. The first one states that for the task of getting a multiplicative \emph{estimate} of the entropy of a distribution, assuming monotonicity enables exponential savings in sample complexity -- $O\big({\log^6 n}\big)$, instead of $\bigOmega{n^c}$ for the general case. The second describes how to test if an unknown $k$-modal distribution is in fact monotone, using only $\bigO{k/\eps^2}$ samples. Note that the latter line of work differs from ours in that it \emph{presupposes} the distributions satisfy some structural property, and uses this knowledge to test something else about the distribution; while we are given \textit{a priori} arbitrary distributions, and must \emph{check} whether the structural property holds. Except for the properties of monotonicity and being a PBD, nothing was previously known on testing the shape restricted properties that we study.

Moreover, for the specific problems of identity and closeness testing,\footnote{Recall that the identity testing problem asks, given the explicit description of a distribution $\D^\ast$ and sample access to an unknown distribution $\D$, to decide whether $\D$ is equal to $\D^\ast$ or far from it; while in closeness testing both distributions to compare are unknown.} recent results of~\cite{DKN:15,DKN:15:FOCS} describe a general algorithm which applies to a large range of shape or structural constraints, and yields optimal identity testers for classes of distributions that satisfy them. We observe that while the question they answer can be cast as a specialized instance of membership testing, our results are incomparable to theirs, both because of the distinction above (testing \emph{with} versus testing \emph{for} structure) and as the structural assumptions they rely on are fundamentally different from ours.

\subparagraph{Concurrent and followup work} 
Independently and concurrently to this work, Acharya, Daskalakis, and Kamath~\cite{ADK:15} obtained a sample near-optimal efficient algorithm for testing log-concavity, as well as sample-optimal algorithms for testing the classes of monotone, unimodal, and monotone hazard rate distributions (along with matching lower bounds on the sample complexity of these tasks). Their work builds on ideas from~\cite{AD:15} and their techniques are orthogonal to ours: namely, while at some level both works follow a ``testing-by-learning'' paradigm, theirs rely on first learning in the (more stringent) $\chi^2$ distance, then applying a testing algorithm which is robust to some amount of noise (i.e., tolerant testing) in this $\chi^2$ sense (as opposed to noise in an $\lp[1]$ sense, which is known to be impossible without a near-linear number of samples~\cite{ValiantValiant:10lb}).

Subsequent to the publication of the conference version of this work,~\cite{Canonne:16} improved on both~\cite{ILR:12} and our results for the specific class of $t$-histograms, providing nearly tight upper and lower bounds on testing membership to this class. Specifically, it obtains an upper bound of $\tilde{O}(\sqrt{n}/\eps^2 + t/\eps^3)$, complemented with an $\Omega(\sqrt{n}/\eps^2+t/(\eps\log t))$ lower bound on the sample complexity.

Building on our work, Fischer, Lachish, and Vasudev recently generalized in~\cite{FischerLV:16} our approach and algorithm to the \emph{conditional sampling model} of~\cite{CFGM:13,CRS:15}, obtaining analogues of our testing results in this different setting of distribution testing where the algorithm is allowed to condition the samples it receives on subsets of the domain of its choosing. In the ``standard'' sampling setting,~\cite{FischerLV:16} additionally provides an alternative to the first subroutine of our testing algorithm: this yields a simpler and non-recursive algorithm, with a factor $\log n$ shaved off at the price of a worse dependency on the distance parameter $\eps$. (Namely, their sample complexity is dominated by $O(\sqrt{nL}\log^2(1/\eps)/{\eps^5})$, to be compared to the $O(\sqrt{nL}\log n/{\eps^3})$ term of~\cref{theo:main:testing:detailed}.)

\subsubsection{Results and Techniques}

A natural way to tackle our membership testing problem would be to first learn the unknown distribution $\D$ \emph{as if} it satisfied the property, before checking if the hypothesis obtained is indeed both close to the original distribution and to the property. Taking advantage of the purported structure, the first step could presumably be conducted with a small number of samples; things break down, however, in the second step. 
Indeed, most approximation results leading to the improved learning algorithms one would apply in the first stage only provide very weak guarantees, that is in the $\lp[1]$ sense only. For this reason, they lack the robustness that would be required for the second part, where it becomes necessary to perform \emph{tolerant} testing between the hypothesis and $\D$ -- a task that would then entail a number of samples almost linear in the domain size. To overcome this difficulty, we need to move away from these global $\lp[1]$ closeness results and instead work with stronger requirements, this time in $\lp[2]$ norm. 

At the core of our approach is an idea of Batu et al.~\cite{BKR:04}, which show that monotone distributions can be well-approximated (in a certain technical sense) 
by piecewise constant densities on a suitable interval partition of the domain; and leverage this fact to reduce monotonicity testing to uniformity testing on each interval of this partition. 
While the argument of~\cite{BKR:04} is tailored specifically for the setting of monotonicity testing, we are able to abstract the key ingredients, and obtain a generic membership tester that applies to a wide range of distribution families. In more detail, we provide a testing algorithm which applies to any class of distributions which admit succinct approximate decompositions -- that is, each distribution in the class can be well-approximated (in a strong $\lp[2]$ sense) by piecewise constant densities on a small number of intervals (we hereafter refer to this approximation property, formally defined in~\cref{def:struct:dec:split}, as \inlineref{label:struct:criterion}; and extend the notation to apply to any \emph{class} \class of distributions for which all $\D\in\class$ satisfy~\eqref{label:struct:criterion}). 
Crucially, the algorithm does not care about \emph{how} these decompositions can be obtained: 
for the purpose of testing these structural properties we only need to establish their \emph{existence}. Specific examples are given in the corollaries below.
Informally, our main algorithmic result, informally stated (see~\cref{theo:main:testing:detailed} for a detailed formal statement), is as follows:

\begin{restatable}[Main Theorem]{theorem}{mainthmtestingalgo}\label{theo:main:testing}
There exists an algorithm \textsc{TestSplittable} which, given sampling access to an unknown distribution $\D$ over $[n]$ and parameter $\eps\in(0,1]$, can distinguish with probability $2/3$ between \textsf{(a)} $\D\in\property$ versus \textsf{(b)} $\lp[1](\D, \property) > \eps$, for \emph{any} property $\property$ that satisfies the above natural structural criterion \eqref{label:struct:criterion}. Moreover, for many such properties this algorithm is computationally efficient, and its sample complexity is optimal (up to logarithmic factors and the exact dependence on $\eps$).
\end{restatable} 

\noindent We then instantiate this result to obtain ``out-of-the-box'' \emph{computationally efficient} testers for several classes of distributions, by showing that they satisfy the premise of our theorem (the definition of these classes is given in~\cref{ssec:class:definitions}):

\begin{corollary}\label{coro:main:testing}
The algorithm \textsc{TestSplittable} can test the classes of monotone, unimodal, log-concave, concave, convex, and monotone hazard rate (MHR) distributions, with $\tildeO{\sqrt{n}/\eps^{7/2}}$ samples.
\end{corollary}
\begin{corollary}\label{coro:main:testing:tmod}
The algorithm \textsc{TestSplittable} can test the class of $t$-modal distributions, with $\tildeO{\sqrt{tn}/\eps^{7/2}}$ samples. \end{corollary}

\begin{corollary}\label{coro:main:testing:piecewise}
The algorithm \textsc{TestSplittable} can test the classes of $t$-histograms and $t$-piecewise degree-$d$ distributions, with $\tildeO{\sqrt{tn}/\eps^3}$ and $\tildeO{\sqrt{t(d+1)n}/\eps^{7/2} + t(d+1)/\eps^3}$ samples respectively. \end{corollary}

\begin{corollary}\label{coro:main:testing:pbd}
The algorithm \textsc{TestSplittable} can test the classes of Binomial and Poisson Binomial Distributions, with $\tildeO{{n}^{1/4}/\eps^{7/2}}$ samples.
\end{corollary}

\newcommand{\pb}[2]{\parbox[c][][c]{#1}{\strut#2\strut}}
  \begin{table}[ht]\centering\footnotesize
    \begin{adjustwidth}{-.75in}{-.5in}\centering
  \begin{tabular}{@{}|l|c|c|@{}}\hline
    { \bf Class }& \textbf{Upperbound} & \bf Lowerbound\\\hline
     {Monotone}  & {$\tildeO{\frac{\sqrt{n}}{\eps^6}}$~\cite{BKR:04}, $\tildeO{\frac{\sqrt{n}}{\eps^{7/2}}}$ (\cref{coro:main:testing})}, $\bigO{\frac{\sqrt{n}}{\eps^{2}}}$~\cite{ADK:15}$^{(\ddagger)}$ 
                 & {$\bigOmega{\frac{\sqrt{n}}{\eps^2}}$~\cite{BKR:04}, $\bigOmega{\frac{\sqrt{n}}{\eps^2}}$ (\cref{coro:lb:sqrtn})} \\\hline
     {Unimodal}  & {$\tildeO{\frac{\sqrt{n}}{\eps^{7/2}}}$ (\cref{coro:main:testing})}, $\bigO{\frac{\sqrt{n}}{\eps^{2}}}$~\cite{ADK:15}$^{(\ddagger)}$ 
                 & {$\bigOmega{\frac{\sqrt{n}}{\eps^2}}$ (\cref{coro:lb:sqrtn})} \\\hline
     {$t$-modal}  & {$\tildeO{\frac{\sqrt{{t}n}}{\eps^{7/2}}}$ (\cref{coro:main:testing:tmod})}
                 & {$\bigOmega{\frac{\sqrt{n}}{\eps^2}}$ (\cref{coro:lb:sqrtn})} \\\hline
     \pb{30mm}{Concave, convex}  & {$\tildeO{\frac{\sqrt{n}}{\eps^{7/2}}}$ (\cref{coro:main:testing})}
                 & {$\bigOmega{\frac{\sqrt{n}}{\eps^2}}$ (\cref{coro:lb:sqrtn})} \\\hline
      \pb{30mm}{Log-concave}  & {$\tildeO{\frac{\sqrt{n}}{\eps^{7/2}}}$ (\cref{coro:main:testing})}, $\bigO{\frac{\sqrt{n}}{\eps^{2}}}$~\cite{ADK:15}$^{(\ddagger)}$ 
                 & {$\bigOmega{\frac{\sqrt{n}}{\eps^2}}$ (\cref{coro:lb:sqrtn})} \\\hline
     \pb{30mm}{Monotone Hazard Rate (MHR)}  & {$\tildeO{\frac{\sqrt{n}}{\eps^{7/2}}}$ (\cref{coro:main:testing})}, $\bigO{\frac{\sqrt{n}}{\eps^{2}}}$~\cite{ADK:15}$^{(\ddagger)}$ 
                 & {$\bigOmega{\frac{\sqrt{n}}{\eps^2}}$ (\cref{coro:lb:sqrtn})} \\\hline
     \pb{30mm}{Binomial, Poisson Binomial (PBD)}  & \pb{90mm}{\centering $\tildeO{\frac{{n}^{1/4}}{\eps^2} + \frac{1}{\eps^6}}$~\cite{AD:15},\\ $\tildeO{\frac{{n}^{1/4}}{\eps^{7/2}}}$ (\cref{coro:main:testing:pbd})}
                 & {$\bigOmega{\frac{{n}^{1/4}}{\eps^2}}$~(\cite{AD:15},~\cref{coro:lb:pbd})} \\\hline
     \pb{30mm}{$t$-histograms}  & \pb{65mm}{ $\tildeO{\frac{\sqrt{tn}}{\eps^5}}$~\cite{ILR:12}, $\tildeO{\frac{\sqrt{n}}{\eps^2} +\frac{t}{\eps^3}}$~\cite{Canonne:16}$^{(\ddagger)}$,\\ $\tildeO{\frac{\sqrt{tn}}{\eps^3}}$ (\cref{coro:main:testing:piecewise}) }
                 & \pb{50mm}{$\bigOmega{\sqrt{tn}}$ (for $t\leq \frac{1}{\eps}$)~\cite{ILR:12}, $\bigOmega{\frac{\sqrt{n}}{\eps^2}+\frac{t}{\eps}}$~\cite{Canonne:16}$^{(\ddagger)}$, \\ $\bigOmega{\frac{\sqrt{n}}{\eps^2}}$ (\cref{coro:lb:sqrtn})}  \\\hline
     \pb{30mm}{$t$-piecewise degree-$d$}  & { $\tildeO{\frac{\sqrt{t(d+1)n}}{\eps^{7/2}} + \frac{t(d+1)}{\eps^3} }$ (\cref{coro:main:testing:piecewise}) }
                 & {$\bigOmega{\frac{\sqrt{n}}{\eps^2}}$ (\cref{coro:lb:sqrtn})}  \\\hline
     \pb{30mm}{$(n,k)$-SIIRV}  & { $\bigO{\frac{{k}^{1/2}{n}^{1/4}}{\eps^2}\log^{1/4}\frac{1}{\eps}}+\tildeO{\frac{k^2}{\eps^2}}$ (\cref{theo:testing:ksiirv},~\cite{CanonneDS:17})$^{(\ddagger)}$ }
                 & {$\bigOmega{ \frac{{k}^{1/2}{n}^{1/4}}{\eps^2} }$ (\cref{coro:lb:ksiirv})} \\\hline
  \end{tabular}
  \end{adjustwidth}
\caption{\label{fig:table:secshaperestrictions:results} Summary of results obtained \emph{via} our first general class testing framework (\cref{theo:main:testing}). The corresponding lower bounds stated in this table originate from the technique covered in the next chapter (specifically,~\cref{sec:learningreductions}); while the symbol $(\ddagger)$ indicates a result independent of or subsequent to our work.}
  \end{table}
  

We remark that the aforementioned sample upper bounds are information-theoretically near-optimal in the domain size $n$ (up to logarithmic factors). See~\cref{fig:table:secshaperestrictions:results} and the following chapter for the corresponding lower bounds. We did not attempt to optimize the dependence on the parameter $\eps$,
though a more careful analysis might lead to such improvements.

We stress that prior to our work, no non-trivial testing bound was known for most of these classes~--~specifically, our nearly-tight bounds for
$t$-modal with $t>1$, log-concave, concave, convex, MHR, and piecewise polynomial distributions are new. Moreover, although a few of our applications 
were known in the literature (the $\tildeO{\sqrt{n}/\eps^6}$ upper and $\bigOmega{\sqrt{n}/\eps^2}$ lower bounds on testing monotonicity can be found in~\cite{BKR:04}, while the $\Theta\big({n^{1/4}}\big)$ sample complexity of testing PBDs was recently given\footnotemark{} in~\cite{AD:15}, and the task of testing $t$-histograms is considered in~\cite{ILR:12}), the crux here is that we are able to derive them in a \emph{unified} way, by applying the same generic algorithm to all these different distribution families. 
We note that our upper bound for $t$-histograms (\cref{coro:main:testing:piecewise}) also significantly improves on the previous $\tildeO{\sqrt{tn}/\eps^5}$-sample tester with regard to the dependence on the proximity parameter $\eps$. In addition to its generality, our framework yields much cleaner and conceptually simpler proofs of the upper and lower bounds from~\cite{AD:15}.
\footnotetext{For the sample complexity of testing monotonicity,~\cite{BKR:04} originally states an $\tildeO{{\sqrt{n}}/{\eps^4}}$ upper bound, but the proof seems to only result in an $\tildeO{{\sqrt{n}}/{\eps^6}}$ bound. Regarding the class of PBDs,~\cite{AD:15} obtain an ${n^{1/4}}\cdot\tilde{O}\big({1/\eps^2}\big) + \tilde{O}\big({1/\eps^6}\big)$ sample complexity, to be compared with our $\tilde{O}\big({n^{1/4}/\eps^{7/2}}) + \bigO{\log^4 n/\eps^4 }$ upper bound; as well as an $\Omega\big({n^{1/4}/\eps^2}\big)$ lower bound.}

\subparagraph{Lower Bounds} To complement our upper bounds, we also give a generic framework for proving lower bounds against testing classes of distributions. While this framework will be the focus of the next chapter, we state here some of the results it enables us to derive for specific structured distribution families; the reader is referred to~\cref{sec:learningreductions} (and specifically~\cref{theo:main:testing:lb}) for the details and formal statement of our reduction-based lower bound theorem.

\begin{restatable}{corollary}{corolbsqrtn}\label{coro:lb:sqrtn}
  Testing log-concavity, convexity, concavity, MHR, unimodality, $t$-modality, $t$-histograms, and $t$-piecewise degree-$d$ distributions each require $\bigOmega{{\sqrt{n}}/{\eps^2}}$ samples (the last three for $t = o(\sqrt{n})$ and $t(d+1) = o(\sqrt{n})$, respectively), for any $\eps\geq 1/n^{O(1)}$.
\end{restatable}

\begin{restatable}{corollary}{corolbpbd}\label{coro:lb:pbd}
  Testing the classes of Binomial and Poisson Binomial Distributions each require $\bigOmega{{n^{1/4}}/{\eps^2}}$ samples, for any $\eps\geq 1/n^{O(1)}$.
\end{restatable}

\begin{restatable}{corollary}{corolbksiirv}\label{coro:lb:ksiirv}
  There exist absolute constants $c>0$ and $\eps_0 > 0$ such that testing the class of $(n,k)$-SIIRV distributions requires $\Omega\big( k^{1/2}n^{1/4}/\eps^2 \big)$ samples, for any $k=\littleO{n^c}$ and ${1}/{n^{O(1)}} \leq \eps \leq \eps_0$.
\end{restatable}

\subparagraph{Tolerant Testing} 
Using our techniques, we also establish nearly--tight upper and lower bounds on tolerant testing\footnotetext{\emph{Tolerant testing} of a property \property is defined as follows: given $0 \leq \eps_1 < \eps_2 \leq 1$, one must distinguish between \textsf{(a)} $\lp[1](\D,\property) \leq \eps_1$ and \textsf{(b)} $\lp[1](\D,\property) \geq \eps_2$. This turns out to be, in general, a much harder task than that of ``regular'' testing (where we take $\eps_1=0$).} for shape restrictions. 
Similarly, our upper and lower bounds are matching as a function of the domain size.
More specifically, we give a simple generic upper bound approach (namely, a learning followed by tolerant testing algorithm).
Our tolerant testing lower bounds follow the same reduction-based approach as in the non-tolerant case, and will be covered in the next chapter,~\cref{chap:unified:lb}. In more detail, our results are as follows (see~\cref{sec:toltesting:ub} for the upper bounds, and further down the road~\cref{sec:learningreductions} for the lower bounds):

\begin{restatable}{corollary}{coromaintoltestingmlogm}\label{coro:main:tol:testing:mlogm}
Tolerant testing of log-concavity, convexity, concavity, MHR, unimodality, and $t$-modality can be performed with $O\big( \frac{1}{(\eps_2-\eps_1)^2}\frac{n}{\log n} \big)$ samples, for $\eps_2 \geq C \eps_1$ (where $C>2$ is an absolute constant).
\end{restatable}

\begin{restatable}{corollary}{coromaintoltestingpbd}\label{coro:main:tol:testing:pbd}
Tolerant testing of the classes of Binomial and Poisson Binomial Distributions can be performed with $O\big( \frac{1}{(\eps_2-\eps_1)^2}\frac{\sqrt{n\log({1}/{\eps_1})}}{\log n} \big)$ samples, for $\eps_2 \geq C \eps_1$ (where $C>2$ is an absolute constant).
\end{restatable}

\begin{restatable}{corollary}{corolbtolnlogn}\label{coro:tol:lb:nlogn}
  Tolerant testing of log-concavity, convexity, concavity, MHR, unimodality, and $t$-modality each require $\bigOmega{\frac{1}{(\eps_2-\eps_1)}\frac{n}{\log n}}$ samples (the latter for $t = o(n)$).
\end{restatable}

\begin{restatable}{corollary}{corolbtolpbd}\label{coro:lb:tol:pbd}
  Tolerant testing of the classes of Binomial and Poisson Binomial Distributions each require $\bigOmega{\frac{1}{(\eps_2-\eps_1)}\frac{\sqrt{n}}{\log n}}$ samples.
\end{restatable}

\subparagraph{On the scope of our results} We point out that our main theorem is likely to apply to many other classes of structured distributions, 
due to the mild structural assumptions it requires. However, we did not attempt here to be comprehensive; but rather to illustrate the generality of our approach. Moreover, for all properties considered in this paper the generic upper and lower bounds we derive through our methods turn out to be optimal up to at most polylogarithmic factors (with regard to the support size). The reader is referred to~\cref{fig:table:secshaperestrictions:results} for a summary of our results and related work.

\subsubsection{Organization of the Paper}
We begin by establishing our main result, the proof of~\cref{theo:main:testing} (our general testing algorithm), in~\cref{sec:algorithm}. 
In~\cref{sec:structural}, we establish the necessary structural theorems for each class of distributions considered, enabling us to derive the upper bounds of~\cref{fig:table:secshaperestrictions:results}.~\cref{sec:effectivesupport} introduces a slight modification of our algorithm which yields stronger testing results for classes of distributions with small effective support, and use it to derive~\cref{coro:main:testing:pbd}, our upper bound for Poisson Binomial distributions. (The details of our lower bound methodology and of its applications to the classes of~\cref{fig:table:secshaperestrictions:results}, however, are deferred to~\cref{chap:unified:lb}.) Finally,~\cref{sec:lowerbounds:tol} is concerned with the extension of this methodology to \emph{tolerant} testing, of which~\cref{sec:toltesting:ub} describes a generic upper bound counterpart.
 

\subsection{The General Algorithm}\label{sec:algorithm}
In this section, we obtain our main result, restated below:
\mainthmtestingalgo*

\subparagraph{Intuition} Before diving into the proof of this theorem, we first provide a high-level description of the argument. The algorithm proceeds in 3 stages: the first, the \emph{decomposition step}, attempts to recursively construct a partition of the domain in a small number of intervals, with a very strong guarantee. If the decomposition succeeds, then the unknown distribution $\D$ will be close (in $\lp[1]$ distance) to its ``flattening'' on the partition; while if it fails (too many intervals have to be created), this serves as evidence that $\D$ does not belong to the class and we can reject. The second stage, the \emph{approximation step}, then learns this flattening of the distribution -- which can be done with few samples since by construction we do not have many intervals. The last stage is purely computational, the \emph{projection step}: where we verify that the flattening we have learned is indeed close to the class \class. If all three stages succeed, then by the triangle inequality it must be the case that $\D$ is close to \class; and by the structural assumption on the class, if $\D\in\class$ then it will admit succinct enough partitions, and all three stages will go through.\medskip

\noindent Turning to the proof, we start by defining formally the ``structural criterion'' we shall rely on, before describing the algorithm at the heart of our result in~\cref{ssec:main:algorithm}. (We note that a modification of this algorithm will be described in~\cref{sec:effectivesupport}, and will allow us to derive~\cref{coro:main:testing:pbd}.)

\begin{definition}[Decompositions]\label{def:struct:dec:split}
Let $\gamma,  \zeta > 0$ and $L=L(\gamma,\zeta,n)\geq 1$.  A class of distributions \class on $[n]$ is said to be \emph{$(\gamma,\zeta,L)$-decomposable} if for every $\D\in\class$ there exists $\ell \leq L$ and a partition $\mathcal{I}(\gamma,\zeta,\D)=(I_1,\dots,I_\ell)$ of the interval $[1,n]$ such that, for all $j\in[\ell]$, one of the following holds:
\begin{enumerate}[(i)]
  \item\label{def:struct:item:light}  $\D(I_j) \leq \frac{\zeta}{L}$; or 
  \item\label{def:struct:item:flat} $\displaystyle \max_{i \in I_j} \D(i)\leq  (1+\gamma)\cdot \min_{i \in I_j} \D(i)$.
\end{enumerate}
Further, if $\mathcal{I}(\gamma,\zeta,\D)$ is \emph{dyadic} (i.e., each $I_k$ is of the form $[j\cdot 2^i+1,(j+1)\cdot 2^i]$ for some integers $i,j$, corresponding to the leaves of a recursive bisection of $[n]$), then \class is said to be \emph{$(\gamma,\zeta,L)$-splittable}.
\end{definition}

\begin{lemma}\label{lemma:decomposable:splittable}
If \class is $(\gamma,\zeta,L)$-decomposable, then it is $(\gamma, \zeta, L')$-splittable for $L'(\gamma,\zeta,n) = O(\log n)\cdot L(\gamma,\frac{\zeta}{2(\log n+1)},n)$.
\end{lemma}
\begin{proof}
We will begin by proving a claim that for every partition $\mathcal{I}=\{I_1,I_2,...,I_L\}$ of the interval $[1,n]$ into $L$ intervals, there exists a refinement of that partition which consists of at most $L\cdot O(\log n)$ dyadic intervals. So, it suffices to prove that every interval $[a,b]\subseteq [1,n]$ can be partitioned in at most $\bigO{\log n}$ dyadic intervals. Indeed, let $\ell$ be the largest integer such that $2^\ell\leq \frac{b-a}{2}$ and let $m$ be the smallest integer such that $m\cdot 2^\ell\geq a$. If follows that $m\cdot 2^\ell\leq a+\frac{b-a}{2}=\frac{a+b}{2}$ and $(m+1)\cdot 2^\ell\leq b$. So, the interval $I=[m\cdot 2^\ell+1,(m+1)\cdot 2^\ell]$ is fully contained in $[a,b]$ and has size at least $\frac{b-a}{4}$. 

We will also use the fact that, for every $\ell^\prime\leq \ell$,
\begin{equation}\label{eq:lem:dec:split:smaller:powers}
m\cdot 2^\ell=m\cdot 2^{\ell-\ell^\prime}\cdot 2^{\ell^\prime}=m^\prime \cdot 2^{\ell^\prime}
\end{equation} 

Now consider the following procedure: Starting from right (resp. left) side of the interval $I$, we add the largest interval which is adjacent to it and fully contained in $[a,b]$ and recurse until we cover the whole interval $[(m+1)\cdot 2^\ell+1,b]$ (resp. $[a,m\cdot 2^\ell]$). Clearly, at the end of this procedure, the whole interval $[a,b]$ is covered by dyadic intervals. It remains to show that the procedure takes $\bigO{\log n}$ steps. Indeed, using~\cref{eq:lem:dec:split:smaller:powers}, we can see that at least half of the remaining left or right interval is covered in each step (except maybe for the first 2 steps where it is at least a quarter). Thus, the procedure will take at most $2\log n +2=\bigO{\log n}$ steps in total. From the above, we can see that each of the $L$ intervals of the partition $\mathcal{I}$ can be covered with $\bigO{\log n}$ dyadic intervals, which completes the proof of the claim. 

In order to complete the proof of the lemma, notice that the two conditions in~\cref{def:struct:dec:split} are closed under taking subsets: so that the second is immediately verified, while for the first we have that for any of the ``new'' intervals $I$ that $\D(I) \leq  \frac{\zeta}{L} \leq \frac{\zeta\cdot (2\log n+2)}{L'}$. 
\end{proof}

\subsubsection{The algorithm}\label{ssec:main:algorithm}

\cref{theo:main:testing}, and with it~\cref{coro:main:testing},~\cref{coro:main:testing:tmod}, and~\cref{coro:main:testing:piecewise} will follow from the theorem below, combined with the structural theorems from~\cref{sec:structural}:
\begin{theorem}\label{theo:main:testing:detailed}
  Let $\class$ be a class of distributions over $[n]$ for which the following holds. 
    \begin{enumerate}
      \item $\class$ is $(\gamma, \zeta,L(\gamma,\zeta,n))$-splittable;
      \item there exists a procedure $\estimdist$ which, given as input a parameter $\alpha\in(0,1)$ and the explicit description of a distribution $\D$ over $[n]$, 
      returns \yes if the distance $\lp[1](\D,\class)$ to $\class$ is at most $\alpha/10$, and \no if $\lp[1](\D,\class) \geq 9\alpha/10$ (and either \yes or \no otherwise).
    \end{enumerate}
  Then the algorithm \textsc{TestSplittable} (\cref{algo:test:splittable}) is a $\bigO{ \max\mleft({\sqrt{nL}}\log n/{\eps^3}, L/{\eps^2}\mright) }$-sample tester for $\class$, for $L=L(O(\eps),O(\eps),n)$. 
  (Moreover, if $\estimdist$ is computationally efficient, then so is \textsc{TestSplittable}.)
\end{theorem}
%  Then, if we set $\gamma=\zeta=\varepsilon$, the algorithm \textsc{TestSplittable} (\cref{algo:test:splittable}) is a $\bigO{ \max\mleft({\sqrt{nL}}\log n/{\eps^3}, L/{\eps^2}\mright) }$-sample tester for $\class$. %, for $L=L(\eps,\eps,n)$. 

\begin{algorithm}[h!t]
  \algblock[block]{Start}{Start}
  \algblockdefx[]{Start}{End}    [1]{\textsc{#1}}    [1][]{\textsc{#1}}
  \begin{algorithmic}[1]
    \Require Domain $I$ (interval), sample access to $\D$ over $I$; subroutine $\estimdist$
    \renewcommand{\algorithmicrequire}{\textbf{Input:}}
    \Require Parameters $\eps$ and ``splittable'' function $L_\class(\cdot,\cdot,\cdot)$.
    \Start{Setting Up}
      \State Define $\gamma\eqdef \frac{\eps}{80}$, $L\eqdef L_\class(\gamma,\gamma,\abs{I})$, $\kappa\eqdef\frac{\eps}{160L}$, $\delta\eqdef\frac{1}{10L}$; and $c>0$ be as in~\cref{lemma:estimate:l2:add}. 
      \State Set $m\eqdef C\cdot\max\left( \frac{1}{\kappa}, \frac{\sqrt{L\abs{I}}}{\eps^3} \right)\cdot\log \abs{I} = \tildeO{ \frac{\sqrt{L\abs{I}}}{\eps^3} + \frac{L}{\eps} }$ \RightComment{{$C$ is an absolute constant.}}
    \State\label{step:get:samples} Obtain a sequence $\textbf{s}$ of $m$ independent  samples from $\D$.  \RightComment{\small  For any $J\subseteq I$, let $m_{J}$ be the number of samples falling in $J$.}
    \End
    \Start{Decomposition}
      \While{ $m_I \geq \max\left( c\cdot\frac{\sqrt{\abs{I}}}{\eps^2}\log\frac{1}{\delta}, \kappa m \right)$ and at most $L$ splits have been performed } \label{step:recursion} 
        \State Run \textsc{Check-Small-$\lp[2]$} (from~\cref{lemma:estimate:l2:add}) with parameters $\frac{\eps}{40}$ and $\delta$, using the samples of $\textbf{s}$ belonging to $I$.
        \If{ \textsc{Check-Small-$\lp[2]$} outputs \no }
          \State Bisect $I$, and recurse on both halves (using the same samples).
        \EndIf
      \EndWhile
      \If{more than $L$ splits have been performed} \label{step:check:succinctness}
        \State \Return \reject
      \Else
        \State Let $\mathcal{I}\eqdef (I_1,\dots,I_\ell)$ be the partition of $[n]$ from the leaves of the recursion. \Comment{$\ell \leq L$.}
      \EndIf
    \End
    \Start{Approximation}
      \State\label{step:learn:flattening} Learn the flattening $\Phi(\D,\mathcal{I})$ of $\D$ to $\lp[1]$ error $\frac{\eps}{20}$ (with probability $1/10$), using $\bigO{\ell/\eps^2}$ new samples. Let $\tilde{\D}$ be the resulting hypothesis. \Comment{{\small $\tilde{\D}$ is an $\ell$-histogram.}}
    \End
    \Start{Offline Check}
      \State \Return \accept if and only if $\estimdist(\eps, \tilde{\D})$ returns \yes. \Comment{{\small No samples needed.}}
    \End
  \end{algorithmic}
  \caption{\label{algo:test:splittable}\sc TestSplittable}
\end{algorithm}

\subsubsection{Proof of~\cref{theo:main:testing:detailed}}
We now give the proof of our main result (\cref{theo:main:testing:detailed}), first analyzing the sample complexity of~\cref{algo:test:splittable} before arguing its correctness. For the latter, we will need the following simple fact from~\cite{ILR:12}, restated below:
\begin{fact}[{\cite[Fact 1]{ILR:12}}]\label{fact:ilr12}
Let $\D$ be a distribution over $[n]$, and $\delta \in (0,1]$. Given $m\geq C\cdot\frac{\log\frac{n}{\delta}}{\eta}$ independent samples from $\D$ (for some absolute constant $C>0$), with probability at least $1-\delta$ we have that, for every interval $I\subseteq[n]$:
\begin{enumerate}[(i)]
  \item\label{fact:ilr12:1} if $\D(I) \geq \frac{\eta}{4}$, then $\frac{\D(I)}{2} \leq \frac{m_I}{m} \leq \frac{3\D(I)}{2}$;
  \item\label{fact:ilr12:2} if $\frac{m_I}{m} \geq \frac{\eta}{2}$, then $\D(I) > \frac{\eta}{4}$;
  \item\label{fact:ilr12:3} if $\frac{m_I}{m} < \frac{\eta}{2}$, then $\D(I) < \eta$;
\end{enumerate}
where $m_I \eqdef \abs{ \setOfSuchThat{j\in[m]}{x_j \in I } }$ is the number of the samples falling into $I$.
\end{fact}

\subsubsection{Sample complexity.}
The sample complexity is immediate, and comes from Steps~\ref{step:get:samples} and \ref{step:learn:flattening}. The total number of samples is
\[
  m+\bigO{\frac{\ell}{\eps^2}} = \bigO{ \frac{\sqrt{\abs{I}\cdot L}}{\eps^3} \log \abs{I} + \frac{L}{\eps}\log \abs{I} + \frac{L}{\eps^2} }
   = \bigO{ \frac{\sqrt{\abs{I}\cdot L}}{\eps^3}\log \abs{I} + \frac{L}{\eps^2} }\;.
\]
\subsubsection{Correctness.}
Say an interval $I$ considered during the execution of the ``Decomposition'' step is \emph{heavy} if $m_I$ is big enough on Step~\ref{step:recursion}, and \emph{light} otherwise; and let $\mathscr{H}$ and $\mathscr{L}$ denote the sets of heavy and light intervals respectively. By choice of $m$, we can assume that with probability at least $9/10$ the guarantees of~\cref{fact:ilr12} hold simultaneously for all intervals considered. We hereafter condition on this event.\medskip 

We first argue that if the algorithm does not reject in Step~\ref{step:check:succinctness}, then with probability at least $9/10$ we have $\normone{\D-\Phi(\D,\mathcal{I})} \leq \eps/20$ (where $\Phi(\D,\mathcal{I})$ denotes the flattening of $\D$ over the partition $\mathcal{I}$). Indeed, we can write
\begin{align*}
  \normone{\D-\Phi(\D,\mathcal{I})} &= 
            \sum_{k\colon I_k \in \mathscr{L} } \D(I_k)\cdot \normone{\D_{I_k} - \uniform_{I_k}} 
          + \sum_{k\colon I_k \in \mathscr{H}} \D(I_k)\cdot \normone{\D_{I_k} - \uniform_{I_k}} \\
          &\leq 2\sum_{k\colon I_k \in \mathscr{L} } \D(I_k) + \sum_{k\colon I_k \in \mathscr{H} } \D(I_k)\cdot \normone{\D_{I_k} - \uniform_{I_k}}\;.
\end{align*}
Let us bound the two terms separately.
    \begin{itemize}
      \item If $I^\prime \in \mathscr{H}$, then by our choice of threshold we can apply~\cref{lemma:estimate:l2:add} with $\delta=\frac{1}{10L}$; conditioning on all of the (at most $L$) events happening, which overall fails with probability at most $1/10$ by a union bound, we get
        \[
            \normtwo{\D_{I^\prime}}^2 = \normtwo{\D_{I^\prime}-\uniform_{I^\prime}}^2 + \frac{1}{\abs{I^\prime}} 
            \leq \left( 1+\frac{\eps^2}{1600} \right) \frac{1}{\abs{I^\prime}}
        \]
        as \textsc{Check-Small-$\lp[2]$} returned \yes; and by~\cref{lemma:small:l2:close:uniform:l1} this implies $\normone{\D_{I^\prime} - \uniform_{I^\prime}}\leq \eps/40$.
      \item 
      If $I^\prime \in \mathscr{L}$, then we claim that 
      $\D(I^\prime) \leq \max( \kappa, 2c\cdot\frac{\sqrt{\abs{I^\prime}}}{m\eps^2}\log\frac{1}{\delta} )$. Clearly, this is true if $\D(I^\prime) \leq \kappa$, so it only remains to show that $\D(I^\prime) \leq 2c\cdot\frac{\sqrt{\abs{I^\prime}}}{m\eps^2}\log\frac{1}{\delta}$. But this follows from~\cref{fact:ilr12} \ref{fact:ilr12:1}, as if we had 
$\D(I^\prime) > 2c\cdot\frac{\sqrt{\abs{I^\prime}}}{m\eps^2}\log\frac{1}{\delta}$ then $m_{I^\prime}$ would have been big enough, and $I^\prime\notin \mathscr{L}$. Overall,
\[
\sum_{I^\prime \in \mathscr{L} } \D(I^\prime) 
\leq \sum_{I^\prime \in \mathscr{L} } \left( \kappa + 2c\cdot\frac{\sqrt{\abs{I^\prime}}}{m\eps^2}\log\frac{1}{\delta} \right)
\leq L\kappa + 2\sum_{I^\prime \in \mathscr{L} } c\cdot\frac{\sqrt{\abs{I^\prime}}}{m\eps^2}\log\frac{1}{\delta}
\leq \frac{\eps}{160}\left(1+ \sum_{I^\prime \in \mathscr{L} } \sqrt{\frac{\abs{I^\prime}}{\abs{I}L}}\right)
\leq \frac{\eps}{80}
\]
for a sufficiently big choice of constant $C>0$ in the definition of $m$; where we first used that $\abs{\mathscr{L}} \leq L$, and then that $\sum_{I^\prime \in \mathscr{L} } \sqrt{\frac{\abs{I^\prime}}{\abs{I}}}\leq \sqrt{L}$ by Jensen's inequality.

    \end{itemize}
Putting it together, this yields
\begin{align*}
  \normone{\D-\Phi(\D,\mathcal{I})} 
  &\leq 2\cdot \frac{\eps}{80} + \frac{\eps}{40} \sum_{ I^\prime \in \mathscr{H} } \D(I_k) \leq \eps/40+\eps/40 = \eps/20.
\end{align*}

\begin{description}
  \item[Soundness.] By contrapositive, we argue that if the test returns \accept, then (with probability at least $2/3$) $\D$ is $\eps$-close to \class. Indeed, conditioning on $\tilde{\D}$ being $\eps/20$-close to $\Phi(\D,\mathcal{I})$, we get by the triangle inequality that 
  \begin{align*}
      \normone{\D-\class} &\leq \normone{\D-\Phi(\D,\mathcal{I})} + \normone{\Phi(\D,\mathcal{I}) - \tilde{\D}} + \dist{\tilde{\D}}{\class} \\
      &\leq \frac{\eps}{20} + \frac{\eps}{20}+\frac{9\eps}{10} = \eps.
  \end{align*}
  Overall, this happens except with probability at most $1/10+1/10+1/10 < 1/3$.
  
  \item[Completeness.] Assume $\D\in\class$. Then the choice of of $\gamma$ and $L$ ensures the existence of a good dyadic partition $\mathcal{I}(\gamma,\gamma,\D)$ in the sense of~\cref{def:struct:dec:split}. For any $I$ in this partition for which~\ref{def:struct:item:light} holds ($\D(I) \leq \frac{\gamma}{L} < \frac{\kappa}{2}$), $I$ will have $\frac{m_I}{m} < \kappa$ and be kept as a ``light leaf'' (this by contrapositive of~\cref{fact:ilr12} \ref{fact:ilr12:2}). For the other ones, \ref{def:struct:item:flat} holds: let $I$ be one of these (at most $L$) intervals.
    \begin{itemize}
      \item If $m_I$ is too small on  Step~\ref{step:recursion}, then $I$ is kept as ``light leaf.''
      \item Otherwise, then by our choice of constants we can use~\cref{lemma:small:l2:close:uniform:l1} and apply~\cref{lemma:estimate:l2:add} with $\delta=\frac{1}{10L}$; conditioning on all of the (at most $L$) events happening, which overall fails with probability at most $1/10$ by a union bound, \textsc{Check-Small-$\lp[2]$} will output \yes, as
        \[
            \normtwo{\D_I-\uniform_I}^2 = \normtwo{\D_I}^2 - \frac{1}{\abs{I}} \leq \left( 1+\frac{\eps^2}{6400} \right) \frac{1}{\abs{I}} - \frac{1}{\abs{I}}
             = \frac{\eps^2}{6400\abs{I}}
        \]
        and $I$ is kept as ``flat leaf.''
    \end{itemize}
  Therefore, as $\mathcal{I}(\gamma,\gamma,\D)$ is dyadic the \textsc{Decomposition} stage is guaranteed to stop within at most $L$ splits (in the worst case, it goes on until $\mathcal{I}(\gamma,\gamma,\D)$ is considered, at which point it succeeds).\footnotemark{} Thus Step~\ref{step:check:succinctness} passes, and the algorithm reaches the \textsc{Approximation} stage. By the foregoing discussion, this implies $\Phi(\D,\mathcal{I})$ is $\eps/20$-close to $\D$ (and hence to \class); $\tilde{\D}$ is then (except with probability at most $1/10$) $(\frac{\eps}{20}+\frac{\eps}{20}=\frac{\eps}{10})$-close to \class, and the algorithm returns \accept.
\end{description}
\footnotetext{In more detail, we want to argue that if $\D$ is in the class, then a decomposition with
at most $L$ pieces is found by the algorithm. Since there \emph{is} a dyadic
decomposition with at most $L$ pieces (namely, $\mathcal{I}(\gamma,\gamma,\D)=(I_1,\dots,I_t)$), it suffices to
argue that the algorithm will never split one of the $I_j$'s (as every
single $I_j$ will eventually be considered by the recursive binary splitting,
unless the algorithm stopped recursing in this ``path'' before  even considering $I_j$,
which is even better). But this is the case by the above argument, which ensures each such $I_j$ will
be recognized as satisfying one of the two conditions for ``good
decomposition'' (being either close to uniform in $\lp[2]$ distance, or having very little mass).}
 
\subsection{Structural Theorems}\label{sec:structural}
In this section, we show that a wide range of  natural distribution families 
are succinctly decomposable, and provide efficient projection algorithms
for each class. 

\subsubsection{Existence of Structural Decompositions} \label{ssec:struct-existence}

\begin{theorem}[Monotonicity]\label{theo:structural:monotone}
For all $\gamma, \zeta > 0$, the class $\classmon$ of monotone distributions on $[n]$ is $(\gamma,\zeta,L)$-splittable for $L \eqdef \bigO{\frac{\log^2 \frac{n}{\zeta}}{\gamma}}$.
\end{theorem}
\noindent Note that this proof can already be found in~\cite[Theorem 10]{BKR:04}, interwoven with the analysis of their algorithm. For the sake of being self-contained, we reproduce the structural part of their argument, removing its algorithmic aspects:
\begin{proofof}{\cref{theo:structural:monotone}}
We define the $\mathcal{I}$ recursively as follows: $\mathcal{I}^{(0)}=([1,n])$, and for $j \geq 0$ the partition $\mathcal{I}^{(j+1)}$ is obtained from $\mathcal{I}^{(j)}=(I_1^{(j)},\dots,I_{\ell_j}^{(j)})$ by going over the $I^{(j)}_i=[a^{(j)}_i, b^{(j)}_i]$ in order, and:
\begin{enumerate}[(a)]
  \item\label{theo:struct:proof:item:light} if $\D(I^{(j)}_i)\leq \frac{\zeta}{L}$, then $I^{(j)}_i$ is added as element of $\mathcal{I}^{(j+1)}$ (``marked as leaf'');
  \item\label{theo:struct:proof:item:flat} else, if $\D(a^{(j)}_i) \leq (1+\gamma)\D(b^{(j)}_i)$, then $I^{(j)}_i$ is added as element of $\mathcal{I}^{(j+1)}$ (``marked as leaf'');
  \item otherwise, bisect $I^{(j)}$ in $I^{(j)}_{\rm L}$, $I^{(j)}_{\rm R}$ (with $\abs{I^{(j)}_{\rm L}}=\clg{\abs{I^{(j)}}/2}$) and add both $I^{(j)}_{\rm L}$ and $I^{(j)}_{\rm R}$ as elements of $\mathcal{I}^{(j+1)}$.
\end{enumerate}
and repeat until convergence (that is, whenever the last item is not applied for any of the intervals). Clearly, this process is well-defined, and will eventually terminate (as $(\ell_j)_j$ is a non-decreasing sequence of natural numbers, upper bounded by $n$). Let $\mathcal{I}=(I_1,\dots,I_{\ell})$ (with $I_i=[a_i,a_{i+1})$) be its outcome, so that the $I_i$'s are consecutive intervals all satisfying either \ref{theo:struct:proof:item:light} or \ref{theo:struct:proof:item:flat}. As \ref{theo:struct:proof:item:flat} clearly implies \ref{def:struct:item:flat}, we only need to show that $\ell \leq L$; for this purpose, we shall leverage as in~\cite{BKR:04} the fact that $\D$ is monotone to bound the number of recursion steps.

The recursion above defines a complete binary tree (with the leaves being the intervals satisfying \ref{theo:struct:proof:item:light} or \ref{theo:struct:proof:item:flat}, and the internal nodes the other ones). Let $t$ be the number of recursion steps the process goes through before converging to $\mathcal{I}$ (height of the tree); as mentioned above, we have $t\leq \log n$ (as we start with an interval of size $n$, and the length is halved at each step.). Observe further that if at any point an interval $I^{(j)}_i=[a^{(j)}_i, b^{(j)}_i]$ has $\D(a^{(j)}_i) \leq \frac{\zeta}{nL}$, then it immediately (as well as all the $I^{(j)}_k$'s for $k\geq i$ by monotonicity) satisfies \ref{theo:struct:proof:item:light} and is no longer split (``becomes a leaf''). So at any $j \leq t$, the number of intervals $i_j$ for which neither \ref{theo:struct:proof:item:light} nor \ref{theo:struct:proof:item:flat} holds must satisfy
\[
  1 \geq \D(a^{(j)}_1) > (1+\gamma)\D(a^{(j)}_2) > (1+\gamma)^2\D(a^{(j)}_3) > \dots > (1+\gamma)^{i_j-1}\D(a^{(j)}_{i_j}) \geq (1+\gamma)^{i_j-1}\frac{\zeta}{nL}
\]
where $a_k$ denotes the beginning of the $k$-th interval (again we use monotonicity to argue that the extrema were reached at the ends of each interval), so that $i_j \leq 1+\frac{\log\frac{nL}{\zeta}}{\log(1+\gamma)}$. In particular, the total number of internal nodes is then 
\[
\sum_{i=1}^t i_j \leq t\cdot\left(1+\frac{\log\frac{nL}{\zeta}}{\log(1+\gamma)}\right) \leq \frac{2\log^2 \frac{n}{\zeta}}{\log(1+\gamma)} \leq L\;.\]
This implies the same bound on the number of leaves $\ell$.
\end{proofof}

\begin{corollary}[Unimodality]\label{theo:structural:unimodal}
For all $\gamma,\zeta > 0$, the class $\classuni$ of unimodal distributions on $[n]$ is $(\gamma,\zeta,L)$-decomposable for $L \eqdef \bigO{\frac{\log^2 \frac{n}{\zeta}}{\gamma}}$.
\end{corollary}
\begin{proof}
For any $\D\in\classuni$, $[n]$ can be partitioned in two intervals $I$, $J$ such that $\D_{I}$, $\D_J$ are either monotone non-increasing or non-decreasing. Applying \cref{theo:structural:monotone} to $\D_{I}$ and $\D_J$ and taking the union of both partitions yields a (no longer necessarily dyadic) partition of $[n]$.
\end{proof}
\noindent The same argument yields an analogous statement for $t$-modal distributions:
\begin{corollary}[$t$-modality]\label{theo:structural:tmodal}
For any $t\geq 1$ and all $\gamma,\zeta > 0$, the class $\classtmo$ of $t$-modal distributions on $[n]$ is $(\gamma,\zeta,L)$-decomposable for $L \eqdef \bigO{\frac{t\log^2 \frac{n}{\zeta}}{\gamma}}$.
\end{corollary}

\begin{corollary}[Log-concavity, concavity and convexity]\label{theo:structural:logconcave}
For all $\gamma,\zeta > 0$, the classes $\classlogconcave$, $\classcve$ and $\classcvx$ of log-concave, concave and convex distributions on $[n]$ are $(\gamma,\zeta,L)$-decomposable for $L \eqdef \bigO{\frac{\log^2 \frac{n}{\zeta}}{\gamma}}$.
\end{corollary}
\begin{proof}
This is directly implied by \cref{theo:structural:unimodal}, recalling that log-concave, concave and convex distributions are unimodal.
\end{proof}

\begin{restatable}[Monotone Hazard Rate]{theorem}{theostructuralmhr}\label{theo:structural:mhr}
For all $\gamma,\zeta > 0$, the class $\classmhr$ of MHR distributions on $[n]$ is $(\gamma,\zeta,L)$-decomposable for $L \eqdef \bigO{\frac{\log \frac{n}{\zeta}}{\gamma}}$.
\end{restatable}
\begin{proof}
This follows from adapting the proof of~\cite{CDSS:13}, which establishes that every MHR distribution can be approximated in $\lp[1]$ distance by a $\bigO{\log(n/\eps)/\eps}$-histogram. For completeness, we reproduce their argument, suitably modified to our purposes, in \cref{app:structural:proofs}.
\end{proof}

\begin{theorem}[Piecewise Polynomials]\label{theo:structural:piecewise}
For all $\gamma,\zeta > 0$, $t,d\geq 0$, the class $\classpoly[t,d]$ of $t$-piecewise degree-$d$ distributions on $[n]$ is $(\gamma,\zeta,L)$-decomposable for $L \eqdef \bigO{\frac{t(d+1)}{\gamma}\log^2 \frac{n}{\zeta}}$. (Moreover, for the class of $t$-histograms $\classhist$ ($d=0$) one can take $L = t$.)
\end{theorem}
\begin{proof}
The last part of the statement is obvious, so we focus on the first claim. Observing that each of the $t$ pieces of a distribution $\D\in\classpoly[t,d]$ can be subdivided in at most $d+1$ intervals on which $\D$ is monotone (being degree-$d$ polynomial on each such piece), we obtain a partition of $[n]$ into at most $t(d+1)$ intervals. $\D$ being monotone on each of them, we can apply an argument almost identical to that of~\cref{theo:structural:monotone} to argue that each interval can be further split into $O(\log^2 n/\gamma)$ subintervals, yielding a good decomposition with $O( t(d+1)\log^2 ({n}/{\zeta})/\gamma )$ pieces.
\end{proof}



\subsubsection{Projection Step: computing the distances} 
This section contains details of the distance estimation procedures for these classes, required in the last stage of \cref{algo:test:splittable}. (Note that some of these results are phrased in terms of distance approximation, as estimating the distance $\lp[1](\D,\class)$ to sufficient accuracy in particular yields an algorithm for this stage.)

We focus in this section on achieving the sample complexities stated in~\cref{coro:main:testing}, \cref{coro:main:testing:tmod}, and~\cref{coro:main:testing:piecewise} -- that is, our procedures will not require any additional sample from the distribution. While almost all the distance estimation procedures we give in this section are efficient, running in time polynomial in all the parameters or even with only a polylogarithmic dependence on $n$, there are two exceptions -- namely, the procedures for monotone hazard rate (\cref{lemma:distance:mhr}) and log-concave (\cref{lemma:distance:log}) distributions. We \emph{do} describe computationally efficient procedures for these two cases as well in~\cref{ssec:efficient:logconcave:mhr}, at a modest additive cost in the sample complexity (that is, these more efficient procedures \emph{will} require some additional samples from the distribution).

\begin{lemma}[Monotonicity {\cite[Lemma 8]{BKR:04}}]\label{lemma:distance:mon}
There exists a procedure $\estimdist[\classmon]$ that, on input $n$ as well as the full (succinct) specification of an $\ell$-histogram $\D$ on $[n]$, computes the (exact) distance $\lp[1](\D,\classmon)$ in time $\poly(\ell)$.
\end{lemma}

A straightforward modification of the algorithm above (e.g., by adapting the underlying linear program to take as input the location $m\in[\ell]$ of the mode of the distribution; then trying all $\ell$ possibilities, running the subroutine $\ell$ times and picking the minimum value) results in a similar claim for unimodal distributions:
\begin{lemma}[Unimodality]\label{lemma:distance:uni}
There exists a procedure $\estimdist[\classuni]$ that, on input $n$ as well as the full (succinct) specification of an $\ell$-histogram $\D$ on $[n]$, computes the (exact) distance $\lp[1](\D,\classuni)$ in time $\poly(\ell)$.
\end{lemma}
A similar result can easily be obtained for the class of $t$-modal distributions as well, with a $\poly(\ell,t)$-time algorithm based on a combination of dynamic and linear programming. 
 Analogous statements hold for the classes of concave and convex distributions $\classcvx, \classcve$, also based on linear programming (specifically, on running $\bigO{n^2}$ different linear programs -- one for each possible support $[a,b]\subseteq[n]$ -- and taking the minimum over them).

\begin{lemma}[MHR]\label{lemma:distance:mhr}
There exists a (non-efficient) procedure $\estimdist[\classmhr]$ that, on input $n$, $\eps$, as well as the full specification of a distribution $\D$ on $[n]$, distinguishes between $\lp[1](\D,\classmhr) \leq \eps$ and $\lp[1](\D,\classmhr)>2\eps$ in time~$2^{\tilde{O}_\eps(n)}$.
\end{lemma}
\begin{lemma}[Log-concavity]\label{lemma:distance:log}
There exists a (non-efficient) procedure $\estimdist[\classlogconcave]$ that, on input $n$, $\eps$, as well as the full specification of a distribution $\D$ on $[n]$, distinguishes between $\lp[1](\D,\classlogconcave) \leq \eps$ and $\lp[1](\D,\classlogconcave)>2\eps$ in time~$2^{\tilde{O}_\eps(n)}$.
\end{lemma}
\begin{proofof}{\cref{lemma:distance:mhr} and \cref{lemma:distance:log}}
We here give a naive algorithm for these two problems, based on an exhaustive search over a (huge) $\eps$-cover $\mathcal{S}$ of distributions over $[n]$. Essentially, $\mathcal{S}$ contains all possible distributions whose probabilities $p_1,\dots,p_n$ are of the form $j\eps/n$, for $j\in\{0,\dots,n/\eps\}$ (so that $\abs{\mathcal{S}} = \bigO{(n/\eps)^{n}}$). It is not hard to see that this indeed defines an \eps-cover of the set of all distributions, and moreover that it can be computed in time $\poly(\abs{\mathcal{S}})$. To approximate the distance from an explicit distribution $\D$ to the class $\class$ (either $\classmhr$ or $\classlogconcave$), it is enough to go over every element $S$ of $\mathcal{S}$, checking (this time, efficiently) if $\normone{S-\D}\leq \eps$ and if there is a distribution $P\in\class$ close to $S$ (this time, pointwise, that is $\abs{P(i)-S(i)} \leq \eps/n$ for all $i$) -- which also implies $\normone{S-P}\leq \eps$ and thus $\normone{P-\D}\leq 2\eps$. The test for pointwise closeness can be done by checking feasibility of a linear program with variables corresponding to the logarithm of probabilities, i.e. $x_i \equiv \ln P(i)$. Indeed, this formulation allows to rephrase the log-concave and MHR constraints as linear constraints, and pointwise approximation is simply enforcing that $\ln(S(i)-\eps/n) \leq x_i \leq \ln(S(i)+\eps/n)$ for all $i$. At the end of this enumeration, the procedure accepts if and only if for some $S$ both $\normone{S-\D}\leq \eps$ and the corresponding linear program was feasible.
\end{proofof}

\begin{lemma}[Piecewise Polynomials]\label{lemma:distance:piecewise}
There exists a procedure $\estimdist[\classpoly]$ that, on input $n$ as well as the full specification of an $\ell$-histogram $\D$ on $[n]$, computes an approximation $\Delta$ of the distance $\lp[1](\D,\classpoly)$ such that $\lp[1](\D,\classpoly) \leq \Delta \leq 3\lp[1](\D,\classpoly)+\eps$, and runs in time $\bigO{n^3}\cdot\poly(\ell,t,d,\frac{1}{\eps})$.

Moreover, for the special case of $t$-histograms ($d=0$) there exists a procedure $\estimdist[\classhist]$, which, given inputs as above, computes an approximation $\Delta$ of the distance $\lp[1](\D,\classhist)$ such that  $\lp[1](\D,\classhist) \leq \Delta \leq 4\lp[1](\D,\classhist)+\eps$, and runs in time $\poly(\ell,t,\frac{1}{\eps})$, independent of $n$.
\end{lemma}
\begin{proof}
We begin with $\estimdist[\classhist]$. Fix any distribution $\D$ on $[n]$. Given any explicit partition of $[n]$ into intervals $\mathcal{I}=(I_1,\dots,I_t)$, one can easily show that $\normone{\D - \Phi(\D,\mathcal{I})} \leq 2\opt_{\mathcal{I}}$, where $\opt_{\mathcal{I}}$ is the optimal distance of $\D$ to any histogram on $\mathcal{I}$ (recall that we write $\Phi(\D,\mathcal{I})$ for the flattening of $\D$ over the partition $\mathcal{I}$). To get a $2$-approximation of $\lp[1](\D,\classhist)$, it thus suffices to find the minimum, over all possible partitionings $\mathcal{I}$ of $[n]$ into $t$ intervals, of the quantity $\normone{\D - \Phi(\D,\mathcal{I})}$ (which itself can be computed in time $T=O(\min(t\ell,n))$). By a simple dynamic programming approach, this can be performed in time $\bigO{t n^2 \cdot T}$. The quadratic dependence on $n$, which follows from allowing the endpoints of the $t$ intervals to be at any point of the domain, is however far from optimal and can be reduced to $(t/\eps)^2$, as we show below.

For $\eta > 0$, define an \emph{$\eta$-granular decomposition} of a distribution $\D$ over $[n]$ to be a partition of $[n]$ into $s=\bigO{1/\eta}$ intervals $J_1,\dots,J_s$ such that each interval $J_i$ is either a singleton or satisfies $\D(J_i) \leq \eta$. (Note that if $\D$ is a known $\ell$-histogram, one can compute an $\eta$-granular decomposition of $\D$ in time $\bigO{\ell/\eta}$ in a greedy fashion.)

\begin{claim}\label{claim:granularity:piecewise:projection}
Let $\D$ be a distribution over $[n]$, and $\mathcal{J} = (J_1,\dots,J_s)$ be an $\eta$-granular decomposition of $\D$ (with $s\geq t$). Then, there exists a partition of $[n]$ into $t$ intervals $\mathcal{I}=(I_1,\dots,I_t)$ and a $t$-histogram $H$ on $\mathcal{I}$ such that $\normone{\D - H} \leq 2\lp[1](\D,\classhist[t])+2t\eta$, and $\mathcal{I}$ is a coarsening of $\mathcal{J}$.
\end{claim}
Before proving it, we describe how this will enable us to get the desired time complexity for $\estimdist[\classhist]$. Phrased differently, the claim above allows us to run our dynamic program using the $\bigO{1/\eta}$ endpoints of the $\bigO{1/\eta}$ instead of the $n$ points of the domain, paying only an additive error $O(t\eta)$. Setting $\eta=\frac{\eps}{4t}$, the guarantee for $\estimdist[\classhist]$ follows.

\begin{proofof}{\cref{claim:granularity:piecewise:projection}} Let $\mathcal{J} =(J_1,\dots,J_s)$ be an $\eta$-granular decomposition of $\D$, and $H^\ast\in\classhist[t]$ be a histogram achieving $\opt=\lp[1](\D,\classhist[t])$. Denote further by $\mathcal{I^\ast} = (I^\ast_1,\dots,I^\ast_t)$ the partition of $[n]$ corresponding to $H^\ast$. Consider now the $r\leq t$ endpoints of the $I^\ast_i$'s that do not fall on one of the endpoints of the $J_i$'s: let $J_{i_1},\dots,J_{i_r}$ be the respective intervals in which they fall (in particular, these cannot be singleton intervals), and $S=\cup_{j=1}^r J_{i_j}$ their union. By definition of $\eta$-granularity, $\D(S) \leq t\eta$, and it follows that $H^\ast(S)\leq t\eta + \frac{1}{2}\opt$. We define $H$ from $H^\ast$ in two stages: first, we obtain a (sub)distribution $H^\prime$ by modifying $H^\ast$ on $S$, setting for each $x\in J_{i_j}$ the value of $H$ to be the minimum value (among the two options) that $H^\ast$ takes on $J_{i_j}$. $H^\prime$ is thus a $t$-histogram, and the endpoints of its intervals are endpoints of $\mathcal{J}$ as wished; but it may not sum to one. However, by construction we have that $H^\prime([n]) \geq 1-H^\ast(S) \geq 1-t\eta - \frac{1}{2}\opt$. Using this, we can finally define our $t$-histogram distribution $H$ as the renormalization of $H^\prime$. It is easy to check that $H$ is a valid $t$-histogram on a coarsening of $\mathcal{J}$, and
\[
    \normone{\D-H} \leq \normone{\D-H^\prime} + (1-H^\prime([n])) \leq \normone{\D-H^\ast} + \normone{H^\ast-H^\prime} + t\eta + \frac{1}{2}\opt \leq 2\opt  + 2t\eta
\]
as stated.
\end{proofof}

Turning now to $\estimdist[\classpoly]$, we apply the same initial dynamic programming approach, which will result on a running time of $\bigO{n^2t\cdot T}$, where $T$ is the time required to estimate (to sufficient accuracy) the distance of a given (sub)distribution over an interval $I$ onto the space $\classpoly[d]$ of degree-$d$ polynomials. Specifically, we will invoke the following result, adapted from~\cite{CDSS:14} to our setting:
\begin{theorem} \label{theo:degreed:poly-projection}
Let $p$ be an $\ell$-histogram over $[-1,1)$. There is an algorithm $\textsc{ProjectSinglePoly}(d,\eta)$
which runs in time $\poly(\ell, d+1,1/\eta)$, and outputs a degree-$d$ polynomial $q$ which defines a pdf over $[-1,1) $
such that $\normone{p-q} \leq 3 \lp[1](p,\classpoly[d]) + O(\eta)$.
\end{theorem}
The proof of this modification of~\cite[Theorem 9]{CDSS:14} is deferred to~\cref{app:structural:projection:proofs}. Applying it as a blackbox with $\eta$ set to $\bigO{\eps/t}$ and noting that computing the $\lp[1]$ distance to our explicit distribution on a given interval of the degree-$d$ polynomial returned incurs an additional $\bigO{n}$ factor, we obtain the claimed guarantee and running time.
\end{proof}

\paragraph{Computationally Efficient Procedures for Log-concave and MHR Distributions}\label{ssec:efficient:logconcave:mhr}

We now describe how to obtain \emph{efficient} testing for the classes \classlogconcave and \classmhr{} -- that is, how to obtain polynomial-time distance estimation procedures for these two classes, unlike the ones described in the previous section. At a very high-level, the idea is in both cases to write down a linear program on variables related \emph{logarithmically} to the probabilities we are searching, as enforcing the log-concave and MHR constraints on these new variables can be done linearly. The catch now becomes the $\lp[1]$ objective function (and, to a lesser extent, the fact that the probabilities must sum to one), now highly non-linear.

The first insight is to leverage the structure of log-concave (resp. monotone hazard rate) distributions to express this objective as slightly stronger constraints, specifically pointwise $(1\pm\eps)$-multiplicative closeness, much easier to enforce in our ``logarithmic formulation.'' Even so, doing this naively fails, essentially because of a too weak distance guarantee between our explicit histogram $\hat{\D}$ and the unknown distribution we are trying to find: in the completeness case, we are only promised $\eps$-closeness in $\lp[1]$, while we would also require good additive pointwise closeness of the order $\eps^2$ or~$\eps^3$.

The second insight is thus to observe that we ``almost'' have this for free: indeed, if we do not reject in the first stage of the testing algorithm, we do obtain an explicit $k$-histogram $\hat{\D}$ with the guarantee that $\D$ is $\eps$-close to the distribution $P$ to test. However, we \emph{also} implicitly have another distribution $\hat{\D}^\prime$ that is $\sqrt{\eps/k}$-close to $P$ \emph{in Kolmogorov distance}: as in the recursive descent we take enough samples to use the DKW inequality (\cref{theo:dkw:ineq}) with this parameter, i.e. an additive overhead of $\bigO{k/\eps}$ samples (on top of the $\tilde{O}(\sqrt{kn}/\eps^{7/2})$). If we are willing to increase this overhead by just a small amount, that is to take $\tildeO{ \max(k/\eps, 1/\eps^4) }$, we can guarantee that $\hat{\D}^\prime$ be also $\tildeO{\eps^2}$-close to $P$ in Kolmogorov distance.\smallskip

\noindent Combining these ideas yield the following distance estimation lemmas:
\begin{restatable}[Monotone Hazard Rate]{lemma}{lemmaefficientdistancemhr}\label{lemma:distance:mhr:eff}
  There exists a procedure $\estimdist[\classmhr]^\ast$ that, on input $n$ as well as the full specification of a $k$-histogram distribution $\D$ on $[n]$ and of an $\ell$-histogram distribution $\D^{\prime}$ on $[n]$, runs in time $\poly(n,1/\eps)$, and satisfies the following.
  \begin{itemize}
    \item If there is $P\in\classmhr$ such that $\normone{\D-P} \leq \eps$  and $\kolmogorov{\D^\prime}{P} \leq \eps^3$, then the procedure returns~\yes;
    \item If $\lp[1](\D,\classmhr) > 100\eps$, then the procedure returns \no.
  \end{itemize}
\end{restatable}

\begin{restatable}[Log-concavity]{lemma}{lemmaefficientdistancelog}\label{lemma:distance:log:eff}
  There exists a procedure $\estimdist[\classlogconcave]^\ast$ that, on input $n$ as well as the full specifications of a $k$-histogram distribution $\D$ on $[n]$ and an $\ell$-histogram distribution $\D^\prime$ on $[n]$, runs in time $\poly(n,k,\ell,1/\eps)$, and satisfies the following.
  \begin{itemize}
    \item If there is $P\in\classlogconcave$ such that $\normone{\D-P}\leq \eps$ \emph{and} $\kolmogorov{\D^\prime}{P}\leq \frac{\eps^2}{\log^2(1/\eps)}$, then the procedure returns~\yes;
    \item If $\lp[1](\D,\classlogconcave) \geq 100\eps$, then the procedure returns \no.
  \end{itemize}
\end{restatable}

The proofs of these two lemmas are quite technical and deferred to~\cref{app:structural:projection:proofs}. With these in hand, a simple modification of our main algorithm (specifically, setting $m = \tilde{O}( \max({\sqrt{L\abs{I}}}/{\eps^3}, {L}/{\eps^2}, {1}/{\eps^c} ) )$ for $c$ either $4$ or $6$ instead of $\tilde{O}( \max ( {\sqrt{L\abs{I}}}/{\eps^3}, {L}/{\eps^2} ) )$, to get the desired Kolmogorov distance guarantee; and providing the empirical histogram defined by these $m$ samples along to the distance estimation procedure) suffices to obtain the following counterpart to~\cref{coro:main:testing}:
\begin{corollary}\label{coro:main:testing:eff}
The algorithm \textsc{TestSplittable}, after this modification, can \emph{efficiently} test the classes of log-concave and monotone hazard rate (MHR) distributions, with respectively $\tilde{O}\big({\sqrt{n}/\eps^{7/2} + 1/\eps^4}\big)$ and $\tilde{O}\big({\sqrt{n}/\eps^{7/2} + 1/\eps^6}\big)$ samples.
\end{corollary}

We observe that~\cref{lemma:distance:mhr:eff} and~\cref{lemma:distance:log:eff} actually imply efficient \emph{proper} learning algorithms for the classes of respectively MHR and log-concave distributions, with sample complexity $\bigO{1/\eps^4}$ and $\bigO{1/\eps^6}$. Along with analogous subroutines of~\cite{ADK:15}, these were the first proper learning algorithms (albeit with suboptimal sample complexity) for these classes. (Subsequent work of Diakonikolas, Kane, and Steward~\cite{DKS:16} recently obtained, through a completely different approach, a sample-optimal and efficient learning algorithm for the class of log-concave distributions which is both \emph{proper} and \emph{agnostic}.)
 
\subsection{Going Further: Reducing the Support Size}\label{sec:effectivesupport}
  The general approach we have been following so far gives, out-of-the-box, an efficient testing algorithm with sample complexity $\tildeO{\sqrt{n}}$ for a large range of properties. However, this sample complexity can for some classes \property be brought down a lot more, by taking advantage in a preprocessing step of good concentration guarantees of distributions in \property.\medskip

\noindent As a motivating example, consider the class of Poisson Binomial Distributions (PBD). It is well-known (see e.g.~\cite[Section 2]{KG:71}) that PBDs are unimodal, and more specifically that $\classpbd[n]\subseteq\classlogconcave\subseteq\classuni$. Therefore, using our generic framework we can test Poisson Binomial Distributions with $\tildeO{\sqrt{n}}$ samples. This is, however, far from optimal: as shown in~\cite{AD:15}, a sample complexity of $\bigTheta{n^{1/4}}$ is both necessary and sufficient. The reason our general algorithm ends up making quadratically too many queries can be explained as follows. PBDs are tightly concentrated around their expectation, so that they ``morally'' live on a support of size $m=\bigO{\sqrt{n}}$. Yet, instead of testing them on this very small support, in the above we still consider the entire range $[n]$, and thus end up paying a dependence $\sqrt{n}$ -- instead of $\sqrt{m}$.

If we could use that observation to first reduce the domain to the \emph{effective support} of the distribution, then we could call our testing algorithm on this reduced domain of size $\bigO{\sqrt{n}}$. In the rest of this section, we formalize and develop this idea, and in~\cref{ssec:testing:pbds} will obtain as a direct application a $\tildeO{n^{1/4}}$-query testing algorithm for $\classpbd[n]$.

\begin{definition}\label{def:effective:support}
Given $\eps > 0$, the \emph{\eps-effective support} of a distribution $\D$ is the smallest interval $I$ such that $\D(I) \geq 1-\eps$.
\end{definition}

The last definition we shall require is that of the \emph{conditioned distributions} of a class \class:
\begin{definition}
For any class of distributions \class over $[n]$, define the set of \emph{conditioned distributions of \class} (with respect to $\eps > 0$ and interval $I\subseteq[n]$) as $\class^{\eps,I}\eqdef \setOfSuchThat{\D_I}{\D\in\class, \D(I) \geq 1-\eps}$.
\end{definition}

Finally, we will require the following simple result:
\begin{lemma}\label{lemma:conditioned:class}
Let $\D$ be a distribution over $[n]$, and $I\subseteq[n]$ an interval such that $\D(I) \geq 1- \frac{\eps}{10}$. Then,
\begin{itemize}
  \item If $\D\in\class$, then $\D_I\in\class^{\frac{\eps}{10},I}$;
  \item If $\lp[1](\D,\class) > \eps$, then $\lp[1](\D_I, \class^{\frac{\eps}{10},I}) > \frac{7\eps}{10}$.
\end{itemize}
\end{lemma}
\begin{proof}
The first item is obvious. As for the second, let $P\in\class$ be any distribution with $P(I)\geq 1-\frac{\eps}{10}$. By assumption, $\normone{\D - P} > \eps$: but we have, writing $\alpha=1/10$,
  \begin{align*}
    \normone{\D_I - P_I} &=\sum_{i\in I}\abs{ \frac{\D(i)}{\D(I)} - \frac{P(i)}{P(I)} }
    = \frac{1}{\D(I)}\sum_{i\in I}\abs{ \D(i) - P(i) + P(i)\mleft(1- \frac{\D(I)}{P(I)}\mright) } \\
    &\geq \frac{1}{\D(I)}\big( \sum_{i\in I}\abs{ \D(i) - P(i) } - \abs{1- \frac{\D(I)}{P(I)} } \sum_{i\in I} P(i) \big)\\
    &= \frac{1}{\D(I)}\big( \sum_{i\in I}\abs{ \D(i) - P(i) } - \abs{P(I)- \D(I) } \big)
    \geq \frac{1}{\D(I)}\big( \sum_{i\in I}\abs{ \D(i) - P(i) } - \alpha\eps \big)\\
   &\geq \frac{1}{\D(I)}\big( \normone{\D-P} - \sum_{i\notin I} \abs{ \D(i) - P(i) } - \alpha\eps \big)
   \geq \frac{1}{\D(I)}\big( \normone{\D-P} - 3\alpha\eps \big) \\
   &> (1- 3\alpha)\eps  
   = \frac{7}{10}\eps.
  \end{align*}
\end{proof}


We now proceed to state and prove our result -- namely, efficient testing of \emph{structured} classes of distributions with nice \emph{concentration properties}.
\begin{theorem}\label{theo:main:testing:effective:support}
  Let $\class$ be a class of distributions over $[n]$ for which the following holds.
    \begin{enumerate}
      \item there is a function $M(\cdot,\cdot)$ such that each $\D\in\class$ has \eps-effective support of size at most $M(n,\eps)$;
      \item for every $\eps \in [0,1]$ and interval $I\subseteq[n]$, $\class^{\eps,I}$ is $(\gamma,\zeta,L)$-splittable;
      \item there exists an efficient procedure $\estimdist[\class^{\eps,I}]$ which, given as input the explicit description of a distribution $\D$ over $[n]$ and interval $I\subseteq[n]$, computes the distance $\lp[1](\D_I,\class^{\eps,I})$.
    \end{enumerate}
    Then, the algorithm \textsc{TestEffectiveSplittable} (\cref{algo:test:effective:support:splittable}) is a $\bigO{ \max\mleft(\frac{1}{\eps^3} \sqrt{m\ell} \log m, \frac{\ell}{\eps^2}\mright) }$-sample tester for $\class$, where $m=M(n,\frac{\eps}{60})$ and $\ell=L(\frac{\eps}{1200},\frac{\eps}{1200}, m)$.
\end{theorem}

\begin{algorithm}[H]
  \algblock[block]{Start}{Start}
  \algblockdefx[]{Start}{End}    [1]{\textsc{#1}}    [1][]{\textsc{#1}}
  \begin{algorithmic}[1]
    \Require Domain $\domain$ (interval of size $n$), sample access to $\D$ over $\domain$; subroutine $\estimdist[\class^{\eps,I}]$
    \renewcommand{\algorithmicrequire}{\textbf{Input:}}
    \Require Parameters $\eps\in(0,1]$, function $L(\cdot,\cdot,\cdot)$, and upper bound function $M(\cdot,\cdot)$ for the effective support of the class \class.
    \State Set $m\eqdef \bigO{1/\eps^2}$, $\tau \eqdef M(n,\frac{\eps}{60})$.     \Start{Effective Support}
      \State\label{step:effesupp:get:samples} Compute $\hat{\D}$, an empirical estimate of $\D$, by drawing $m$ independent  samples from $\D$.
      \State\label{step:effesupp:int:J} Let $J$ be the largest interval of the form $\{1,\dots,j\}$ such that $\hat{\D}(J) \leq \frac{\eps}{30}$.
      \State\label{step:effesupp:int:K} Let $K$ be the largest interval of the form $\{k,\dots,n\}$ such that $\hat{\D}(K) \leq \frac{\eps}{30}$.
      \State\label{step:effesupp:test:support} Set $I\gets [n]\setminus(J\cup K)$.
      \If{ $\abs{I} > \tau$ } \Return \reject \EndIf
    \End
    \Start{Testing}
      \State\label{step:effesupp:main:test:call} Call $\textsc{TestSplittable}$ with $I$ (providing simulated access to $\D_I$ by rejection sampling, returning \fail if the number of samples $q$ from $\D_I$ required by the subroutine is not obtained after $\bigO{q}$ samples from $\D$), $\estimdist[\class^{\eps,I}]$, parameters $\eps^\prime\eqdef \frac{7\eps}{10}$ and $L(\cdot,\cdot,\cdot)$.
      \State\label{step:effesupp:main:test} \Return \accept if $\textsc{TestSplittable}$ accepts, \reject otherwise.
    \End
  \end{algorithmic}
  \caption{\label{algo:test:effective:support:splittable}\sc TestEffectiveSplittable}
\end{algorithm}


\subsubsection{Proof of~\cref{theo:main:testing:effective:support}}
By the choice of $m$ and the DKW inequality, with probability at least $23/24$ the estimate $\hat{\D}$ satisfies $\kolmogorov{\D}{\hat{\D}} \leq \frac{\eps}{60}$. Conditioning on that from now on, we get that $\D(I) \geq \hat{\D}(I) - \frac{\eps}{30} \geq 1-\frac{\eps}{10}$. Furthermore, denoting by $j$ and $k$ the two inner endpoints of $J$ and $K$ in Steps~\ref{step:effesupp:int:J} and \ref{step:effesupp:int:K}, we have $\D(J\cup\{j+1\}) \geq \hat{\D}(J\cup\{j+1\}) -  \frac{\eps}{60}  > \frac{\eps}{60}$ (similarly for $\D(K\cup\{k-1\})$), so that $I$ has size at most $\sigma+1$, where $\sigma$ is the $\frac{\eps}{60}$-effective support size of $\D$. 

Finally, note that since $\D(I) = \bigOmega{1}$ by our conditioning, the simulation of samples by rejection sampling will succeed with probability at least $23/24$ and the algorithm will not output \fail.

\subparagraph{Sample complexity}
The sample complexity is the sum of the $\bigO{1/\eps^2}$ in Step~\ref{step:effesupp:get:samples} and the $\bigO{q}$ in Step~\ref{step:effesupp:main:test:call}. From~\cref{theo:main:testing} and the choice of $I$, this latter quantity is $\bigO{ \max\mleft(\frac{1}{\eps^3} \sqrt{m\ell} \log m, \frac{\ell}{\eps^2}\mright) }$ where $m = M(n,\frac{\eps}{60})$ and $\ell=L(\frac{\eps}{1200},\frac{\eps}{1200}, M(n,\frac{\eps}{60}))$.

\subparagraph{Correctness} If $\D\in\class$, then by the setting of $\tau$ (set to be an upper bound on the $\frac{\eps}{60}$-effective support size of any distribution in \class) the algorithm will go beyond Step~\ref{step:effesupp:test:support}. The call to $\textsc{TestSplittable}$ will then end up in the algorithm returning \accept in Step~\ref{step:effesupp:main:test}, with probability at least $2/3$ by~\cref{lemma:conditioned:class},~\cref{theo:main:testing} and our choice of parameters.

Similarly, if $\D$ is \eps-far from \class, then either its effective support is too large (and then the test on Step~\ref{step:effesupp:test:support} fails), or the main tester will detect that its conditional distribution on $I$ is $\frac{7\eps}{10}$-far from $\class$ and output \reject in Step~\ref{step:effesupp:main:test}.

Overall, in either case the algorithm is correct except with probability at most $1/24+1/24+1/3=5/12$ (by a union bound). Repeating constantly many times and outputting the majority vote brings the probability of failure down to $1/3$. \qed

\subsubsection{Application: Testing Poisson Binomial Distributions}\label{ssec:testing:pbds}

In this section, we illustrate the use of our generic two-stage approach to test the class of Poisson Binomial Distributions. Specifically, we prove the following result:
\begin{corollary}\label{coro:main:testing:effective:support:pbd}
The class of Poisson Binomial Distributions can be tested with $\tildeO{{n}^{1/4}/\eps^{7/2}} + \tildeO{\log^2 n/\eps^3 }$ samples, using~\cref{algo:test:effective:support:splittable}.
\end{corollary}

This is a direct consequence of~\cref{theo:main:testing:effective:support} and the lemmas below. The first one states that, indeed, PBDs have small effective support:
\begin{fact}\label{fact:pbd:effective:support}
For any $\eps > 0$, a PBD has \eps-effective support of size $\bigO{\sqrt{n\log(1/\eps)}}$.
\end{fact}
\begin{proof}
By an additive Chernoff Bound, any random variable $X$ following a Poisson Binomial Distribution has $\probaOf{\abs{X-\shortexpect X} > \gamma n} \leq 2e^{-2\gamma^2n}$. Taking $\gamma\eqdef\sqrt{\frac{1}{2n}\ln\frac{2}{\eps}}$, we get that $\probaOf{X\in I} \geq 1-\eps$, where $I\eqdef [\shortexpect X-\sqrt{\frac{n}{2}\ln\frac{2}{\eps}}, \shortexpect X+\sqrt{\frac{n}{2}\ln\frac{2}{\eps}}]$.
\end{proof}

It is clear that if $\D\in\classpbd[n]$ (and therefore is unimodal), then for any interval $I\subseteq[n]$ the conditional distribution $\D_I$ is still unimodal, and thus the class of \emph{conditioned PBDs} $\classpbd[n]^{\eps,I}\eqdef \setOfSuchThat{\D_I}{\D\in\classpbd[n], \D(I) \geq 1-\eps}$ falls under~\cref{theo:structural:unimodal}. The last piece we need to apply our generic testing framework is the existence of an algorithm to compute the distance between an (explicit) distribution and the class of conditioned PBDs. This is provided by our next lemma:
\begin{claim}\label{lemma:distance:pbd}
There exists a procedure $\estimdist[{\classpbd[n]^{\eps,I}}]$ that, on input $n$ and $\eps \in [0,1]$, $I\subseteq[n]$ as well as the full specification of a distribution $\D$ on $[n]$, computes a value $\tau$ such that $\tau \in [1\pm 2\eps] \cdot \lp[1](\D,\classpbd[n]^{\eps,I}) \pm \frac{\eps}{100}$, in time $n^2 \left( {1/\eps} \right)^{\bigO{\log{1/\eps}}}$.
\end{claim}
\begin{proof}
The goal is to find a $\gamma = \Theta(\eps)$-approximation of the minimum value of $\sum_{i\in I}\abs{\frac{P(i)}{P(I)} - \frac{\D(i)}{\D(I)}}$, subject to $P(I)=\sum_{i\in I} P(i) \geq 1-\eps$ and $P\in\classpbd[n]$. We first note that, given the parameters $n \in \N$ and $p_1,\dots,p_n\in[0,1]$ of a PBD $P$, the vector of $(n+1)$ probabilities $P(0),\dots,P(n)$ can be obtained in time $\bigO{n^2}$ by dynamic programming. 
Therefore, computing the $\lp[1]$ distance between $\D$ and any PBD with known parameters can be done efficiently. To conclude, we invoke a result of Diakonikolas, Kane, and Stewart, that guarantees the existence of a succinct (proper) cover of $\classpbd[n]$:
\begin{theorem}[{\cite[Theorem 4]{DKS:15}} (rephrased)]
  For all $n, \gamma >0$, there exists a set $\mathcal{S}_{\gamma} \subseteq \classpbd[n]$ such that:
  \begin{enumerate}[(i)]
  \item $\mathcal{S}_{\gamma}$ is a $\gamma$-cover of $\classpbd[n]$; that is, for all $\D \in \classpbd[n]$ there exists some $\D^\prime \in \mathcal{S}_{\gamma}$ such that $\normone{\D-\D^\prime} \leq \gamma$
  \item {$\abs{\mathcal{S}_{\gamma}} \leq n \left({1/\gamma}\right)^{\bigO{ \log{1/\gamma}}}$}
  \item $\mathcal{S}_{\gamma}$ can be computed in time {$n\left( {1/\gamma} \right)^{\bigO{\log{1/\gamma}}}$}
  \end{enumerate}
and each $\D\in\mathcal{S}_{\gamma}$  is explicitly described by its set of parameters.
\end{theorem}
\noindent We further observe that the factor $n$ in both the size of the cover and running time can be easily removed in our case, as we know a good approximation of the support size of the candidate PBDs. (That is, we only need to enumerate over a subset of the cover of~\cite{DKS:15}, that of the PBDs with effective support compatible with our distribution $\D$.)
\iffalse
Therefore, computing the $\lp[1]$ distance between $\D$ and any PBD with known parameters can be done efficiently. To conclude, we invoke a result of Daskalakis and Papadimitriou, that guarantees the existence of a succinct (proper) cover of $\classpbd[n]$:
\begin{theorem}[{\cite[Theorem 1]{DP:13}}]
  For all $n, \gamma >0$, there exists a set $\mathcal{S}_{\gamma} \subseteq \classpbd[n]$ such that:
  \begin{enumerate}[(i)]
  \item $\mathcal{S}_{\gamma}$ is a $\gamma$-cover of $\classpbd[n]$ in $\lp[1]$; that is, for all $\D \in \classpbd[n]$, there exists some $\D^\prime \in \mathcal{S}_{\gamma}$ such that $\normone{\D-\D^\prime} \leq \gamma$
  \item {$\abs{\mathcal{S}_{\gamma}} \le n^2 + n \cdot \left({1/\gamma}\right)^{\bigO{ \log^2{1/\gamma}}}$}
  \item $\mathcal{S}_{\gamma}$ can be computed in time {$\bigO{ n^2 \log n} + \bigO{n \log n} \cdot \left( {1/\gamma} \right)^{\bigO{\log{1/\gamma}}}$}
  \end{enumerate}
and each $\D\in\mathcal{S}_{\gamma}$  is explicitly described by its set of parameters.
\end{theorem}
\fi

Set $\gamma\eqdef\frac{\eps}{250}$. Fix $P\in\classpbd[n]$ such that $P(I) \geq 1-\eps$, and $Q\in \mathcal{S}_{\gamma}$ such that $\normone{P-Q}\leq \gamma$. In particular, it is easy to see via the correspondence between $\lp[1]$ and total variation distance that $\abs{P(I)-Q(I)} \leq \gamma/2$.
By a calculation similar to that of~\cref{lemma:conditioned:class}, we have 
\begin{align*}
  \normone{P_I-Q_I} &= \sum_{i\in I}\abs{\frac{P(i)}{P(I)} - \frac{Q(i)}{Q(I)}} 
  = \sum_{i\in I}\abs{\frac{P(i)}{P(I)} - \frac{Q(i)}{P(I)} + Q(i)\left( \frac{1}{P(I)} - \frac{1}{Q(I)} \right) } \\
  &= \sum_{i\in I}\abs{\frac{P(i)}{P(I)} - \frac{Q(i)}{P(I)} } \pm \sum_{i\in I} Q(i)\abs{ \frac{1}{P(I)} - \frac{1}{Q(I)} }
  = \frac{1}{P(I)}\left( \sum_{i\in I}\abs{ P(i) - Q(i) } \pm \abs{P(I)-Q(I)}\right) \\
  &= \frac{1}{P(I)}\left( \sum_{i\in I}\abs{ P(i) - Q(i) } \pm \frac{\gamma}{2}\right) = \frac{1}{P(I)}\left( \normone{ P - Q } \pm \frac{5\gamma}{2}\right) \\
  &\in [ \normone{ P - Q } - {5\gamma}/{2}, (1+2\eps)\left( \normone{ P - Q } + {5\gamma}/{2} \right) ]
  \end{align*}
where we used the fact that $\sum_{i\notin I}\abs{ P(i) - Q(i) } = 2\left(\sum_{i\notin I\colon P(i) > Q(i)} (P(i)-Q(i))\right) + Q(I)-P(I) \in [-2\gamma,2\gamma]$.
By the triangle inequality, this implies that the minimum of $\normone{P_I-\D_I}$ over the distributions $P$ of $\mathcal{S}_\eps$ with $P(I)\geq 1-(\eps+\gamma/2)$ will be within an additive $\bigO{\eps}$ of $\lp[1](\D,\classpbd[n]^{\eps,I})$. The fact that the former can be found (by enumerating over the cover of size $\left( {1/\eps} \right)^{\bigO{\log{1/\eps}}}$ by the above discussion, and for each distribution in the cover computing the vector of probabilities and the distance to $\D$) in time $O(n^2)\cdot\abs{S_\eps}=n^2 \cdot \left( {1/\eps} \right)^{\bigO{\log{1/\eps}}}$ concludes the proof.
\end{proof}
As previously mentioned, this approximation guarantee for $\lp[1](\D,\classpbd[n]^{\eps,I})$ is sufficient for the purpose of~\cref{algo:test:splittable}.

\begin{proofof}{\cref{coro:main:testing:effective:support:pbd}}
Combining the above, we invoke~\cref{theo:main:testing:effective:support} with $M(n,\eps)=O( \sqrt{n\log(1/\eps)} )$ (\cref{fact:pbd:effective:support}) and $L(\gamma,\zeta,m)=O\big( \frac{1}{\gamma}\log^2 \frac{m}{\zeta} \big)$ (\cref{theo:structural:unimodal}). This yields the claimed sample complexity; finally, the efficiency is a direct consequence of~\cref{lemma:distance:pbd}.
\end{proofof}
 
 
\todonote{Moved the LB section. Check for refs here and in this section in general!} 
 
\subsection{A Generic Tolerant Testing Upper Bound}\label{sec:toltesting:ub}
To conclude this work, we address the question of tolerant testing of distribution classes. In the same spirit as before, we focus on describing a generic approach to obtain such bounds, in a clean conceptual manner. The most general statement of the result we prove in this section is stated below, which we then instantiate to match the lower bounds from~\cref{sec:lowerbounds:tol}:

\begin{theorem}\label{theo:main:tol:testing:ub:almost}
Let \class be a class of distributions over $[n]$ for which the following holds: 
  \begin{enumerate}[(i)]
    \item there exists a semi-agnostic learner $\Learner$ for $\class$, with sample complexity $q_L(n,\eps, \delta)$ and ``agnostic constant''~$c$;
    \item for any $\eta\in[0,1]$, every distribution in \class has $\eta$-effective support of size at most $M(n,\eta)$.
  \end{enumerate}
Then, there exists an algorithm that, for any fixed $\kappa > 1$ and on input $\eps_1,\eps_2 \in (0,1)$ such that $\eps_2 \geq C \eps_1$, has the following guarantee (where $C > 2$ depends on $c$ and $\kappa$ only). The algorithm takes $\bigO{\frac{1}{(\eps_2-\eps_1)^2}\frac{m}{\log m}} + q_L(n,\frac{\eps_2-\eps_1}{\kappa}, \frac{1}{10})$ samples (where $m=M(n,\eps_1)$), and with probability at least $2/3$ distinguishes between \textsf{(a)} $\lp[1](\D,\class) \leq \eps_1$ and \textsf{(b)} $\lp[1](\D,\class) >~\eps_2$. (Moreover, one can take $C=(1+(5c+6)\frac{\kappa}{\kappa-1})$.)
\end{theorem}

\coromaintoltestingmlogm*

Applying now the theorem with $M(n,\eps)=\sqrt{n\log(1/\eps)}$ (as per~\cref{coro:main:testing:effective:support:pbd}), we obtain an improved upper bound for Binomial and Poisson Binomial distributions: 
\coromaintoltestingpbd*

\subparagraph{High-level idea} Somewhat similar to the lower bound framework described later in~\cref{sec:toltesting:ub}, the gist of the approach is to reduce the problem of tolerant testing membership of $\D$ to the \emph{class} \class to that of tolerant testing identity to a known \emph{distribution} -- namely, the distribution $\hat{\D}$ obtained after trying to agnostically learn $\D$. Intuitively, an agnostic learner for \class should result in a good enough hypothesis $\hat{\D}$ (i.e., $\hat{\D}$ close enough to both $\D$ and $\class$) when $\D$ is $\eps_1$-close to \class; but output a $\hat{\D}$ that is significantly far from either $\D$ or $\class$ when $\D$ is $\eps_2$-far from \class~--~sufficiently for us to be able to tell.
Besides the many technical details one has to control for the parameters to work out, one key element is the use of a tolerant testing algorithm for closeness of two distributions due to~\cite{ValiantValiant:11}, whose (tight) sample complexity scales as $n/\log n$ for a domain of size $n$. In order to get the right dependence on the effective support (required in particular for~\cref{coro:main:tol:testing:pbd}), we have to perform a first test to identify the effective support of the distribution and check its size, in order to only call this tolerant closeness testing algorithm on this much smaller subset. (This additional preprocessing step itself has to be carefully done, and comes at the price of a slightly worse constant $C=C(c,\kappa)$ in the statement of the theorem.)

\subsubsection{Proof of~\cref{theo:main:tol:testing:ub:almost}}

As described in the preceding section, the algorithm will rely on the ability to perform tolerant testing of equivalence between two unknown distributions (over some known domain of size $m$). This is ensured by an algorithm of Valiant and Valiant, restated below:
\begin{theorem}[{\cite[Theorem 3 and 4]{ValiantValiant:11}}]\label{theo:samp:closeness:tolerant}
    There exists an algorithm $\mathcal{E}$ which, given sampling access to two unknown distributions $\D_1,\D_2$ over $[m]$, satisfies the following. On input $\eps\in(0,1]$, it takes $O( \frac{1}{\eps^2}\frac{m}{\log m} )$ samples from $\D_1$ and $\D_2$, and outputs a value $\Delta$ such that $\dabs{\normone{\D_1-\D_2}-\Delta} \leq \eps$ with probability $1-1/\poly(m)$. (Furthermore, $\mathcal{E}$ runs in time $\poly(m)$.)
\end{theorem}

\noindent For the proof, we will also need this fact, similar to~\cref{lemma:conditioned:class}, which relates the distance of two distributions to that of their conditional distributions on a subset of the domain:
\begin{fact}\label{lemma:conditioned:distr:distances}
Let $\D$ and $P$ be distributions over $[n]$, and $I\subseteq[n]$ an interval such that $\D(I) \geq 1- \alpha$ and $P(I) \geq 1- \beta$. Then,
\begin{itemize}
  \item $\normone{\D_I - P_I} \leq \frac{3}{2}\frac{\normone{\D - P}}{\D(I)} \leq 3\normone{\D - P}$ (the last inequality for $\alpha \leq \frac{1}{2}$); and 
  \item $\normone{\D_I - P_I} \geq \frac{}{}\normone{\D - P} - 2(\alpha+\beta)$.
\end{itemize}
\end{fact}

\iftrue
\begin{proof}
To establish the first item, write:
  \begin{align*}
      \normone{\D_I - P_I} &=\sum_{i\in I}\abs{ \frac{\D(i)}{\D(I)} - \frac{P(i)}{P(I)} }
      = \frac{1}{\D(I)}\sum_{i\in I}\abs{ \D(i) - P(i) + P(i)\big(1- \frac{\D(I)}{P(I)}\big) } \\
      &\leq \frac{1}{\D(I)}\big( \sum_{i\in I}\abs{ \D(i) - P(i) } + \abs{1- \frac{\D(I)}{P(I)} } \sum_{i\in I} P(i) \big)\\
      &= \frac{1}{\D(I)}\big( \sum_{i\in I}\abs{ \D(i) - P(i) } + \abs{P(I)- \D(I) } \big)
      \leq \frac{1}{\D(I)}\big( \sum_{i\in I}\abs{ \D(i) - P(i) } + \frac{1}{2}\normone{\D-P} \big)\\
     &\leq \frac{1}{\D(I)} \cdot \frac{3}{2}\normone{\D-P} 
  \end{align*}
  where we used the fact that $\abs{P(I)- \D(I) }\leq \totalvardist{\D}{P} = \frac{1}{2}\normone{\D-P}$.
 Turning now to the second item, we have:
  \begin{align*}
      \normone{\D_I - P_I} &= \frac{1}{\D(I)}\sum_{i\in I}\abs{ \D(i) - P(i) + P(i)\mleft(1- \frac{\D(I)}{P(I)}\mright) } 
      \geq \frac{1}{\D(I)}\big( \sum_{i\in I}\abs{ \D(i) - P(i) } - \abs{1- \frac{\D(I)}{P(I)} } \sum_{i\in I} P(i) \big)\\
      &= \frac{1}{\D(I)}\big( \sum_{i\in I}\abs{ \D(i) - P(i) } - \abs{ P(I)- \D(I) } \big)
      \geq \frac{1}{\D(I)}\big( \sum_{i\in I}\abs{ \D(i) - P(i) } - (\alpha+\beta) \big)\\
     &\geq \frac{1}{\D(I)}\big( \normone{\D-P} - \sum_{i\notin I} \abs{ \D(i) - P(i) } - (\alpha+\beta) \big)
     \geq \frac{1}{\D(I)}\big( \normone{\D-P} - 2(\alpha+\beta) \big) \\
     &\geq \normone{\D-P} - 2(\alpha+\beta).  
  \end{align*}
\end{proof}
\fi

\noindent With these two ingredients, we are in position to establish our theorem:
\begin{proofof}{\cref{theo:main:tol:testing:ub:almost}}
The algorithm proceeds as follows, where we set $\eps\eqdef\frac{\eps_2-\eps_1}{17\kappa}$, $\theta\eqdef\eps_2 - ((6+c)\eps_1+11\eps)$, and $\tau \eqdef2 \frac{(3+c)\eps_1+5\eps}{2}$:
\begin{enumerate}[(1)]
  \item\label{algo:effectivesupport:toltest:step:0} using $O(\frac{1}{\eps^2})$ samples, get (with probability at least $1-1/10$, by~\cref{theo:dkw:ineq}) a distribution $\tilde{\D}$ $\frac{\eps}{2}$-close to $\D$ in Kolmogorov distance; and let $I\subseteq[n]$ be the smallest interval such that $\tilde{\D}(I) > 1-\frac{3}{2}\eps_1-\eps$. Output \reject if $\abs{I} > M(n,\eps_1)$.
  \item\label{algo:effectivesupport:toltest:step:1} invoke $\Learner$ on $\D$ with parameters $\eps$ and failure probability $\frac{1}{10}$, to obtain a hypothesis $\hat{\D}$;
  \item\label{algo:effectivesupport:toltest:step:2}  call $\mathcal{E}$ (from~\cref{theo:samp:closeness:tolerant}) on $\D_I$, $\hat{\D}_I$ with parameter $\frac{\eps}{6}$ to get an estimate $\hat{\Delta}$ of $\normone{\D_I-\hat{\D}_I}$;
  \item\label{algo:effectivesupport:toltest:step:2.5} output \reject if $\hat{\D}(I) < 1-\tau$; 
  \item\label{algo:effectivesupport:toltest:step:3}  compute ``offline'' (an estimate accurate within $\eps$ of) $\lp[1](\hat{\D},\class)$, denoted $\Delta$;
  \item\label{algo:effectivesupport:toltest:step:4}  output \reject is $\Delta+\hat{\Delta} > \theta$, and output \accept otherwise.
\end{enumerate}
The claimed sample complexity is immediate from Steps~\ref{algo:effectivesupport:toltest:step:1} and~\ref{algo:effectivesupport:toltest:step:2}, along with~\cref{theo:samp:closeness:tolerant}. Turning to correctness, we condition on both subroutines meeting their guarantee (i.e., $\normone{\D-\hat{\D}} \leq c\cdot\opt+\eps$ and $\normone{\D-\hat{\D}} \in [\hat{\Delta} - \eps,\hat{\Delta} + \eps]$), which happens with probability at least $8/10-1/\poly(n)\geq 3/4$ by a union bound. 

\subparagraph{Completeness} If $\lp[1](\D,\class) \leq \eps_1$, then $\D$ is $\eps_1$-close to some $P\in\class$, for which there exists an interval $J\subseteq[n]$ of size at most $M(n,\eps_1)$ such that $P(J) \geq 1-\eps_1$. It follows that $\D(J) \geq 1-\frac{3}{2}\eps_1$ (since $\abs{\D(J)-P(J)} \leq \frac{\eps_1}{2}$) and $\tilde{\D}(J) \geq 1-\frac{3}{2}\eps_1-2\cdot\frac{\eps}{2}\eps$; establishing existence of a good interval $I$ to be found (and Step~\ref{algo:effectivesupport:toltest:step:0} does not end with \reject). Additionally, $\normone{\D-\hat{\D}} \leq c\cdot\eps_1+\eps$ and by the triangle inequality this implies $\lp[1](\hat{\D},\class) \leq (1+c)\eps_1+\eps$.
  
  Moreover, as $\D(I) \geq \tilde{\D}(I) - 2\cdot\frac{\eps}{2} \geq 1-\frac{3}{2}\eps_1-2\eps$ and $\abs{\hat{\D}(I) - \D(I)} \leq \frac{1}{2}\normone{\D-\hat{\D}}$, we do have
  \[
    \hat{\D}(I) \geq 1-\frac{3}{2}\eps_1-2\eps - \frac{c\eps_1}{2}-\frac{\eps}{2} = 1-\tau
  \]
  and the algorithm does not reject in Step~\ref{algo:effectivesupport:toltest:step:2.5}. 
  To conclude, one has by~\cref{lemma:conditioned:distr:distances} that
  \[
      \normone{\D_I-\hat{\D}_I} \leq \frac{3}{2}\frac{\normone{\D-\hat{\D}}}{\D(I)} \leq \frac{3}{2}\frac{(c\eps_1+\eps)}{1-\frac{3}{2}\eps_1-2\eps} \leq 3(c\eps_1+\eps) \tag{for $\eps_1 < 1/4$, as $\eps < 1/17$}
  \]
  Therefore, $\Delta+\hat{\Delta} \leq \lp[1](\hat{\D},\class)+\eps + \normone{\D_I-\hat{\D}_I}+\eps \leq (4c+1)\eps_1+6\eps \leq \eps_2 - ((6+c)\eps_1+11\eps) = \theta$ (the last inequality by the assumption on $\eps_2,\eps_1$), and the tester accepts.
\subparagraph{Soundness} If $\lp[1](\D,\class) > \eps_2$, then we must have $\normone{\D-\hat{\D}} + \lp[1](\hat{\D},\class) > \eps_2$. If the algorithm does not already reject in Step~\ref{algo:effectivesupport:toltest:step:2.5}, then $\hat{\D}(I) \geq 1-\tau$. But, by~\cref{lemma:conditioned:distr:distances},
  \begin{align*}
      \normone{\D_I-\hat{\D}_I} &\geq \normone{ \D - \hat{\D} } - 2(\D(I^c) + \hat{\D}(I^c)) \geq \normone{\D_I-\hat{\D}_I} - 2\Big(\frac{3}{2}\eps_1 + 2\eps+\tau\Big) \\
      &= \normone{ \D - \hat{\D} } - ((6+c)\eps_1+9\eps) 
  \end{align*}
  we then have $\normone{\D_I-\hat{\D}_I} + \lp[1](\hat{\D},\class) > \eps_2  - ((6+c)\eps_1+9\eps)$. 
  This implies $\Delta+\hat{\Delta} > \eps_2 - ((6+c)\eps_1+9\eps) -2\eps = \eps_2 - ((6+c)\eps_1+11\eps)  = \theta$, and the tester rejects.
Finally, the testing algorithm defined above is computationally efficient as long as both the learning algorithm (Step~\ref{algo:effectivesupport:toltest:step:1}) and the estimation procedure (Step~\ref{algo:effectivesupport:toltest:step:3}) are.

\end{proofof}
 
  
\subsection{Proof of~\cref{theo:structural:mhr}}\label{app:structural:proofs}

In this section, we prove our structural result for MHR distributions,~\cref{theo:structural:mhr}:
\theostructuralmhr*
\begin{proof} We reproduce and adapt the argument of~\cite[Section 5.1]{CDSS:13} to meet our definition of decomposability (which, albeit related, is incomparable to theirs). First, we modify the algorithm at the core of their constructive proof, in~\cref{algo:mhr}: note that the only two changes are in Steps~\ref{algo:mhr:step:iiprime} and \ref{algo:mhr:step:iprimeprime}, where we use parameters respectively $\frac{\zeta}{n}$ and $\frac{\zeta}{n^2}$.
\begin{algorithm}
  \begin{algorithmic}[1]
    \Require explicit description of distribution $\D$ over $[n]$; interval $J=[a,b]\subseteq[n]$; threshold $\tau > 0$
    \If{$\D(b)>\tau$}
      \State Set $i^\prime\gets b$
    \Else
      \State Set $i^\prime\gets \min\setOfSuchThat{a\leq i\leq b}{ \D([i,b]) \leq \tau }$
    \EndIf
    \State \Return $[i',b]$
  \end{algorithmic}
  \caption{\label{algo:mhr:rightinterval:cdss} $\textsc{Right-Interval}(\D,J,\tau)$}
\end{algorithm}
\begin{algorithm}
  \begin{algorithmic}[1]
    \Require explicit description of MHR distribution $\D$ over $[n]$; accuracy parameter $\gamma > 0$
    \State Set $J\gets [n]$ and $\mathcal{Q}\gets \emptyset$.
    \State Let $I\gets\textsc{Right-Interval}(\D,J,\frac{\zeta}{n})$ and $I^\prime\gets\textsc{Right-Interval}(\D,J\setminus I,\frac{\zeta}{n})$. Set $J\gets J\setminus (I\cup I^\prime)$. \label{algo:mhr:step:iiprime}
    \State Set $i\in J$ to be the smallest integer such that $\D(i)\geq \frac{\zeta}{n^2}$. 
            If no such $i$ exists, let $I^{\prime\prime}\gets J$ and go to Step~\ref{algo:mhr:laststep}. 
            Otherwise, let $I^{\prime\prime}\gets \{1,\dots,i-1\}$ and $J\gets J\setminus I^{\prime\prime}$.
            \label{algo:mhr:step:iprimeprime}
    \While{ $J\neq\emptyset$ }\label{algo:mhr:whileloop}
      \State Let $j\in J$ bet the smallest integer such that $\D(j) \notin [\frac{1}{1+\gamma}, 1+\gamma]\D(i)$.
            If no such $j$ exists, let $I^{\prime\prime\prime}\gets J$; otherwise let $I^{\prime\prime\prime}\gets \{i,\dots, j-1\}$.
      \State Add $I^{\prime\prime\prime}$ to $\mathcal{Q}$ and set $J\gets J\setminus I^{\prime\prime\prime}$.
      \State Let $i\gets j$.
    \EndWhile
    \State \Return $\mathcal{Q}\cup\{I,I^{\prime},I^{\prime\prime}\}$\label{algo:mhr:laststep}
  \end{algorithmic}
  \caption{\label{algo:mhr} $\textsc{Decompose-MHR}^\prime(\D,\gamma)$}.
\end{algorithm}
Following the structure of their proof, we write $\mathcal{Q}=\{I_1,\dots,I_{\abs{\mathcal{Q}}}\}$ with $I_i=[a_i,b_i]$, and define $\mathcal{Q}^\prime=\setOfSuchThat{ I_i\in \mathcal{Q} }{ \D(a_i) > \D(a_{i+1})}$, $\mathcal{Q}^{\prime\prime}=\setOfSuchThat{ I_i\in \mathcal{Q} }{ \D(a_i) \leq \D(a_{i+1})}$.

\noindent We immediately obtain the analogues of their Lemmas~5.2 and~5.3:
\begin{lemma}\label{lemma:lemma52:cdss:13}
We have $\prod_{I_i\in\mathcal{Q}^\prime} \frac{\D(a_i)}{\D(a_{i+1})} \leq \frac{n}{\zeta}$.
\end{lemma}
\begin{lemma}
Step~\ref{algo:mhr:whileloop} of~\cref{algo:mhr} adds at most $\bigO{\frac{1}{\gamma}\log\frac{n}{\zeta}}$ intervals to $\mathcal{Q}$.
\end{lemma}
\begin{proof}[Sketch]
This derives from observing that now $\D(I\cup I^\prime) \geq \zeta/n$, which as in~\cite[Lemma~5.3]{CDSS:13} in turn implies 
\[
  1 \geq \frac{\zeta}{n}(1+\gamma)^{\abs{\mathcal{Q}^{\prime}}-1}
\]
so that $\abs{\mathcal{Q}^{\prime}} = \bigO{\frac{1}{\gamma}\log\frac{n}{\zeta}}$.

\noindent Again following their argument, we also get 
\[
  \frac{\D(a_{\abs{\mathcal{Q}}+1})}{\D(a_1)} = \prod_{I_i\in\mathcal{Q}^{\prime\prime}} \frac{\D(a_{i+1})}{\D(a_i)}\cdot \prod_{I_i\in\mathcal{Q}^\prime} \frac{\D(a_{i+1})}{\D(a_i)}
\]
by combining~\cref{lemma:lemma52:cdss:13} with the fact that $\D(a_{\abs{\mathcal{Q}}+1} \leq 1$ and that by construction $\D(a_i) \geq \zeta/n^2$, we get 
\[
    \prod_{I_i\in\mathcal{Q}^{\prime\prime}} \frac{\D(a_{i+1})}{\D(a_i)} \leq \frac{n}{\zeta} \cdot \frac{n^2}{\zeta} = \frac{n^3}{\zeta^2}\ .
\] 
But since each term in the product is at least $(1+\gamma)$ (by construction of $\mathcal{Q}$ and the definition of $\mathcal{Q}^{\prime\prime}$), this leads to
\[
  (1+\gamma)^{\abs{ \mathcal{Q}^{\prime\prime} }} \leq \frac{n^3}{\zeta^2}
\]
and thus $\abs{ \mathcal{Q}^{\prime\prime} } = \bigO{\frac{1}{\gamma}\log\frac{n}{\zeta}}$ as well.
\end{proof}

It remains to show that $\mathcal{Q}\cup\{I,I^{\prime},I^{\prime\prime}\}$ is indeed a good decomposition of $[n]$ for $\D$, as per~\cref{def:struct:dec:split}. Since by construction every interval in $\mathcal{Q}$ satisfies~\cref{def:struct:item:flat}, we only are left with the case of $I$, $I^{\prime}$ and $I^{\prime\prime}$. For the first two, as they were returned by $\textsc{Right-Interval}$ either \textsf{(a)} they are singletons, in which case~\cref{def:struct:item:flat} trivially holds; or \textsf{(b)} they have at least two elements, in which case they have probability mass at most $\frac{\zeta}{n}$ (by the choice of parameters for $\textsc{Right-Interval}$) and thus~\cref{def:struct:item:light} is satisfied. Finally, it is immediate to see that by construction $\D(I^{\prime\prime}) \leq n\cdot \zeta/n^2 = \zeta/n$, and~\cref{def:struct:item:light} holds in this case as well.
\end{proof}
 
\subsection{Proofs from~\cref{sec:structural}}\label{app:structural:projection:proofs}

This section contains the proofs omitted from~\cref{sec:structural}, namely the distance estimation procedures for $t$-piecewise degree-$d$ (\cref{theo:degreed:poly-projection}), monotone hazard rate (\cref{lemma:distance:mhr:eff}), and log-concave distributions (\cref{lemma:distance:log:eff}).

\subsubsection{Proof of~\cref{theo:degreed:poly-projection}}\label{app:structural:projection:proofs:degreed}

In this section, we prove the following:
\begin{theorem}[{\cref{theo:degreed:poly-projection}, restated}]\label{thm:project:single:poly}
Let $p$ be an $\ell$-histogram over $[-1,1)$. There is an algorithm $\textsc{ProjectSinglePoly}(d,\eps)$
which runs in time $\poly(\ell, d+1,1/\eps)$, and outputs a degree-$d$ polynomial $q$ which defines a pdf over $[-1,1) $
such that $\normone{p-q} \leq 3 \lp[1](p,\classpoly[d]) + O(\eps)$.
\end{theorem}
As mentioned in~\cref{sec:structural}, the proof of this statement is a rather straightforward adaptation of the proof of~\cite[Theorem 9]{CDSS:14}, with two differences: first, in our setting there is no uncertainty or probabilistic argument due to sampling, as we are provided with an explicit description of the histogram $p$. Second, Chan et al. require some ``well-behavedness'' assumption on the distribution $p$ (for technical reasons essentially due to the sampling access), that we remove here. Besides these two points, the proof is almost identical to theirs, and we only reproduce (our modification of) it here for the sake of completeness.
(Any error introduced in the process, however, is solely our responsibility.)
\begin{proof}
\newcommand{\rr}{{r}}
\newcommand{\pp}{{p}}
Some preliminary definitions will be helpful:
\begin{definition}[Uniform partition]
  Let $p$ be a subdistribution on an interval $I \subseteq [-1,1)$.
  A partition $\mathcal{I} = \{I_1, \dots, I_\ell\}$ of $I$ is
  \emph{$(p,\eta)$-uniform} if $p(I_j) \leq \eta$ for all $1\leq j\leq \ell$.
\end{definition}

We will also use the following notation:  For this subsection, let $I = {[-1,1)}$ ({$I$ will denote a subinterval of $[-1,1)$ when the results
are applied in the next subsection}).
We write $\|f\|^{(I)}_{1}$ to denote $\int_{I} |f(x)| dx$,
and we write $\dtv^{(I)}(p,q)$ to denote $\normone{p-q}^{(I)}/2$.
We write $\opt^{(I)}_{1,d}$ to denote the {infimum of the} 
distance $\normone{p-g}^{(I)}$ between $p$ and any degree-$d$
subdistribution $g$ on $I$ that satisfies $g(I) = p(I)$.

The key step of \textsc{ProjectSinglePoly} is 
Step~\ref{step:projectsinglepoly:find} where it calls the \textsc{FindSinglePoly} procedure.
In this procedure
$T_i(x)$ denotes the degree-$i$ Chebychev polynomial
of the first kind.
The function \textsc{FindSinglePoly} should be thought of
as the CDF of a ``quasi-distribution'' $f$; we say that
$f=F'$ is a ``quasi-distribution'' 
and not a \textit{bona fide} probability distribution because it is
not guaranteed to be non-negative everywhere on $[-1,1)$.  Step~\ref{step:findsinglepoly:2}
of \textsc{FindSinglePoly} processes
$f$ slightly to obtain a polynomial $q$ which is an actual distribution over
$[-1,1).$
\begin{algorithm}
  \caption{\textsc{ProjectSinglePoly}}
  \begin{algorithmic}[1]
    \Require parameters $d,\eps$; and the full description of an $\ell$-histogram $p$ over $[-1,1)$.
    \Ensure a degree-$d$ distribution $q$ such that $\dtv(p,q) \leq 3 \cdot \opt_{1,d} + O(\eps)$
    \State Partition $[-1,1)$ into $z = \Theta((d+1)/\eps)$ intervals $I_0 = [i_0,i_1), \dots, I_{z-1}=[i_{z-1},i_z)$, where $i_0=-1$ and $i_z=1$, such that
  for each $j \in \{1,\dots,z\}$ we have $p(I_j) = \Theta(\eps/(d+1))$ or ($\abs{I_j}=1$ and $p(I_j)=\bigOmega{\eps/(d+1)}$).
    \State\label{step:projectsinglepoly:find} Call \textsc{FindSinglePoly}($d$, $\eps$, $\eta:=\Theta(\eps/(d+1))$, $\{I_0,\dots,I_{z-1}\}$, $p$ and output the hypothesis $q$ that it returns.
  \end{algorithmic}
\end{algorithm}


\begin{algorithm}
  \caption{\textsc{FindSinglePoly}}
  \begin{algorithmic}[1]
    \Require  degree parameter $d$;
error parameter $\eps$; parameter $\eta$; $(p,\eta)$-uniform partition
$\mathcal{I}_I = \{I_1, \dots, I_{{z}}\}$ of interval $I$ into ${{z}}$ intervals {such that $\sqrt{
\eps z}\cdot \eta \leq \eps/2$}; a subdistribution $p$ on $I$

    \Ensure a number $\tau$ and a degree-$d$
subdistribution $q$ on $I$ such that $q(I) = p(I)$,
\[ 
  \opt^{(I)}_{1,d} \leq \normone{p-q}^{(I)} \leq 3\opt^{(I)}_{1,d} + \sqrt{\eps z (d+1)}
  \cdot \eta + {\rm error},
\]
$0 \leq \tau \leq \opt^{(I)}_{1,d}$ {and ${\rm error} = O({(d+1)}\eta)$}.

\State\label{step:findsinglepoly:1} Let $\tau$ be the solution to the following LP:
\[
\text{minimize~}\tau~\text{subject to the following constraints:}
\]
(Below 
$F(x) = \sum_{i=0}^{d+1} c_i T_i(x)$ where $T_i(x)$ is the degree-$i$
Chebychev polynomial of the first kind, and $f(x)=F'(x) =
\sum_{i=0}^{d+1} c_i T'_i(x)$.)

\begin{enumerate}[(a)]

\item \label{item:total} $F(-1)=0$ and $F(1)=p(I)$;

\item \label{item:phat} For each $0 \leq j < k \leq z$,
\begin{equation} \label{eq:agno-phat}
  \abs{ \left(p([i_j,i_k)) + \sum_{j\leq \ell < k} w_\ell \right) - (F(i_k) - F(i_j)) } \leq \sqrt{\eps \cdot (k-j)} \cdot \eta;
\end{equation}

\item \label{item:robust}
\begin{align}
  \sum_{0\leq \ell < {z}} w_\ell &= 0, \label{item:robust:zerosum} \\
  -y_\ell \leq w_\ell &\leq y_\ell \qquad \text{for all $0\leq \ell < {z}$,} \label{item:robust:absval} \\
  \sum_{0\leq \ell < {z}} y_\ell &\leq \tau \label{item:robust:lb};
\end{align}



\item \label{item:AK}  The constraints $|c_i| \leq \sqrt{2}$ for $i=0,\dots,d+1$;

\item \label{item:AK2} The constraints
\[
0 \leq F(z)  \leq 1 \quad \text{for all~} z \in J,
\]
where $J$ is a set of {$\bigO{(d+1)^6}$} equally spaced points across $[-1,1)$;

\item \label{item:nonneg-1} The constraints
\[
\sum_{i=0}^d c_i T'_i(x) \geq 0 \quad \text{for all~}x \in K,
\]
where $K$ is a set of $O((d+1)^2/\eps)$ equally spaced points across
$[-1,1)$.

\end{enumerate}

\State\label{step:findsinglepoly:2} Define
$q(x) = { \eps f(I)/\abs{I} + (1-\eps)f(x)}.$
Output $q$ as the hypothesis pdf.

\end{algorithmic}
\end{algorithm}



The rest of this subsection gives the proof of~\cref{theo:degreed:poly-projection}.
The claimed running time bound is obvious
 (the computation is dominated by
solving the $\poly(d,1/\eps)$-size LP in \textsc{ProjectSinglePoly}, with an additional term linear in $\ell$ when partitioning $[-1,1)$ in the initial first step),
so it suffices to prove correctness.

Before launching into the proof we give some intuition for the linear
program.
Intuitively $F(x)$ represents the cdf of a degree-$d$ polynomial
distribution $f$ where $f=F'.$  Constraint~\ref{item:total} captures the endpoint
constraints that any cdf must obey {if it has the same total weight as $p$}.
Intuitively, constraint~\ref{item:phat} ensures that for each interval $[i_j,i_k)$,
the value $F(i_k)-F(i_j)$ (which we may alternately write as
$f([i_j,i_k))$) is close to the weight 
$p([i_j,i_k))$ that the distribution 
puts on the interval.  
Recall that by assumption
$p$ is $\opt_{1,d}$-close to some degree-$d$ polynomial $r$.
Intuitively the variable $w_\ell$ represents $\int_{[i_\ell, i_{\ell+1})}
(r-p)$ (note that these values sum to zero by
constraint~\ref{item:robust}\eqref{item:robust:zerosum}, and $y_\ell$ represents the absolute value of $w_\ell$
(see constraint~\ref{item:robust}\eqref{item:robust:absval}).
The value $\tau$, which by constraint~\ref{item:robust}\eqref{item:robust:lb} is at least the 
sum of the $y_\ell$'s, represents a lower bound on 
$\opt_{1,d}.$
The constraints in~\ref{item:AK} and~\ref{item:AK2} reflect the fact that
as a cdf, $F$ should be bounded between 0 and 1 (more on this below),
and the~\ref{item:nonneg-1} constraints reflect the fact that the pdf $f=F'$ should be
everywhere nonnegative (again more on this below).

\medskip

We begin by observing that 
\textsc{ProjectSinglePoly} calls \textsc{FindSinglePoly} with input parameters that satisfy
\textsc{FindSinglePoly}'s input requirements:

\begin{enumerate}
\item [(I)] the non-singleton intervals $I_0,\dots,I_{z-1}$ are $(p,\eta)$-uniform; and
\item [(II)] the singleton intervals each have weight at least $\frac{\eta}{10}$.
\end{enumerate}

We then proceed to show that, from there, \textsc{FindSinglePoly}'s LP is feasible and has a high-quality
optimal solution.

\begin{lemma} \label{lem:feasible}
Suppose $p$ is an $\ell$-histogram over $[-1,1)$, so that conditions (I) and (II) above hold;
then the LP defined in Step~\ref{step:findsinglepoly:1} of \textsc{FindSinglePoly}
is feasible; and the optimal solution $\tau$ is at most $\opt_{1,d}$.
\end{lemma}

\begin{proof}
As above, let $r$ be a degree-$d$ polynomial pdf such that $\opt_{1,d}=
\normone{p-r}$ {and $r(I) = p(I)$}.We exhibit a feasible solution as follows:
take $F$ to be the cdf of {$\rr$} (a degree $d$ polynomial).
Take $w_\ell$ to be $\int_{[i_\ell,i_{\ell+1})} ({\rr-\pp})$,
and take $y_\ell$ to be $\abs{w_\ell}$.
Finally, take $\tau$ to be $\sum_{0 \leq \ell < {z}} y_\ell.$

We first argue feasibility of the above solution.  
We first take care of the easy constraints:
since $F$ is the cdf of a {sub}distribution over $I$ it is clear that
constraints~\ref{item:total} and~\ref{item:AK2} are satisfied,
and since both $r$ and $p$ are pdfs {with the same total weight} it is clear
that constraints~\ref{item:robust}\eqref{item:robust:zerosum} and~\ref{item:nonneg-1} are both satisfied. 
Constraints~\ref{item:robust}\eqref{item:robust:absval} and~\ref{item:robust}\eqref{item:robust:lb} also hold.  
So it remains to argue constraints~\ref{item:phat} and~\ref{item:AK}.

Note that constraint~\ref{item:phat} is equivalent to $p + (\rr - p) = \rr$ 
and $\rr$ satisfying $(\mathcal{I}, \eps/(d+1), \eps)$-inequalities, 
therefore this constraint is satisfied.

To see that constraint~\ref{item:AK} is satisfied we recall some of the analysis
of Arora and Khot~\cite[{Section~3}]{AK:03}.  This analysis shows that since
$F$ is a cumulative distribution function (and in particular a function bounded between 0 and 1 on $I$) each of its
Chebychev coefficients is at most $\sqrt{2}$ in magnitude.

To conclude the proof of the lemma we need to argue that 
$\tau \leq \opt_{1,d}$.
Since $w_\ell = \int_{[i_\ell,i_{\ell+1})} ({\rr-\pp})$ it
is easy to see that $\tau = \sum_{0 \leq \ell < {z}} y_\ell = \sum_{0
\leq \ell < {z}} |w_\ell| \leq \normone{\pp-\rr}$, and
hence indeed $\tau \leq \opt_{1,d}$ as required.
\end{proof}

Having established that with high probability the LP is indeed feasible,
henceforth we let $\tau$ denote the optimal solution to the LP and
$F$, $f$, $w_\ell$, $c_i$, $y_\ell$ denote the values in the optimal solution.
A simple argument (see e.g. the proof of {\cite[Theorem~8]{AK:03}}) gives that $\norminf{F}\leq 2$.
Given this bound on $\norminf{F}$, the Bernstein--Markov inequality implies that $\norminf{f} = \norminf{F^\prime}\leq O((d+1)^2)$.
Together with \ref{item:nonneg-1} this implies that
$f(z) \geq -\eps/2$ for all $z \in [-1,1)$.
Consequently $q(z) \geq 0$ for all $z \in [-1,1)$,
and
\[
\int_{-1}^1 q(x) dx = \eps + (1 - \eps) \int_{-1}^1 f(x)dx = \eps +
(1-\eps)(F(1)-F(-1)) = 1. \]
So $q(x)$ is indeed a degree-$d$ pdf.  To prove~\cref{theo:degreed:poly-projection} it remains to show that $\normone{p-q} \leq 3 \opt_{1,d} + O(\eps).$

We sketch the argument that we shall use to bound $\normone{p-q}$.
A key step in achieving this bound is to 
bound the $\norm{\cdot}_{\cal A}$ distance between $f$ and
$\widehat{p}_m + w$ where ${\cal A} = {\mathcal A_{d+1}}$ is the class of
all unions of $d+1$ intervals and $w$ is a function based on the $w_\ell$
values (see \eqref{eq:good} below).
If we can bound $\norm{(p+w)- f}_{\cal A} \leq O(\eps)$ then it will not be difficult to show that
$\norm{r - f}_{\cal A} \leq \opt_{1,d} + O(\eps)$.. 
Since $r$ and $f$ are both degree-$d$ polynomials we have 
$\normone{r - f} = 2\norm{r - f}_{\cal A}  \leq 2 \opt_{1,d} + O(\eps)$, 
so the triangle inequality (recalling that
$\normone{p-r} = \opt_{1,d}$) gives
$\normone{p-f} \leq 3 \opt_{1,d}+O(\eps).$
 From this point a simple argument 
(Proposition~\ref{prop:epsmixture}) gives that
$\normone{p-q} \leq \normone{p-f} + O(\eps)$, which gives the theorem.

We will use the following lemma {that translates $(\mathcal{I}, \eta,\eps)$-inequalities into a bound on $\mathcal A_{d+1}$ distance}.

\begin{lemma} \label{lem:ad-dist}
Let $\mathcal{I} = \{I_0=[i_0, i_1), \dots, I_{z-1}=[i_{z-1}, i_z)\}$ be a
$(p,\eta)$-uniform partition of $I$, possibly augmented with singleton intervals.
If $h\colon I\to \R$ and $p$ satisfy the $(\mathcal{I}, \eta,
\eps)$-inequalities, then
\[ {\norm{p-h}_{\mathcal A_{{d+1}}}^{(I)} \leq \sqrt{\eps z {(d+1)}}\cdot \eta + {\rm error},} \]
{where ${\rm error} = O({(d+1)}\eta)$}.
\end{lemma}

\begin{proof}
To analyze
$\norm{p-h}_{\mathcal A_{d+1}}$,
consider any union of ${d+1}$ 
disjoint non-overlapping intervals $S = J_1 \cup\dots \cup J_{d+1}$.
We will bound $\norm{ p - h }_{\mathcal A_{d+1}}$ 
by bounding $\abs{ p(S) - h(S)}$.

We lengthen intervals in $S$ slightly to obtain $T = J'_1 \cup \dots \cup J'_{{d+1}}$ 
so that each $J'_j$ is a union of intervals of the form $[i_\ell,i_{\ell+1})$.
Formally, if $J_j = [a,b)$, then $J'_j = [a',b')$, where $a' = \max_\ell \setOfSuchThat{ i_\ell }{ i_\ell \leq a }$ and $b' = \min_\ell \setOfSuchThat{ i_\ell }{ i_\ell \geq b }$.
We claim that
\begin{equation} \label{eq:lengthen}
  \abs{ p(S) - h(S) } \leq O({(d+1)}\eta) + \abs{ p(T) - f(T) } .
\end{equation}
Indeed, consider any interval of the form $J = [i_\ell, i_{\ell+1})$ 
such that $J \cap S \neq J \cap T$ (in particular, such an interval cannot be one of the singletons).  We have
\begin{equation} \label{eq:lengthen-single}
\abs{ p(J \cap S) - p(J \cap T) } \leq p(J) \leq {O(\eta)},
\end{equation}
where the first inequality uses non-negativity of $p$ 
and the second inequality follows from the bound
$p([i_\ell,i_{\ell + 1})) \leq \eta$.
The {$(\mathcal{I}, \eta, \eps)$-inequalities 
(between $h$ and $p$)}
implies that the inequalities in 
\eqref{eq:lengthen-single} also hold with $h$ in place of $p$.
Now \eqref{eq:lengthen} follows by 
adding \eqref{eq:lengthen-single} across all
$J = [i_\ell, i_{\ell+1})$ such that $J\cap S\neq J\cap T$
(there are at most $2{(d+1)}$ such intervals $J$), 
since each interval $J_j$ in $S$ can change at most two such
$J$'s when lengthened.

Now rewrite $T$ as a 
disjoint union of $s \leq {d+1}$ intervals
$[i_{L_1}, i_{R_1}) \cup \dots \cup [i_{L_s}, i_{R_s})$.
We have
\[ \abs{ p(T) - h(T) } \leq \sum_{j=1}^s \sqrt{R_j - L_j} \cdot \sqrt
\eps\eta \]
by {$(\mathcal{I}, \eta, \eps)$-inequalities between $p$ and $h$}.
Now observing that 
that $0 \leq L_1 \leq R_1 \cdots \leq L_s \leq R_s \leq t =
O((d+1)/\eps)$, we get that the largest possible value of $\sum_{j=1}^s
\sqrt{R_j - L_j}$ is $\sqrt{sz} \leq {\sqrt{{(d+1)}z}}$, 
so the RHS of
(\ref{eq:lengthen}) is at most $O({(d+1)}\eta) + {\sqrt{
{(d+1)}z\eps}\eta}$, as
desired.
\end{proof}

Recall from above that $F$, $f$, $w_\ell$, $c_i$, $y_\ell$, $\tau$
denote the values in the optimal solution.
We claim that 
\begin{equation}
\label{eq:good}
 \norm{ (p+ w) - f }_{\cal A} = O(\eps) ,
\end{equation}
where $w$ is the subdistribution 
which is constant on each $[i_\ell, i_{\ell+1})$
and has weight $w_\ell$ there, so in particular $\normone{w} \leq \tau \leq \opt_{1,d}$. Indeed, this equality follows by applying~\cref{lem:ad-dist} with ${h
= f-w}$.
{The lemma requires $h$ and $p$ to satisfy $(\mathcal{I}, \eta,
\eps)$-inequalities, which follows from constraint~\ref{item:phat} ($(\mathcal{I}, \eta,
\eps)$-inequalities between $p+w$ and $f$) and observing that
$(p+ w) - f = p- (f - w)$.
We have also used $\eta = \Theta(\eps/{(d+1)})$ 
to bound the {\rm error} term of the lemma by $O(\eps)$.}

Next, by the triangle inequality we have
{(writing ${\cal A}$ for ${\cal A}_{d+1}$)}
\[
\norm{ r - f }_{\cal A} \leq \norm{ r - (p+w) }_{\cal A} 
+ \norm{ (p+w) - f }_{\cal A}.
\]
The last term on the RHS has just been shown to be $O(\eps)$.
The first term is bounded by
\[ \| r-(p+w)\|_{\cal A} \leq \frac{1}{2}\normone{ r-(p+w) } 
\leq \frac{1}{2}(\normone{r-p} + \normone{w}) \leq \opt_{1,d}. \]
Altogether, we get that $\norm{ r - f }_{\cal A} \leq \opt_{1,d}+ O(\eps)$.

Since $r$ and $f$ are degree $d$ polynomials, $\normone{ r - f } = 2\norm{ r - f }_{\cal A} \leq 2\opt_{1,d}+ O(\eps)$.
This implies $\normone{ p - f } \leq \normone{p-r} + \normone{ r - f } \leq 3\opt_{1,d} +  O(\eps)$.
{Finally, we turn our quasidistribution $f$ which has value $\geq -\eps/2$
everywhere into a distribution $q$ (which is nonnegative), by redistributing
the weight.}
The following simple proposition {bounds the error incurred}.

\begin{proposition} \label{prop:epsmixture}
{Let $f$ and $p$ be any sub-quasidistribution on $I$.}
If $q = {\eps f(I)/\abs{I} + (1- \eps)f}$, then $\norm{q - p}_1 \leq \norm{f
- p}_1 + {\eps(f(I)+p(I))}$.
\end{proposition}

\begin{proof}
  We have
  \[ q - p = {\eps(f(I)/\abs{I} - p) + (1-\eps)(f - p)}. \]
  Therefore
  \[ \normone{ q - p } \leq { \eps \norm{f(I)/|I| - p}_1 + (1-\eps) \norm{ f
    - p }_1 \leq \eps(f(I)+p(I)) + \norm{ f - p }_1 } .
\qedhere \]
\end{proof}

We now have $\normone{ p - q } \leq \normone{ p-f } + O(\eps)$ by~\cref{prop:epsmixture},
concluding the proof of~\cref{theo:degreed:poly-projection}.
\end{proof}


\subsubsection{Proof of~\cref{lemma:distance:mhr:eff}}\label{app:structural:projection:proofs:mhr}

\lemmaefficientdistancemhr*

\begin{proof}
For convenience, let $\alpha \eqdef \eps^3$; we also write $[i,j]$ instead of $\{i,\dots,j\}$.

First, we note that it is easy to reduce our problem to the case where, in the completeness case, we have $P\in\classmhr$ such that $\normone{\D-P} \leq 2\eps$  and $\kolmogorov{\D}{P} \leq 2\alpha$; while in the soundness case $\lp[1](\D,\classmhr) \geq 99\eps$. Indeed, this can be done with a linear program on $\poly(k,\ell)$ variables, asking to find a $(k+\ell)$-histogram $\D^{\prime\prime}$ on a refinement of $\D$ and $\D^\prime$ minimizing the $\lp[1]$ distance to $\D$, under the constraint that the Kolmogorov distance to $\D^\prime$ be bounded by $\eps$. (In the completeness case, clearly a feasible solution exists, as $P$ is one.) We therefore follow with this new formulation: either
  \begin{enumerate}[\sf(a)]
    \item $\D$ is $\eps$-close to a monotone hazard rate distribution $P$ (in $\lp[1]$ distance) \emph{and} $\D$ is $\alpha$-close to $P$ (in Kolmogorov distance); and
    \item $\D$ is $32\eps$-far from monotone hazard rate
  \end{enumerate} 
where $\D$ is a $(k+\ell)$-histogram.\medskip


We then proceed by observing the following easy fact: suppose $P$ is a MHR distribution on $[n]$, i.e. such that the quantity $h_i \eqdef \frac{P(i)}{\sum_{j=i}^n P(i)}$, $i\in[n]$ is non-increasing. Then, we have
\begin{equation}\label{eq:mhr:parameterization}
  P(i) = h_i \prod_{j=1}^{i-1} (1-h_j), \qquad i\in[n].
\end{equation}
and there is a bijective correspondence between $P$ and $(h_i)_{i\in[n]}$.\medskip

We will write a linear program with variables $y_1,\dots,y_n$, with the correspondence $y_i\eqdef\ln(1-h_i)$. Note that with this parameterization, we get that if the $(y_i)_{i\in[n]}$ correspond to a MHR distribution $P$, then for $i\in[n]$
\[
  P([i,n]) = \prod_{j=1}^{i-1} e^{y_j} = e^{\sum_{j=1}^{i-1} y_j}
\]
and asking that $\ln(1-\eps) \leq \sum_{j=1}^{i-1} y_{j} - \ln \D([i,n]) \leq  \ln(1+\eps)$ amounts to requiring
\[
   P([i,n]) \in [1\pm\eps] \D([i,n]).
\]

We focus first on the completeness case, to provide intuition for the linear program. Suppose there exists $P\in\classmhr$ such $P\in\classmhr$ such that $\normone{\D-P} \leq \eps$  and $\kolmogorov{\D^\prime}{P} \leq \alpha$. This implies that for all $i\in[n]$, $\abs{ P([i,n]) - \D([i,n]) } \leq 2\alpha$. Define $I=\{b+1,\dots,n\}$ to be the longest interval such that $\D(\{b+1,\dots,n\})\leq \frac{\eps}{2}$. It follows that for every $i\in [n]\setminus I$,
\begin{equation}\label{eq:mhr:mult:approx:cdf}
    \frac{P([i,n])}{\D([i,n])} \leq \frac{\D([i,n])+2\alpha}{\D([i,n])} \leq 1+\frac{2\alpha}{\eps/2} = 1+4\eps^2 \leq 1+\eps
\end{equation}
and similarly
$
    \frac{P([i,n])}{\D([i,n])} \geq \frac{\D([i,n])-2\alpha}{\D([i,n]} \geq 1-\eps
$.
This means that for the points $i$ in $[n]\setminus I$, we can write constraints asking for multiplicative closeness (within $1\pm \eps)$ between $e^{\sum_{j=1}^{i-1} y_j}$ and $\D([i,n])$, which is very easy to write down as linear constraints on the $y_i$'s.

\subparagraph{The linear program} 
Let $T$ and $S$ be respectively the sets of ``light'' and ``heavy'' points, defined as $T=\setOfSuchThat{ i\in \{1,\dots,b\} }{ \D(i) \leq \eps^2 }$ and  $S=\setOfSuchThat{ i\in \{1,\dots,b\} }{ \D(i) > \eps^2 }$, where $b$ is as above. (In particular, $\abs{S} \leq 1/\eps^2$.)

\begin{algorithm}
\caption{\label{algo:lp:mhr}Linear Program}
  \begin{align}
  \text{Find }\qquad  & y_1,\dots,y_b & \notag\\
  \text{s.t.}\qquad & \hfill& \notag\\
   &y_i \leq 0 &     \label{lp:mhr:01}\\
   & y_{i+1} \leq y_{i} & \forall i \in \{1,\dots,b-1\}     \label{lp:mhr}\\
   & \ln\!\left(1-\eps\right)  \leq \sum_{j=1}^{i-1} y_{j} - \ln \D([i,n]) \leq  \ln\!\left(1+\eps\right) & \forall i \in \{1,\dots,b\}     \label{lp:mhr:mult:close}\\
   &\frac{\D(i)-\eps_i}{(1+\eps)\D[i,n]} \leq -y_i \leq (1+4\eps)\frac{\D(i)+\eps_i}{(1-\eps)\D[i,n]}      & \forall i\in T  \label{lp:mhr:bound:yi}\\
    &\sum_{i\in T} \eps_i \leq \eps \label{lp:mhr:bound:sum:epsi}\\
    & 0 \leq \eps_i \leq 2\alpha & \forall i\in T \label{lp:mhr:noneg:epsi}\\
    & \ln\left( 1-\frac{\D(i)+2\alpha}{(1-\eps)\D[i,n]} \right) \leq y_i \leq \ln\left( 1-\frac{\D(i)-2\alpha}{(1+\eps)\D[i,n]} \right) & \forall i\in S \label{lp:mhr:heavypoints}
  \end{align}
\end{algorithm}

Given a solution to the linear program above, define $\tilde{P}$ (a non-normalized probability distribution) by setting $\tilde{P}(i) = (1-e^{y_i})e^{\sum_{j=1}^{i-1} y_j}$ for $i\in \{1,\dots,b\}$, and $\tilde{P}(i) = 0$ for $i\in I = \{b+1,\dots, n\}$. A MHR distribution is then obtained by normalizing $\tilde{P}$. 

\subparagraph{Completeness} Suppose $P\in\classmhr$ is as promised. In particular, by the Kolmogorov distance assumption we know that every $i\in T$ has $P(i) \leq \eps^2+2\alpha < 2\eps^2$.
\begin{itemize}
  \item For any $i\in T$, we have that $\frac{P(i)}{P[i,n]} \leq \frac{2\eps^2}{(1-\eps)\eps} \leq 4\eps$, and 
\begin{equation}
  \frac{\D(i)-\eps_i}{(1+\eps)\D[i,n]} \leq \frac{P(i)}{P[i,n]} \leq \underbrace{-\ln(1-\frac{P(i)}{P[i,n]})}_{-y_i}
  \leq (1+4\eps)\frac{P(i)}{P[i,n]} = (1+4\eps)\frac{\D(i)+\eps_i}{P[i,n]} \leq \frac{1+4\eps}{1-\eps}\frac{\D(i)+\eps_i}{\D[i,n]}
\end{equation}
where we used~\cref{eq:mhr:mult:approx:cdf} for the two outer inequalities; and so~\eqref{lp:mhr:bound:yi},~\eqref{lp:mhr:bound:sum:epsi}, and~\eqref{lp:mhr:noneg:epsi} would follow from setting $\eps_i \eqdef \abs{P(i)-\D(i)}$ (along with the guarantees on $\lp[1]$ and Kolmogorov distances between $P$ and $\D$).
  \item For $i\in S$, Constraint~\eqref{lp:mhr:heavypoints} is also met, as 
  $\frac{P(i)}{P([i,n])} \in \left[\frac{\D(i)-2\alpha}{P([i,n])},\frac{\D(i)+2\alpha}{P([i,n])}\right] 
    \subseteq \left[\frac{\D(i)-2\alpha}{(1+\eps)\D([i,n])},\frac{\D(i)+2\alpha}{(1-\eps)\D([i,n])}\right]$.
\end{itemize}

\subparagraph{Soundness}
\noindent Assume a feasible solution to the linear program is found. We argue that this implies $\D$ is $\bigO{\eps}$-close to some MHR distribution, namely to the distribution obtained by renormalizing $\tilde{P}$.

In order to do so, we bound separately the $\lp[1]$ distance between $\D$ and $\tilde{P}$, from $I$, $S$, and $T$. 
First, $\sum_{i\in I} \abs{\D(i) - \tilde{P}(i)}  = \sum_{i\in I} \D(i) \leq \frac{\eps}{2}$ by construction.
For $i\in T$, we have $\frac{\D(i)}{\D[i,n]} \leq \eps$, and
\begin{align*}
  \tilde{P}(i) = (1-e^{y_i)}) e^{\sum_{j=1}^{i-1} y_j} \in \left[1\pm \eps\right] (1-e^{y_i}) \D([i,n]).
\end{align*}
Now,
\begin{align*}
  1-(1-\eps)\frac{\D(i)-\eps_i}{(1+\eps)\D[i,n]} \geq e^{-\frac{\D(i)-\eps_i}{(1+\eps)\D[i,n]}} 
  \geq e^{y_i} 
  \geq e^{-(1+4\eps)\frac{\D(i)+\eps_i}{(1-\eps)\D[i,n]}} 
  \geq 1-(1+4\eps)\frac{\D(i)+\eps_i}{(1-\eps)\D[i,n]}
\end{align*}
so that
\[
  (1-\eps)\frac{(1-\eps)}{(1+\eps)}(\D(i)-\eps_i)
  \leq
  \tilde{P}(i) 
  \leq
  (1+4\eps)\frac{(1+\eps)}{(1-\eps)}(\D(i)+\eps_i)
\]
which implies
\[
  (1-10\eps)(\D(i)-\eps_i)
  \leq
  \tilde{P}(i) 
  \leq
  (1+10\eps)(\D(i)+\eps_i)
\]
so that $\sum_{i\in T} \abs{\D(i) - \tilde{P}(i)} \leq 10\eps \sum_{i\in T} \D(i) + (1+10\eps)\sum_{i\in T} \eps_i \leq 10\eps + (1+10\eps)\eps \leq 20\eps$
where the last inequality follows from Constraint~\eqref{lp:mhr:bound:sum:epsi}.

To analyze the contribution from $S$, we observe that Constraint~\eqref{lp:mhr:heavypoints} implies that, for any $i\in S$,
\[
    \frac{\D(i)-2\alpha}{(1+\eps)\D([i,n])} \leq \frac{\tilde{P}(i)}{\tilde{P}([i,n])} \leq \frac{\D(i)+2\alpha}{(1-\eps)\D([i,n])}
\]
which combined with Constraint~\eqref{lp:mhr:mult:close} guarantees
\[
    \frac{\D(i)-2\alpha}{(1+\eps)^2\tilde{P}([i,n])} \leq \frac{\tilde{P}(i)}{\tilde{P}([i,n])} \leq \frac{\D(i)+2\alpha}{(1-\eps)^2\tilde{P}([i,n])}
\]
which in turn implies that $\abs{\tilde{P}(i) - \D(i) } \leq 3\eps\tilde{P}(i) + 2\alpha$. Recalling that $\abs{S} \leq \frac{1}{\eps^2}$ and $\alpha=\eps^3$, this yields
$\sum_{i\in S} \abs{\D(i) - \tilde{P}(i)} \leq 3\eps \sum_{i\in S}  \tilde{P}(i)  + 2\eps \leq 3\eps(1+\eps) + 2\eps \leq 8\eps$. Summing up, we get
$
  \sum_{i=1}^n \abs{\D(i) - \tilde{P}(i)} \leq 30\eps
$
which finally implies by the triangle inequality that the $\lp[1]$ distance between $\D$ and the normalized version of $\tilde{P}$ (a valid MHR distribution) is at most $32\eps$.

\subparagraph{Running time} The running time is immediate, from executing the two linear programs on $\poly(n,1/\eps)$ variables and constraints.
\end{proof}


\subsubsection{Proof of~\cref{lemma:distance:log:eff}}\label{app:structural:projection:proofs:log}

\lemmaefficientdistancelog*
\begin{proof}
We set $\alpha \eqdef \frac{\eps^2}{\log^2(1/\eps)}$, $\beta \eqdef \frac{\eps^2}{\log(1/\eps)}$, and $\gamma \eqdef \frac{\eps^2}{10}$ (so that $\alpha \ll \beta \ll \gamma \ll \eps$),

Given the explicit description of a distribution $\D$ on $[n]$, which a $k$-histogram over a partition $\mathcal{I}=(I_1,\dots, I_k)$ of $[n]$ with $k=\poly(\log n, 1/\eps)$ and the explicit description of a distribution $\D^\prime$ on $[n]$, one must \emph{efficiently} distinguish between:
\begin{enumerate}[\sf(a)]
  \item $\D$ is $\eps$-close to a log-concave $P$ (in $\lp[1]$ distance) \emph{and} $\D^\prime$ is $\alpha$-close to $P$ (in Kolmogorov distance); and
  \item $\D$ is $100\eps$-far from log-concave.
\end{enumerate} 
If we are willing to pay an extra factor of $\bigO{n}$, we can assume without loss of generality that we know the mode of the closest log-concave distribution (which is implicitly assumed in the following: the final algorithm will simply try all possible modes).

\subparagraph{Outline} First, we argue that we can simplify to the case where $\D$ is unimodal. Then, reduce to the case where where $\D$ and $\D^\prime$ are only one distribution, satisfying both requirements from the completeness case. Both can be done efficiently (\cref{stage:1}), and make the rest much easier.
Then, perform some \emph{ad hoc} partitioning of $[n]$, using our knowledge of $\D$, into $\tildeO{1/\eps^2}$ pieces such that each piece is either a ``heavy'' singleton, or an interval $I$ with weight very close (multiplicatively) to $\D(I)$ \emph{under the target log-concave distribution, if it exists} (\cref{stage:2}). This in particular simplifies the type of log-concave distribution we are looking for: it is sufficient to look for distributions putting that very specific weight on each piece, up to a $(1+o(1))$ factor. Then, in \cref{stage:3}, we write and solve a linear program to try and find such a ``simplified'' log-concave distribution, and reject if no feasible solution exists.

Note that the first two sections allow us to argue that instead of additive (in $\lp[1]$) closeness, we can enforce constraints on \emph{multiplicative} (within a $(1+\eps)$ factor) closeness between $\D$ and the target log-concave distribution. This is what enables a linear program with variables being the logarithm of the probabilities, which plays very nicely with the log-concavity constraints. \medskip

\noindent We will require the following result of Chan, Diakonikolas, Servedio, and Sun:
\begin{theorem}[{\cite[Lemma 4.1]{CDSS:13}}]\label{lemma:cdss13:41}
Let $\D$ be a distribution over $[n]$, log-concave and non-decreasing over $\{1,\dots,b\} \subseteq [n]$. Let $a\leq b$ such that
 $\sigma = \D(\{1,\dots,a-1\}) > 0$, and write $\tau=\D(\{a,\dots,b\})$. Then 
 		$\frac{\D(b)}{\D(a)} \leq 1+\frac{\tau}{\sigma}$.
\end{theorem}

\paragraph{Step 1}\label{stage:1}

\subparagraph{Reducing to $\D$ unimodal}
Using a linear program, find a closest \emph{unimodal} distribution $\tilde{\D}$ to $\D$ (also a $k$-histogram on $\mathcal{I}$) under the constraint that $\kolmogorov{\D}{P} \leq \alpha$: this can be done in time $\poly(k)$. If $\normone{\D-\tilde{\D}} > \eps$, output \reject.

\begin{itemize}
  \item If $\D$ is $\eps$-close to a log-concave distribution $P$ as above, then it is in particular $\eps$-close to unimodal and we do not reject. Moreover, by the triangle inequality $\normone{\tilde{\D} - P} \leq 2\eps$ and $\kolmogorov{\tilde{\D}}{P} \leq 2\alpha$.
  \item If $\D$ is $100\eps$-far from log-concave and we do not reject, then $\lp[1](\tilde{\D},\classlogconcave) \geq 99\eps$.
\end{itemize}

\subparagraph{Reducing to $\D=\D^\prime$}
First, we note that it is easy to reduce our problem to the case where, in the completeness case, we have $P\in\classlogconcave$ such that $\normone{\D-P} \leq 4\eps$  and $\kolmogorov{\D}{P} \leq 4\alpha$; while in the soundness case $\lp[1](\D,\classlogconcave) \geq 97\eps$. Indeed, this can be done with a linear program on $\poly(k,\ell)$ variables and constraints, asking to find a $(k+\ell)$-histogram $\D^{\prime\prime}$ on a refinement of $\D$ and $\D^\prime$ minimizing the $\lp[1]$ distance to $\D$, under the constraint that the Kolmogorov distance to $\D^\prime$ be bounded by $2\alpha$. (In the completeness case, clearly a feasible solution exists, as (the flattening on this $(k+\ell)$-interval partition) of $P$ is one.) We therefore follow with this new formulation: either
  \begin{enumerate}[\sf(a)]
    \item $\D$ is $4\eps$-close to a log-concave $P$ (in $\lp[1]$ distance) \emph{and} $\D$ is $4\alpha$-close to $P$ (in Kolmogorov distance); and
    \item $\D$ is $97\eps$-far from log-concave;
  \end{enumerate} 
where $\D$ is a $(k+\ell)$-histogram.\medskip

\noindent This way, we have reduced the problem to a slightly more convenient one, that of~\cref{stage:2}.

\subparagraph{Reducing to knowing the support $[a,b]$}
The next step is to compute a good approximation of the support of any target log-concave distribution. This is easily obtained in time $O(k)$ as the interval $\{a,\cdots,b\}$ such that
\begin{itemize}
  \item $\D(\{1,\dots,a-1\}) \leq \alpha$ but $\D(\{1,\dots,a\}) > \alpha$; and
  \item $\D(\{b+1,\dots,\}n) \leq \alpha$ but $\D(\{b,\dots,n\}) > \alpha$.
\end{itemize} 
Any log-concave distribution that is $\alpha$-close to $\D$ must include  $\{a,\cdots,b\}$ in its support, since otherwise the $\lp[1]$ distance between $\D$ and $P$ is already greater than $\alpha$. Conversely, if $P$ is a log-concave distribution $\alpha$-close to $\D$, it is easy to see that the distribution obtained by setting $P$ to be zero outside $\{a,\cdots,b\}$ and renormalizing the result is still log-concave, and $O(\alpha)$-close to $\D$.

\paragraph{Step 2}\label{stage:2}
Given the explicit description of a \emph{unimodal} distribution $\D$ on $[n]$, which a $k$-histogram over a partition $\mathcal{I}=(I_1,\dots, I_k)$ of $[n]$ with $k=\poly(\log n, 1/\eps)$, one must \emph{efficiently} distinguish between:
  \begin{enumerate}[\sf(a)]
    \item $\D$ is $\eps$-close to a log-concave $P$ (in $\lp[1]$ distance) and $\alpha$-close to $P$ (in Kolmogorov distance); and
    \item $\D$ is $24\eps$-far from log-concave,
  \end{enumerate} 
  assuming we know the mode of the closest log-concave distribution, which has support $[n]$.

In this stage, we compute a partition $\mathcal{J}$ of $[n]$ into $\tildeO{1/\eps^2}$ intervals (here, we implicitly use the knowledge of the mode of the closest log-concave distribution, in order to apply~\cref{lemma:cdss13:41} differently on two intervals of the support, corresponding to the non-decreasing and non-increasing parts of the target log-concave distribution).

As $\D$ is unimodal, we can efficiently ($\bigO{\log k}$) find the interval $S$ of heavy points, that is 
\[
  S\eqdef \setOfSuchThat{ x \in [n] }{  \D(x) \geq \beta }.
\]
Each point in $S$ will form a singleton interval in our partition. 
Let $T\eqdef [n]\setminus S$ be its complement ($T$ is the union of at most two intervals $T_1,T_2$ on which $\D$ is monotone, the head and tail of the distribution). For convenience, we focus on only one of these two intervals, without loss of generality the ``head'' $T_1$ (on which $\D$ is non-decreasing).

\begin{enumerate}
  \item Greedily find $J=\{1,\dots,a\}$, the smallest prefix of the distribution satisfying $\D(J)\in\left[\frac{\eps}{10}-\beta, \frac{\eps}{10}\right]$.
  \item Similarly, partition $T_1\setminus J$ into intervals $I^\prime_1,\dots,I^\prime_s$ (with $s=\bigO{1/\gamma}=\bigO{1/\eps^2}$) such that
    $
      \frac{\gamma}{10} \leq \D(I^\prime_j) \leq \frac{9}{10}\gamma
    $
    for all $1\leq j \leq s-1$, and $\frac{\gamma}{10} \leq \D(I^\prime_s) \leq \gamma$. This is possible as all points not in $S$ have weight less than $\beta$, and $\beta \ll \gamma$.
\end{enumerate}

\subparagraph{Discussion: why doing this?}\label{ssec:logconcave:completeness}
We focus on the completeness case: let $P\in\classlogconcave$ be a log-concave distribution such that $\normone{\D-P} \leq \eps$ and $\kolmogorov{\D}{P} \leq \alpha$.
Applying~\cref{lemma:cdss13:41} on $J$ and the $I^\prime_j$'s, we obtain (using the fact that $\abs{P(I^\prime_j) - \D(I^\prime_j)} \leq 2\alpha$) that:
\[
    \frac{\max_{x\in I^\prime_j} P(x)}{\min_{x\in I^\prime_j} P(x)} 
    \leq 1+\frac{\D(I^\prime_j)+2\alpha}{\D(J)-2\alpha} 
    \leq 1 + \frac{\gamma+2\alpha}{\frac{\eps}{10}-2\alpha}
    = 1+ \eps + \bigO{\frac{\eps^2}{\log^2(1/\eps)}} \eqdef 1+\kappa.
\]
Moreover, we also get that each resulting interval $I^\prime_j$ will satisfy
\[
      \D(I^\prime_j)(1-\kappa_j) = \D(I^\prime_j)-2\alpha \leq P(I^\prime_j) \leq \D(I^\prime_j)+2\alpha = \D(I^\prime_j)(1+\kappa_j)
\]
with $\kappa_j \eqdef \frac{2\alpha}{\D(I^\prime_j)} = \bigTheta{1/\log^2(1/\eps)}$.\medskip

Summing up, we have a partition of $[n]$ into $\abs{S}+2 = \tildeO{1/\eps^2}$ intervals such that:
\begin{itemize}
  \item The (at most) two end intervals have $\D(J)\in\left[\frac{\eps}{10}-\beta, \frac{\eps}{10}\right]$, and thus $P(J)\in\left[\frac{\eps}{10}-\beta-2\alpha, \frac{\eps}{10}+2\alpha\right]$;
  \item the $\tildeO{1/\eps^2}$ singleton-intervals from $S$ are points $x$ with $\D(x) \geq \beta$, so that $P(x) \geq \beta -2\alpha \geq \frac{\beta}{2}$;
  \item each other interval $I=I^\prime_j$ satisfies 
  \begin{equation}\label{eq:logconcave:completeness:1}
    (1-\kappa_j) \D(I) \leq P(I) \leq (1+\kappa_j) \D(I)
  \end{equation}
  with $\kappa_j=\bigO{1/\log^2(1/\eps)}$; and
  \begin{equation}\label{eq:logconcave:completeness:2}
  \frac{\max_{x\in I}P(x)}{\min_{x\in I}P(x)} \leq 1+\kappa < 1+\frac{3}{2}\eps.
  \end{equation}
\end{itemize}
We will use in the constraints of the linear program the fact that $(1+\frac{3}{2}\eps)(1+\kappa_j) \leq 1+2\eps$, and $\frac{1-\kappa_j}{1+\frac{3}{2}\eps} \geq \frac{1}{1+2\eps}$.

\paragraph{Step 3}\label{stage:3}

We start by computing the partition $\mathcal{J}=(J_1,\dots,J_{\ell})$ as in~\cref{stage:2}; with $\ell=\tildeO{1/\eps^2}$; and write $J_j=\{a_j,\dots,b_j\}$ for all $j\in[\ell]$. We further denote by $S$ and $T$ the set of heavy and light points, following the notations from~\cref{stage:2}; and let $T^\prime \eqdef T_1\cup T_2$ be the set obtained by removing the two ``end intervals'' (called $J$ in the previous section) from $T$.

\begin{algorithm}
\caption{\label{algo:lp:logconcave}Linear Program}
\begin{align}
\text{Find }\qquad  &x_1,\dots,x_n, \eps_1,\dots,\eps_{\abs{S}} \notag\\
\text{s.t.}\qquad & \hfill& \notag\\
 &x_i \leq 0      \label{lp:01}\\
 &x_{i}-x_{i-1} \geq x_{i+1}-x_{i} & \forall i \in [n]     \label{lp:logconcave}\\
 &-\ln(1+2\eps) \leq x_i - \mu_j \leq \ln(1+2\eps), & \forall j\in T^\prime, \forall i \in J_j     \label{lp:right:weight:light:js}\\
 &-2\frac{\eps_i}{\D(i)} \leq x_i - \ln \D(i) \leq \frac{\eps_i}{\D(i)}, & \forall i\in S    \label{lp:right:weight:heavy:js}\\
 &\sum_{i\in S} \eps_i \leq \eps \label{lp:bound:sum:epsi}\\
 & 0 \leq \eps_i \leq 2\alpha & \forall i\in S \label{lp:noneg:epsi}\\
\end{align}
where $\mu_j\eqdef \ln\frac{\D(J_j)}{\abs{J_j}}$ for $j\in T^\prime$.\medskip
\end{algorithm}

\begin{lemma}[Soundness]\label{lemma:lp:logconcave:soundness}
If the linear program (\cref{algo:lp:logconcave}) has a feasible solution, then $\lp[1](\D, \classlogconcave)\leq \bigO{\eps}$.
\end{lemma}
\begin{proof}
\noindent A feasible solution to this linear program will define (setting $p_i=e^{x_i}$) a sequence $p=(p_1,\dots,p_n) \in (0,1]^n$ such that
\begin{itemize}
  \item $p$ takes values in $(0,1]$ (from \eqref{lp:01});
  \item $p$ is log-concave (from \eqref{lp:logconcave});
  \item $p$ is ``$(1+O(\eps))$-multiplicatively constant'' on each interval $J_j$ (from \eqref{lp:right:weight:light:js});
  \item $p$ puts roughly the right amount of weight on each $J_i$:
    \begin{itemize}
      \item weight $(1\pm O(\eps))\D(J)$ on every $J$ from $T$ (from~\eqref{lp:right:weight:light:js}), so that the $\lp[1]$ distance between $\D$ and $p$ coming from $T^\prime$ is at most $O(\eps)$;
      \item it puts weight approximately $\D(J)$ on every singleton $J$ from $S$, i.e. such that $\D(J) \geq \beta$. To see why, observe that each $\eps_i$ is in $[0,2\alpha]$ by constraints~\eqref{lp:noneg:epsi}. In particular, this means that $\frac{\eps_i}{\D(i)} \leq 2\frac{\alpha}{\beta} \ll 1$, and 
      we have
      \[
           \D(i) - 4\eps_i \leq \D(i)\cdot e^{-4\frac{\eps_i}{\D(i)}} \leq p_i = e^{x_i} \leq \D(i)\cdot e^{2\frac{\eps_i}{\D(i)}} \leq \D(i)+4\eps_i
      \]
      
       and together with~\eqref{lp:bound:sum:epsi} this guarantees that the $\lp[1]$ distance between $\D$ and $p$ coming from $S$ is at most $\eps$.
    \end{itemize}
\end{itemize}
Note that the solution obtained this way may not sum to one -- i.e., is not necessarily a probability distribution. However, it is easy to renormalize $p$ to obtain a \emph{bona fide} probability distribution $\tilde{P}$ as follows: set $\tilde{P} = \frac{p(i)}{\sum_{i\in S\cup T^\prime} p(i)}$ for all $i\in S\cup T^\prime$, and $p(i) =0$ for $i\in T\setminus T^\prime$.

Since by the above discussion we know that $p(S\cup T^\prime)$ is within $\bigO{\eps}$ of $\D(S\cup T^\prime)$ (itself in $[1-\frac{9\eps}{5}, 1+\frac{9\eps}{5}]$ by construction of $T^\prime$), $\tilde{P}$ is a log-concave distribution such that $\normone{\tilde{P}-\D} = \bigO{\eps}$.
\end{proof}

\begin{lemma}[Completeness]\label{lemma:lp:logconcave:completeness}
If there is $P$ in $\classlogconcave$ such that $\normone{\D-P}\leq \eps$ and $\kolmogorov{\D}{P}\leq \alpha$, then the linear program (\cref{algo:lp:logconcave}) has a feasible solution.
\end{lemma}
\begin{proof}
 Let $P\in\classlogconcave$ such that $\normone{\D - P}\leq \eps$ and $\kolmogorov{\D}{P}\leq \alpha$.
 Define $x_i\eqdef \ln P(i)$ for all $i\in[n]$. Constraints~\eqref{lp:01} and~\eqref{lp:logconcave} are immediately satisfied, since $P$ is log-concave. By the discussion from~\cref{ssec:logconcave:completeness} (more specifically, Eq.~\eqref{eq:logconcave:completeness:1} and~\eqref{eq:logconcave:completeness:2}), constraint~\eqref{lp:right:weight:light:js} holds as well.
 
 Letting $\eps_i\eqdef \abs{P(i)-\D(i)}$ for $i\in S$, we also immediately have~\eqref{lp:bound:sum:epsi} and~\eqref{lp:noneg:epsi} (since $\normone{P-\D} \leq \eps$ and $\kolmogorov{\D}{P}\leq \alpha$ by assumption). Finally, to see why~\eqref{lp:right:weight:heavy:js} is satisfied, we rewrite
 \[
      x_i - \ln\D(i) = \ln\frac{P(i)}{\D(i)} =  \ln\frac{\D(i)\pm\eps_i}{\D(i)} = \ln(1\pm \frac{\eps_i}{\D(i)})
 \] 
 and use the fact that $\ln(1+x) \leq x$ and $\ln(1-x) \geq -2x$ (the latter for $x < \frac{1}{2}$, along with $\frac{\eps_i}{\D(i)} \leq \frac{2\alpha}{\beta} \ll 1$).
\end{proof}

\paragraph{Putting it all together: Proof of~\cref{lemma:distance:log:eff}}

The algorithm is as follows (keeping the notations from~\cref{stage:1} to~\cref{stage:3}):
\begin{itemize}
  \item Set $\alpha,\beta,\gamma$ as above.
  \item Follow~\cref{stage:1} to reduce it to the case where $\D$ is unimodal and satisfies the conditions for Kolmogorov and $\lp[1]$ distance; and a good $[a,b]$ approximation of the support is known
  \item For each of the $\bigO{n}$ possible modes $c\in[a,b]$:
  \begin{itemize}
    \item Run the linear program~\cref{algo:lp:logconcave}, return \accept if a feasible solution is found
  \end{itemize}
  \item None of the linear programs was feasible: return \reject.
\end{itemize}

The correctness comes from~\cref{lemma:lp:logconcave:soundness} and~\cref{lemma:lp:logconcave:completeness} and the discussions in~\cref{stage:1} to~\cref{stage:3}; as for the claimed running time, it is immediate from the algorithm and the fact that the linear program executed each step has $\poly(n,1/\eps)$ constraints and variables.

\end{proof}


\section{The Fourier Knife}\label{sec:fourier}
\subsection{Introduction}  \label{sec:introduction:fourier}
\subsubsection{Background and Motivation} \label{ssec:background}

The prototypical inference question in the area of \emph{distribution property testing}~\cite{BFRSW:00}
is the following: Given a set of samples from a collection of probability distributions, can we
determine whether these distributions satisfy a certain property?
During the past two decades, this broad
question -- whose roots lie in statistical hypothesis testing~\cite{NeymannPearson:33,Lehmann:2005:book} --
has received considerable attention by the computer science community,
see~\cite{Rubinfeld:12:Survey, Canonne:15:Survey} for two recent surveys.
After two decades of study, for many properties of interest there exist
sample-optimal testers (matched by information-theoretic lower bounds)
~\cite{Paninski:08, CDVV:14, VV:14, DKN:15, DK:16}.

In this work, we focus on the problem of testing whether the unknown distribution
belongs to a given family of discrete \emph{structured} distributions.
Let $\property$ be a family of discrete distributions over a total order (e.g., $[n]$)
or a partial order (e.g., $[n]^k$). 
The problem of \emph{membership testing for $\property$} is the following:
Given sample access to an unknown distribution $\p$ (effectively supported 
on the same domain as $\property$),
we want to distinguish between the case that $\p \in \property$ versus $\totalvardist{\p}{\property} \geq \eps$. 
The sample complexity of this problem depends on the underlying family
$\property$. For example, if $\cal P$ contains a single distribution over a domain of size $n$,
the sample complexity of the testing problem is $\bigTheta{n^{1/2}/\eps^2}$~\cite{CDVV:14, DKN:15}.

In this work, we give a general technique to test membership in various distribution families over discrete domains.
Before we state our results in full generality, we present concrete applications to 
a number of well-studied distribution families.

\subsubsection{Our Results} \label{ssec:results}

Our first concrete application is a nearly sample-optimal algorithm for testing
sums of independent integer random variables (SIIRVs). 
Formally, an $(n, k)$-SIIRV is a sum of independent integer random variables each supported in $\modulo{k}=\{0,\dots,k-1\}$.
SIIRVs comprise a rich class of distributions that arise in many settings. The special case of $k=2$, $\classksiirv[n]{2}$,
was first considered by Poisson~\cite{Poisson:37} as a non-trivial extension of the Binomial distribution,
and is known as Poisson binomial distribution (PBD). In application domains, SIIRVs have many uses in research areas
such as survey sampling, case-control studies, and survival analysis, see e.g.,~\cite{CL:97} for a survey of the many practical uses of these distributions.
We remark that these distributions are of fundamental interest and have been extensively 
studied in probability and statistics~\cite{Chernoff:52,Hoeffding:63,DP:09, Presman:83,Kruopis:86,BHJ:92, CL10,CGS11}.
We show the following:

\begin{theorem}[Testing SIIRVs]\label{theo:testing:ksiirv}
    Given parameters $k,n\in\N$, $\eps\in(0,1]$, and sample access to a distribution $\p$ over $\N$, there exists an algorithm (\cref{algo:ft:effective:support}) which outputs either \accept or \reject, and satisfies the following:
    \begin{enumerate}
        \item if $\p \in \classksiirv[n]{k}$, then it outputs \accept with probability at least $3/5$;
        \item if $\totalvardist{\p}{\classksiirv[n]{k}}>\eps$, then it outputs \reject with probability at least $3/5$.
    \end{enumerate}
    Moreover, the algorithm takes $\bigO{\frac{k n^{1/4}}{\eps^2}\log^{1/4}\frac{1}{\eps} + \frac{k^2}{\eps^2} \log^2\frac{k}{\eps}}$ samples from $\p$, and runs in time $n(k/\eps)^{\bigO{k\log(k/\eps)}}$.
\end{theorem}

Prior to our work, no non-trivial tester was known for $(n, k)$-SIIRVs for any $k>2$. 
\cite{CDGR:16} showed a sample lower bound of $\bigOmega{\frac{k^{1/2}n^{1/4}}{\eps^2}}$, but their techniques
did not yield a corresponding sample upper bound. The special case of PBDs ($k=2$) was studied by 
Acharya and Daskalakis~\cite{AD:15} who obtained a tester with sample complexity 
$\bigO{\frac{n^{1/4}}{\eps^2}\sqrt{\log1/\eps}+\frac{\log^{5/2}1/\eps}{\eps^6}}$ (and running time $\bigO{\frac{n^{1/4}}{\eps^2}\sqrt{\log1/\eps}+(1/\eps)^{O(\log^2 1/\eps})}$ and a sample lower bound of $\Omega(n^{1/4}/\eps^2)$. 
Our techniques also yield the following corollary: 

\begin{theorem}[Testing PBDs]\label{theo:testing:[bd}
    Given parameters $n\in\N$, $\eps\in(0,1]$, and sample access to a distribution $\p$ over $\N$, there exists an algorithm (\cref{algo:ft:effective:support}) which outputs either \accept or \reject, and satisfies the following.
    \begin{enumerate}
        \item if $\p \in \classpbd[n]$, then it outputs \accept with probability at least $3/5$;
        \item if $\totalvardist{\p}{\classpbd[n]} > \eps$, then it outputs \reject with probability at least $3/5$.
    \end{enumerate}
    Moreover, the algorithm takes $\bigO{\frac{n^{1/4}}{\eps^2}\log^{1/4}\frac{1}{\eps} + \frac{\log^2 1/\eps}{\eps^2}}$ samples from $\p$, and runs in time $n^{1/4}\cdot\tildeO{{1}/{\eps^2}}+(1/\eps)^{\bigO{\log\log(1/\eps)}}$.
\end{theorem}

The sample complexity in the theorem above follows from~\cref{theo:testing:ksiirv}, for $k=2$. 
The improved running time relies on a more efficient computational ``projection step''  in our general framework,
which builds on the geometric structure of Poisson Binomial distributions and allows us to avoid an $(1/\eps)^{\bigO{\log(1/\eps)}}$ dependence. 
In summary, as a special case of~\cref{theo:testing:ksiirv}, we obtain a tester for PBDs 
whose sample complexity is optimal as a function of both $n$ and $1/\eps$ (up to a logarithmic factor).

We further remark that the guarantees provided by the above two theorems 
are actually stronger than the usual property testing one; namely, whenever the algorithm returns \accept, 
then it also provides a (proper) hypothesis $\h$ such that $\totalvardist{\p}{\h}\leq \eps$ with probability at least $3/5$.


An alternate generalization of PBDs to the high-dimensional setting is the family of
Poisson Multinomial Distributions (PMDs).
Formally, an $(n, k)$-PMD is any random variable of the form $X = \sum_{i=1}^n X_i,$
where the $X_i$'s are independent random vectors supported on the set
$\{e_1, e_2, \ldots, e_k \}$ of standard basis vectors in $\R^k$.
PMDs comprise a broad class of discrete distributions of fundamental importance in computer science, probability, and statistics.
A large body of work in the probability and statistics literature has been devoted to the study of the behavior
of PMDs under various structural conditions~\cite{Barbour:88, Loh:92, BHJ:92, Bentkus:03, Roos:99, Roos:10}.
PMDs generalize the familiar multinomial distribution, and describe many distributions commonly encountered in computer science (see, e.g.,~\cite{DP:07:AGT, DP:08:AGT, Valiant:11,VV:11:stoc}).
Recent years have witnessed a flurry of research activity on PMDs and related distributions,
from several perspectives of theoretical computer science,including learning~\cite{DDS:PBD:15, DDOST:13, DKS:15, DKT:15, DKS:15b}, property testing~\cite{Valiant:11, ValiantValiant:10lb, VV:11:stoc}, computational game theory~\cite{DP:07:AGT, DP:08:AGT, BorgsCIKMP08, DP:09:AGT, DP:14:AGT, GT14,CDS:17},
 and derandomization~\cite{GMRZ:11, BDS:12, De:15, GKM:15}. 


\begin{theorem}[Testing PMDs]\label{theo:testing:pmd}
    Given parameters $k,n\in\N$, $\eps\in(0,1]$, and sample access to a distribution $\p$ over $\N$, there exists an algorithm (\cref{algo:pmd:tester}) which outputs either \accept or \reject, and satisfies the following.
    \begin{enumerate}
        \item if $\p \in \classpmd[n]{k}$, then it outputs \accept with probability at least $3/5$;
        \item if $\totalvardist{\p}{\classpmd[n]{k}} > \eps$, then it outputs \reject with probability at least $3/5$.
    \end{enumerate}
    Moreover, the algorithm takes $\bigO{\frac{n^{(k-1)/4} k^{2k} \log(k/\eps)^k}{\eps^2}}$ samples from $\p$, 
    and runs in time $n^{O(k^3)} \cdot (1/\eps)^{O(k^3\frac{\log(k/\eps)}{\log\log(k/\eps)})^{k-1}}$ or alternatively in time $n^{O(k)} \cdot  2^{O(k^{5k} \log(1/\eps)^{k+2})}$.
\end{theorem}
We also show a nearly matching sample lower bound\footnote{Here, we use the notation $\Omega_k(\cdot)$, $O_k(\cdot)$ to indicate that the parameter $k$ is seen as a constant, focusing on the asymptotics with regard to $n,\eps$.}\ of $\Omega_k( n^{{(k-1)}/{4}}/\eps^2 )$ (\cref{theo:lb:pmd}). Finally, we demonstrate the versatility of our techniques by obtaining in~\cref{sec:log:concaves} a testing algorithm for discrete log-concavity with sample complexity $O(\sqrt{n}/\eps^2 + (\log(1/\eps)/\eps)^{5/2})$; improving on the previous bounds of~$\bigO{{\sqrt{n}}/{\eps^{2}}+{1}/{\eps^5}}$~\cite{ADK:15} and $\tildeO{{\sqrt{n}}/{\eps^{7/2}}}$~\cite{CDGR:16}.


\subsubsection{Our Techniques and Comparison to Previous Work} \label{ssec:techniques}

The common property of these distribution families $\property$ that allows for our unified testing approach
is the following: Let $\p$ be the probability mass function of any distribution in $\property$. Then the Fourier transform
of $\p$ is approximately sparse, in a well-defined sense. 

For concreteness and due to space limitations,  we elaborate for the case of SIIRVs.
The starting point of our approach is the observation from~\cite{DKS:15} that $(n,k)$-SIIRVs, 
in addition to having a relatively small effective support, also enjoy an approximately sparse Fourier representation. 
Roughly speaking, most of their Fourier mass is concentrated on a small subset of Fourier coefficients, which can be computed efficiently.

This suggests the following natural approach to testing $(n,k)$-SIIRVs: first, identify the effective support $I$ of the distribution $\p$ 
and check that it is as small as it ought to be. Then, compute the corresponding small subset $S$ of the Fourier domain, 
and check that almost no Fourier mass of $\p$ lies outside $S$ (otherwise, one can safely reject, as this is a certificate that $\p$ is not an $(n,k)$-SIIRV). 
Combining the two, one can show that learning (in $L_2$ norm) the Fourier transform of $\p$ on this small subset $S$ only, 
is sufficient to learn $\p$ itself in total variation distance. The former goal can be performed with relatively few samples, as $S$ is sufficiently small.

Doing so results in a distribution $\h$, represented succinctly by its Fourier transform on $S$, such that $\p$ and $\h$ are close in total variation distance. 
It only remains to perform a computational ``projection step'' to verify that $\h$ itself is close to some $(n,k)$-SIIRV. 
This will clearly be the case if indeed $\p\in\classksiirv[n]{k}$.

We note that although the above idea is at the core of the SIIRV testing algorithm of~\cref{algo:ksiirv:tester}, 
the actual tester has to address separately the case where $\p$ has small variance, which can be handled by a brute-force learning-and-testing approach. 
Our main contribution is thus to describe how to efficiently perform the second step, i.e., the Fourier sparsity testing. 
This is done in~\cref{theo:ft:effective:support}, which describes a simple algorithm to perform this step: essentially, by considering the Fourier coefficients 
of the empirical distribution obtained by taking a small number of samples. Interestingly, the main idea underlying~\cref{theo:ft:effective:support} 
is to avoid analyzing directly the behavior of these Fourier coefficients -- which would naively require too high a time complexity. 
Instead, we rely on Plancherel's identity and reduce the problem to the analysis of a different task:  
that of the sample complexity of $L_2$ identity testing (\cref{prop:l2:identity:tester}). By a tight analysis of this $L_2$ tester, 
we get as a byproduct that several Fourier quantities of interest (of our empirical distribution) simultaneously enjoy 
good concentration -- while arguing concentration of each of these terms separately would yield a suboptimal time complexity. 

A nearly identical method works for PMDs as well. Moreover, our approach can be abstracted 
to yield a general testing framework, as we explain in~\cref{sec:general:testing}. It is interesting to
remark that the Fourier transform has been used to learn PMDs and SIIRVs~\cite{DKS:15, DKT:15, DKS:15b, DDKT:16}, and therefore it may not be entirely surprising that it has applications to testing as well. 
However, testing membership to a class using the Fourier transform is significantly more challenging than learning: a fundamental reason being that, in contrast to the learning 
setting, we need to handle distributions that are \emph{not} SIIRVs and PMDs (but, indeed, are far from those). The learning algorithms, on the other hand, work under the promise that the distribution is in the class, and thus can leverage the specific structure of SIIRVs and PMDs. Moreover, our Fourier testing techniques gives
improved algorithms for other structured families as well, e.g., log-concavity, for which no Fourier learning algorithm was known.

\subparagraph{Learning and testing the Fourier transform: the advantage}

One may wonder how the detour via the Fourier transform enables us to obtain better sample complexity than an approach purely based on $L_2$ testing.  Indeed, all distributions in the classes we consider, crucially, have a small $L_2$ norm: for testing identity to such a distribution $\p$, the standard $L_2$ identity tester (see, e.g.,~\cite{CDVV:14} or~\cref{prop:l2:identity:tester}), which works by checking how large the $L_2$ distance between the empirical and the hypothesis distribution is, will be optimal. We can thus test membership of a class of such distributions by (i) learning $\p$ assuming it belongs to the class, and then (ii) test whether what we learned is indeed close to $\p$ using the $L_2$ identity tester. The catch is to get guarantees in $L_1$ distance out of this, applying Cauchy--Schwarz would require us to learn to very small $L_2$ distance. Namely, if $\p$ has support size $n$, we would have to learn to $L_2$ distance $\frac{\eps}{\sqrt{n}}$ in (i), and then in (ii) test that we are within $L_2$ distance $\frac{\eps}{\sqrt{n}}$ of the learned hypothesis.

However, if a distribution $\p$ has a sparse discrete Fourier transform whose effective support is known, then it is enough to estimate only these few Fourier coefficients~\cite{DKS:15,DKS:15c}. This enables us to learn $\p$ in (i) not just to within $L_1$ distance $\eps$ but indeed crucially within $L_2$ distance $\frac{\eps}{\sqrt{n}}$ with good sample complexity. Additionally, the identity tester algorithm can be put into a simpler form for a hypothesis with sparse Fourier transform, as previously mentioned. Now, the tester has a higher sample complexity, roughly $\sqrt{n}/\eps^2$; but if it passes, then we have learned the distribution $\p$ to within $\eps$ total variation distance, with much fewer samples than the $\bigOmega{n/\eps^2}$ required for arbitrary distributions over support size $n$.

Lastly, we note that instead of $\sqrt{n}/\eps^2$ in the sample complexity above, we can get $n^{1/4}/\eps^2$ for $(n,k)$-SIIRVs by considering the effective support of the distribution.

%%%%%%%%%%%%%%%%%%%%%%%%%%%%%%%%%%%%%%%%%%%%%%%%%%%%%%%%%%%%%%%%%%%%%%%%%%%%%%%%%%%%%%%%%%%%%%%%%%%%%%%%%% 
\subsection{Testing Effective Fourier Support}\label{sec:fourier:support:testing}
In this section, we prove the following theorem, which will be invoked as a crucial ingredient of our testing algorithms. Broadly speaking, the theorem ensures one can efficiently test whether an unknown distribution $\q$ has its Fourier transform concentrated on some (small) effective support $S$ (and if this is the case, learn the vector $\fourier{\q}\indicSet{S}$, the restriction of this Fourier transform to $S$, in $L_2$ distance).

\begin{theorem}\label{theo:ft:effective:support}
    Given parameters $M\geq 1$, $\eps,b\in(0,1]$, as well as a subset $S\subseteq \modulo{M}$ and sample access to a distribution $\q$ over $\modulo{M}$, \cref{algo:ft:effective:support} outputs either \reject or a collection of Fourier coefficients $\fourier{\h'}=(\fourier{\h'}(\xi))_{\xi\in S}$ such that with probability at least $7/10$, all the following statements hold simultaneously.
    \begin{enumerate}
        \item\label{theo:ft:effective:support:i} if $\normtwo{\q}^2 > 2b$, then it outputs \reject;
        \item\label{theo:ft:effective:support:ii} if $\normtwo{\q}^2 \leq 2b$ and every function $\q^\ast\colon\modulo{M}\to\R$ with $\fourier{\q^\ast}$ supported entirely on $S$ is such that $\normtwo{\q-\q^\ast} > \eps$, then it outputs \reject;
        \item\label{theo:ft:effective:support:iii} if $\normtwo{\q}^2 \leq b$ and there exists a function $\q^\ast\colon\modulo{M}\to\R$ with $\fourier{\q^\ast}$ supported entirely on $S$ such that $\normtwo{\q-\q^\ast} \leq \frac{\eps}{2}$, then it does not output \reject;
        \item\label{theo:ft:effective:support:iv} if it does not output \reject, then $\normtwo{\fourier{\q}\indicSet{S}-\fourier{\h'}} \leq \frac{\eps\sqrt{M}}{10}$ and the inverse Fourier transform (modulo $M$) $\h'$ of the Fourier coefficients $\fourier{\h'}$ it outputs satisfies $\normtwo{\q-\h'} \leq \frac{6\eps}{5}$.
    \end{enumerate}
    Moreover, the algorithm takes $m=\bigO{\frac{\sqrt{b}}{\eps^2}+ \frac{\abs{S}}{M\eps^2} +\sqrt{M}}$ samples from $\q$, and runs in time $\bigO{m\abs{S}}$.
\end{theorem}
Note that the rejection condition in~\cref{theo:ft:effective:support:ii} is equivalent to $\normtwo{\fourier{\q}\indicSet{\bar{S}}} > \eps\sqrt{M}$, that is to having Fourier mass more than $\eps^2$ outside of $S$; this is because for any $\q^\ast$ supported on $S$,
\[
    M\normtwo{\q-\q^\ast}^2 = \normtwo{\fourier{\q}-\fourier{\q^\ast}}^2
    = \normtwo{\fourier{\q}\indicSet{S}-\fourier{\q^\ast}\indicSet{S}}^2 + \normtwo{\fourier{\q}\indicSet{\bar{S}}-\fourier{\q^\ast}\indicSet{\bar{S}}}^2
    \geq \normtwo{\fourier{\q}\indicSet{\bar{S}}-\fourier{\q^\ast}\indicSet{\bar{S}}}^2
    = \normtwo{\fourier{\q}\indicSet{\bar{S}}}^2
\]
and the inequality is tight for $\q^\ast$ being the inverse Fourier transform (modulo $M$) of $\fourier{\q}\indicSet{S}$.

\medskip

\noindent \textbf{High-level idea.} 
Let $\q$ be an unknown distribution supported on $M$ consecutive integers (we will later apply this to $\q\eqdef \p \bmod M$), and $S\subseteq\modulo{M}$ be a set of Fourier coefficients (symmetric with regard to $M$: $\xi\in S$ implies $-\xi \bmod M \in S$) such that $0\in S$. We can further assume that we know $b\geq 0$ such that $\normtwo{\q}^2 \leq b$.

Given $\q$, we can consider its ``truncated Fourier expansion'' (with respect to $S$) $\fourier{\h}=\hat{\q}\indicSet{S}$ defined as
\[
    \fourier{\h}(\xi) \eqdef
      \begin{cases}
          \hat{\q}(\xi) & \text{ if } \xi\in S\\
          0 & \text{ otherwise}
      \end{cases}
\]
for $\xi\in\modulo{M}$; and let $\h$ be the inverse Fourier transform (modulo $M$) of $\fourier{\h}$. Note that $\h$ is no longer in general a probability distribution.\medskip

To obtain the guarantees of~\cref{theo:ft:effective:support}, a natural idea is to take some number $m$ of samples from $\q$, and consider the empirical distribution $\q'$ they induce over $\modulo{M}$. By computing the Fourier coefficients (restricted to $S$) of this $\q'$, as well as the Fourier mass ``missed'' when doing so (i.e., the Fourier mass $\normtwo{\fourier{\q'}\indicSet{\bar{S}}}^2$ that $\q'$ puts outside of $S$) to sufficient accuracy, one may hope to prove~\cref{theo:ft:effective:support} with a reasonable bound on $m$.

The issue is that analyzing \emph{separately} the behavior of $\normtwo{\fourier{\q'}\indicSet{\bar{S}}}^2$ and $\normtwo{\fourier{\q'}\indicSet{S}-\fourier{\q'}\indicSet{S}}^2$ to show that they are both estimated sufficiently accurately, and both small enough, is not immediate. Instead, we will get a bound on both at the same time, by arguing concentration in a different manner -- namely, by analyzing a different tester for tolerant identity testing in $L_2$ norm.

In more detail, letting $\h$ be as above, we have by Plancherel that
\[
  \sum_{i\in \modulo{M}} (\q'(i)-\h(i))^2 = \normtwo{\q'-\h}^2 = \frac{1}{M}\normtwo{\fourier{\q'}-\fourier{\h}}^2 = \frac{1}{M}\sum_{\xi=0}^{M-1} \dabs{\fourier{\q'}(\xi)-\fourier{\h}(\xi)}^2
\]
and, expanding the definition of $\fourier{\h}$ and using Plancherel again, this can be rewritten as
\begin{align*}
  M\sum_{i\in \modulo{M}} (\q'(i)-\h(i))^2 &=  \normtwo{\fourier{\q}\indicSet{S}-\fourier{\q'}\indicSet{S}}^2 + \normtwo{\q'}^2 - \normtwo{\fourier{\q'}\indicSet{S}}^2.
\end{align*}
(The full derivation will be given in the proof.) The left-hand side has two non-negative compound terms: the first, $\normtwo{\fourier{\p}\indicSet{S}-\fourier{\q'}\indicSet{S}}^2$, corresponds to the $L_2$ error obtained when learning the Fourier coefficients of $\q$ on $S$. The second, $\normtwo{\q'}^2 - \normtwo{\fourier{\q'}\indicSet{S}}^2 = \normtwo{\fourier{\q'}\indicSet{\bar{S}}}^2$, is the Fourier mass that our empirical $\q'$ puts ``outside of $S$.''

So if the LHS is small (say, order $\eps^2$), then in particular both terms of the RHS will be small as well, effectively giving us bounds on our two quantities in one shot. But this very same LHS is very reminiscent of a known statistic~\cite{CDVV:14} for testing identity of distributions in $L_2$. So, 
one can analyze the number of samples required by analyzing such an $L_2$ tester instead. 
This is what we will do in~\cref{prop:l2:identity:tester}.

\begin{algorithm}
  \begin{algorithmic}[1]
    \Require parameters $M\geq 1$, $b,\eps\in(0,1]$; set $S\subseteq \modulo{M}$; sample access to distribution $\q$ over $\modulo{M}$
    \State\label{algo:ft:step:choosemprime} Set $m\gets \clg{C(\frac{\sqrt{b}}{\eps^2}+ \frac{\abs{S}}{M\eps^2}+ \sqrt{M})}$ \Comment{$C>0$ is an absolute constant}    \State Draw $m'\gets \poisson{m}$; if $m'>2m$, \Return \reject
    \State\label{algo:ft:step:empr} Draw $m'$ samples from $\q$, and let $\q'$ be the corresponding empirical distribution over $\modulo{M}$
    \State\label{algo:ft:step:norm} Compute $\normtwo{\q'}^2$, $\fourier{\q'}(\xi)$ for every $\xi\in S$, and $\normtwo{\fourier{\q'}\indicSet{S}}^2$ \Comment{Takes time $\bigO{m\abs{S}}$}
    \If{ $m'^2\normtwo{\q'}^2 - m' > \frac{3}{2}bm^2$ }\label{algo:ft:step:norm:check} \Return \reject
    \ElsIf{ $\normtwo{\q'}^2 - \frac{1}{M}\normtwo{\fourier{\q'}\indicSet{S}}^2 \geq 3\eps^2\left(\frac{m'}{m}\right)^2+\frac{1}{m'}$ } \Return \reject
    \Else
      \State \Return $\fourier{\h'}=(\fourier{\q'}(\xi))_{\xi\in S}$
    \EndIf
  \end{algorithmic}
  \caption{Testing the Fourier Transform Effective Support}\label{algo:ft:effective:support}
\end{algorithm}

\begin{proof}[Proof of~\cref{theo:ft:effective:support}]
Given $m'\sim\poisson{m}$ samples from $\q$, let $\q'$ be the empirical distribution they define. 
We first observe that with probability $2^{-\Omega(\eps^2m/b)}< \frac{1}{100}$, we have $m' \in [1\pm \frac{\eps}{100\sqrt{b}}]m$ and thus the algorithm does not output \reject in Step~\ref{algo:ft:step:choosemprime} (this follows from standard concentration bounds on Poisson random variables). We will afterwards assume this holds. By Plancherel, we have
\[
  \sum_{i\in \modulo{M}} (\q'(i)-\h(i))^2 = \normtwo{\q'-\h}^2 =  \frac{1}{M}\normtwo{\fourier{\q'}-\fourier{\h}}^2 = \frac{1}{M}\sum_{\xi=0}^{M-1} \dabs{\fourier{\q'}(\xi)-\fourier{\h}(\xi)}^2
\]
and, expanding the definition of $\fourier{\h}$, this yields
\begin{align}
  \sum_{i\in \modulo{M}} (\q'(i)-\h(i))^2 &=  \frac{1}{M}\sum_{\xi\in S} \dabs{\fourier{\q'}(\xi)-\fourier{\h}(\xi)}^2 + \frac{1}{M}\sum_{\xi\notin S} \dabs{\fourier{\q'}(\xi)}^2 \notag\\
  &=  \frac{1}{M}\sum_{\xi\in S} \dabs{\fourier{\q'}(\xi)-\fourier{\q}(\xi)}^2 + \frac{1}{M}\sum_{\xi=0}^{M-1} \dabs{\fourier{\q'}(\xi)}^2 
  - \frac{1}{M}\sum_{\xi\in S} \dabs{\fourier{\q'}(\xi)}^2 \notag\\
  &=  \frac{1}{M}\left(\normtwo{\fourier{\q}\indicSet{S}-\fourier{\q'}\indicSet{S}}^2 + \normtwo{\fourier{\q'}}^2 - \normtwo{\fourier{\q'}\indicSet{S}}^2\right)  \notag\\
  &=  \frac{1}{M}\normtwo{\fourier{\q}\indicSet{S}-\fourier{\q'}\indicSet{S}}^2 + \normtwo{\q'}^2 - \frac{1}{M}\normtwo{\fourier{\q'}\indicSet{S}}^2  \label{eq:plancherel:statistic}
\end{align}
where in the last step we invoked  Plancherel again to argue that $\frac{1}{M}\normtwo{\fourier{\q'}}^2=\normtwo{\q'}^2$.

To analyze the correctness of the algorithm (specifically, the completeness), we will adopt the point of view suggested by~\eqref{eq:plancherel:statistic} and analyze instead the statistic
$\sum_{i\in \modulo{M}} (\q'(i)-\h(i))^2$, when $\h$ is an explicit (pseudo) distribution on $\modulo{M}$ assumed known, and $\q'$ is the empirical distribution obtained by drawing $\poisson{m}$ samples from some unknown distribution $\q$. (Namely, we want to see this as a tolerant $L_2$ identity tester between $\q$ and $\h$.)

\begin{itemize}
  \item We first show that, given that $m'=\bigOmega{\frac{\abs{S}}{M\eps^2}}$, with probability at least $\frac{99}{100}$ we have
    $
        \normtwo{\fourier{\q}\indicSet{S}-\fourier{\h'}} \leq \frac{\sqrt{M}\eps}{10}
    $.
	We note that $m'\fourier{\q'}(\xi)$ is an sum of $m'$ i.i.d. numbers each of absolute value $1$ and mean $\fourier{\q}(\xi)$ (which has absolute value less than $1$). If $X$ is one of these numbers, $|X-\fourier{\q}(\xi)| \leq 2$ with probability $1$ and so the variance of the real and imaginary parts of $X$ is at most $4$. Thus the variance of the real and imaginary  parts of $m'\fourier{\q'}(\xi)$ is at most $4m'$. Then we have
	$\shortexpect[|\fourier{\q}(\xi) - \fourier{\q'}(\xi)|^2]=\shortexpect[ (\Re (\fourier{\q}(\xi) - \fourier{\q'}(\xi)))^2 + (\Im ( \fourier{\q}(\xi) - \fourier{\q'}(\xi)))^2] \leq 8/m'$. Summing over $S$, using that $\q'$ and $\h'$ have the same Fourier coefficients there, yields
	
\[
    \expect{ \sum_{\xi\in S} \abs{\fourier{\q}(\xi)-\fourier{\h'}(\xi)}^2 } \leq \frac{8|S|}{m'} \leq \frac{M\eps^2}{10000}
\]
and by Markov's inequality we get 
$\probaOf{ \normtwo{\fourier{\q}\indicSet{S}-\fourier{\h'}}^2 \leq \frac{M\eps^2}{100} } = \probaOf{ \sum_{\xi\in S} \abs{\fourier{\q}(\xi)-\fourier{\h'}(\xi)}^2 \leq \frac{M\eps^2}{100} } \geq \frac{1}{100}$,
 concluding the proof.


  \item Then, let us consider~\cref{theo:ft:effective:support:i}: assume $\normtwo{\q}^2 > 2b$, and set $X\eqdef m'^2\normtwo{\q'}^2 - m'$. Then, 
\[
  \shortexpect[ X ] = \sum_{i=1}^M \shortexpect[ m'^2\q'(i)^2 ] - \sum_{i=1}^M \shortexpect[ m'\q'(i) ] = \sum_{i=1}^M ( m\q(i)+m^2\q(i)^2) - \sum_{i=1}^M m\q(i)= m^2\normtwo{\q}^2
\]
since the $m'\q'(i)$ are distributed as $\poisson{m\q(i)}$. As all $m'\q'(i)$'s are independent by Poissonization, we also have
\[
  \var[ X ] = \sum_{i=1}^M \var[ m'^2\q'(i)^2 - m'\q'(i) ] = \sum_{i=1}^M ( 2m^2\q(i)^2 + 4m^3\q(i)^3) 
  = 2m^2 \normtwo{\q}^2 + 4m^3 \norm{\q}_3^3
\]
and by Chebyshev,
\[
    \Pr[ X \leq \frac{3}{2}m^2b] \leq \Pr\left[ \abs{ X - \shortexpect[ X ] } > \frac{1}{4} \shortexpect[ X ] \right]
    \leq 16\frac{\var[ X ]}{\shortexpect[ X ]^2}
    \leq \frac{32}{m^2\normtwo{\q}^2} + \frac{64 \norm{\q}_3^3}{m \normtwo{\q}^4 }
\]
Since $\q$ is supported on $\modulo{M}$, $\normtwo{\q}^2 \geq \frac{1}{M}$ and the first term is at most $\frac{32M}{m^2}$. The second term, by monotonicity of $\ell_p$-norms, is at most 
$\frac{64 \normtwo{\q}^3}{m \normtwo{\q}^4 } = \frac{48}{m \normtwo{\q} } \leq \frac{48\sqrt{M}}{m}$. The RHS is then at most $\frac{1}{100}$ for a large enough choice of $C>0$ in the definition of $m$. Thus, with probability at least $1-\frac{1}{100}$ we have $m'^2\normtwo{\q'}^2 - m' > \frac{3}{2}b$, and the algorithm outputs \reject in Step~\ref{algo:ft:step:norm:check}.

Moreover, if $\normtwo{\q}^2 \leq b$, then the same analysis shows that
\[
    \Pr[ X > \frac{3}{2}m^2b ] \leq \Pr\left[ \abs{ X - \shortexpect[ X ] } > \frac{1}{2}\shortexpect[ X ] \right]
    \leq 4\frac{\var[ X ]}{\shortexpect[ X ]^2} \leq \frac{1}{100}
\]
and with probability at least $1-\frac{1}{100}$ the algorithm does not output \reject in Step~\ref{algo:ft:step:norm}.

  \item Turning now to~\cref{theo:ft:effective:support:ii,theo:ft:effective:support:iii,theo:ft:effective:support:iv}: we assume that the algorithm does not output \reject in Step~\ref{algo:ft:step:norm} (which by the above happens with probability $99/100$ if $\normtwo{\q}^2 \leq b$; and can be assumed without loss of generality otherwise, since we then want to argue that the algorithm \emph{does} reject at some point in that case).
  
    By the remark following the statement of the theorem, it is sufficient to show that the algorithm outputs \reject (with high probability) if $\normtwo{\fourier{\q}\indicSet{\bar{S}}}^2 > \eps^2 M$, and that if both $\normtwo{\q}^2 \leq b$ and $\normtwo{\fourier{\q}\indicSet{\bar{S}}}^2 \leq \frac{\eps^2}{4}M$ then it does not output~\reject; and that whenever the algorithm does not output \reject, then $\normtwo{\fourier{\q}-\fourier{\h}} \leq \eps^2 M$.
    
    Observe that calling~\cref{algo:tol:l2:identity:tester} with our $m'=\poisson{m}$ samples from $\q$ (distribution over $\modulo{M}$), parameters $\frac{\eps}{2}$ and $2b$, and the explicit description of the pseudo distribution $\p^\ast\eqdef \frac{m'}{m}\h$ (which one would obtain for $\h$ being the inverse Fourier transform of $\fourier{\q}\indicSet{S}$) would result by~\cref{prop:l2:identity:tester} (since $m \geq c\frac{\sqrt{2b}}{(\eps/2)^2} = 244\sqrt{2}\frac{\sqrt{b}}{\eps^2}$, where $c$ is as in~\cref{prop:l2:identity:tester}) in having the following guarantees on $\frac{\sqrt{Z}}{m}$, where $Z$ is the statistic defined in~\cref{algo:tol:l2:identity:tester}
    \begin{itemize}
      \item if $\normtwo{\q-\p^\ast} \leq \frac{\eps}{2}$, then $\frac{\sqrt{Z}}{m} \leq \sqrt{2.9}\eps$ with probability at least $3/4$;
      \item if $\normtwo{\q-\p^\ast} \geq \eps$, then $\frac{\sqrt{Z}}{m} \geq \sqrt{3.1}\eps$ with probability at least $3/4$;
    \end{itemize}
    as $\normtwo{\q}^2 \leq 2b$ (note that then $\normtwo{\h}^2 \leq b$ as well). Since $\sqrt{M}\normtwo{\q-\p^\ast} = \normtwo{\fourier{\q}-\fourier{\p^\ast}} = \normtwo{\fourier{\q}-\frac{m}{m'}\fourier{\q}\indicSet{S}}$ and     \[
      \frac{Z}{m'^2} = \sum_{i=1}^M \left( (\q'(i)-\frac{m}{m'}\p^\ast(i))^2 - \frac{\q'(i)}{m'} \right) = \sum_{i=1}^M (\q'(i)-\h(i))^2 - \frac{1}{m'}
    \]
    which is equal to $\frac{1}{M}\normtwo{\fourier{\q}\indicSet{S}-\fourier{\q'}\indicSet{S}}^2 + \normtwo{\q'}^2 - \frac{1}{M}\normtwo{\fourier{\q'}\indicSet{S}}^2 - \frac{1}{m'}$ by~\cref{eq:plancherel:statistic}, we thus get the following.
    \begin{itemize}
      \item if $\normtwo{\fourier{\q}\indicSet{\bar{S}}}^2 \leq \frac{\eps^2M}{9}$, then $\normtwo{\fourier{\q}-\fourier{\q}\indicSet{S}} \leq \frac{\eps}{3}\sqrt{M}$, and
      \[
          \sqrt{M}\normtwo{\p^\ast-\q} = \normtwo{\fourier{\p^\ast}-\fourier{\q}} \leq \normtwo{\fourier{\p^\ast}-\fourier{\q}\indicSet{S}} + \normtwo{\fourier{\q}\indicSet{S}-\fourier{\q}}
          = \abs{\frac{m}{m'}-1}\normtwo{\fourier{\q}\indicSet{S}} + \normtwo{\fourier{\q}-\fourier{\q}\indicSet{S}}
      \]
      Since we have $m'\in[1\pm\frac{\eps}{100\sqrt{b}}]m$ by the above discussion and $\normtwo{\fourier{\q}\indicSet{S}}\leq \sqrt{2b}\sqrt{M}$, the RHS is upper bounded by $\frac{\eps}{6}\sqrt{M} + \frac{\eps}{3}\sqrt{M} = \frac{\eps}{2}\sqrt{M}$, and $\normtwo{\p^\ast-\q} \leq \frac{\eps}{2}$.
      Then $\frac{1}{M}\normtwo{\fourier{\q}\indicSet{S}-\fourier{\q'}\indicSet{S}}^2 + \normtwo{\q'}^2 - \frac{1}{M}\normtwo{\fourier{\q'}\indicSet{S}}^2 = \frac{Z}{m'^2}+\frac{1}{m'} \leq 2.9\eps^2\left(\frac{m'}{m}\right)^2+\frac{1}{m'}$ with probability at least $3/4$, and in particular 
      $\normtwo{\q'}^2 - \frac{1}{M}\normtwo{\fourier{\q'}\indicSet{S}}^2 \leq 2.9\eps^2\left(\frac{m'}{m}\right)^2+\frac{1}{m'} < 3\eps^2\left(\frac{m'}{m}\right)^2+\frac{1}{m'}$;
      \item if $\normtwo{\fourier{\q}\indicSet{\bar{S}}}^2 > \eps^2M$,
           then $\frac{1}{M}\normtwo{\fourier{\q}\indicSet{S}-\fourier{\q'}\indicSet{S}}^2 + \normtwo{\q'}^2 - \frac{1}{M}\normtwo{\fourier{\q'}\indicSet{S}}^2 =\frac{Z}{m'^2}+\frac{1}{m'} > 3.1\eps^2\left(\frac{m'}{m}\right)^2+\frac{1}{m'}$ with probability at least $3/4$; since by the first part we established we have $\normtwo{\fourier{\q}\indicSet{S}-\fourier{\q'}\indicSet{S}}^2\leq \frac{\eps^2M}{100}$, this implies $\normtwo{\q'}^2 - \frac{1}{M}\normtwo{\fourier{\q'}\indicSet{S}}^2 > 3.1\eps^2\left(\frac{m'}{m}\right)^2+\frac{1}{m'} - \frac{\eps^2}{100}>  3\eps^2\left(\frac{m'}{m}\right)^2+\frac{1}{m'}$.
    \end{itemize}
    
    This immediately takes care of~\cref{theo:ft:effective:support:ii,theo:ft:effective:support:iii}; moreover, this implies that whenever \cref{algo:ft:effective:support} does \emph{not} output \reject, then the inverse Fourier transform $\h'$ of the collection of Fourier coefficients it returns (which are supported on $S$) satisfies
    \begin{align*}
       \normtwo{\q-\h'}^2 &= \frac{1}{M}\normtwo{\fourier{\q}-\fourier{\h'}}^2 = \frac{1}{M}\normtwo{\fourier{\q}\indicSet{S}-\fourier{\h'}}^2 + \frac{1}{M}\normtwo{\fourier{\q}\indicSet{\bar{S}}}^2 \\
        &\leq \frac{\eps^2}{100} + \frac{1}{M}\normtwo{\fourier{\q}\indicSet{\bar{S}}}^2 \\
        &\leq \frac{\eps^2}{100} + \eps^2 = \frac{101}{100}\eps^2
    \end{align*}
    and thus
    $
        \normtwo{\q-\h'} \leq  \sqrt{\frac{101}{100}}\eps < \frac{6}{5}\eps
    $
    which establishes~\cref{theo:ft:effective:support:iv}. Finally, by a union bound, all the above holds except with probability $\frac{1}{100}+\frac{1}{100}+\frac{1}{100}+\frac{1}{4}<\frac{3}{10}$. This concludes the proof.

\end{itemize}

\end{proof}



\subsubsection{A tolerant $L_2$ tester for identity to a pseudodistribution}

As previously mentioned, one building block in the proof of~\cref{theo:ft:effective:support} (and a result that may be of independent interest) is an optimal $L_2$ identity testing algorithm. Our tester and its analysis are very similar to the tolerant $L_2$ closeness testing algorithm of Chan et al.~\cite{CDVV:14}, with the obvious simplifications pertaining to identity (instead of closeness). The main difference is that we emphasize here the fact that $\p^\ast$ need not be an actual distribution: any $\p^\ast\colon[r]\to\R$ would do, even taking negative values. This will turn out to be crucial for our applications.

\begin{algorithm}
  \begin{algorithmic}
  \Require $\eps\in(0,1)$, $m$ samples from distributions $\p$ over $[r]$, with $X_i$ denoting the number of occurrences of the $i$-th domain elements in the samples from $\p$, and $\p^\ast$ being a fixed, known pseudo distribution over $[r]$.
  \Ensure Returns \textsf{accept} if $\normtwo{\p - \p^\ast } \leq \eps$ and \textsf{reject} if $\normtwo{\p - \p^\ast }  \geq 2\eps$.
    \State Define $Z=\sum_{i=1}^r (X_i-m\p^\ast(i))^2-X_i.$ \Comment{Can actually be computed in $O(m)$ time}
    \State Return  \textsf{reject} if $\frac{\sqrt{Z}}{m} > \sqrt{3}\eps$, \textsf{accept} otherwise.
  \end{algorithmic}
  \caption{Tolerant $L_2$ identity tester}\label{algo:tol:l2:identity:tester}
\end{algorithm}

\begin{proposition}\label{prop:l2:identity:tester}
There exists an absolute constant $c > 0$ such that the above algorithm (\cref{algo:tol:l2:identity:tester}), when given $\poisson{m}$ samples drawn from a distribution $\p$ and an explicit function $\p^\ast\colon[r]\to\R$ will, with probability at least $3/4$, 
distinguishes between \textsf{(a)} $\normtwo{ \p-\p^\ast } \leq \eps$ and \textsf{(b)} $\normtwo{ \p-\p^\ast } \geq 2\eps$ provided that $m \geq c\frac{\sqrt{b}}{\eps^2}$, where $b$ is an upper bound on $\normtwo{ \p }^2, \normtwo{ \p^\ast }^2$. (Moreover, one can take $c = 61$.)

Moreover, we have the following stronger statement: in case (a), the statistic $Z$ computed in the algorithm satisfies $\frac{\sqrt{Z}}{m} \leq \sqrt{2.9}\eps$ with probability at least $3/4$, while in case (b) we have $\frac{\sqrt{Z}}{m} \geq \sqrt{3.1}\eps$ with probability at least $3/4$.
\end{proposition}
\begin{proof}Letting $X_i$ denote the number of occurrences of the $i$-th domain element in the samples from $\p$, define $Z_i=(X_i-m\p^\ast(i))^2-X_i$. Since $X_i$ is distributed as $\poisson{m \p(i)}$,  $\shortexpect[Z_i] = m^2(\p(i) - \p^\ast(i))^2$; thus, $Z$ is an unbiased estimator for $m^2\normtwo{ \p-\p^\ast }^2$. (Note that this holds even when $\p^\ast$ is allowed to take negative values.)

We compute the variance of $Z_i$ via a straightforward calculation involving standard expressions for the moments of a Poisson distribution: getting
\[
  \var[Z]  = \sum_{i=1}^r \var[Z_i] = \sum_{i=1}^r \left(4m^3(\p(i)-\p^\ast(i))^2 \p(i)  + 2m^2\p(i)^2\right).
\]
By Cauchy--Schwarz, and since $\sum_{i=1}^r \p(i)^2 \leq b$ by assumption, we have 
\begin{align*}
\sum_{i=1}^r (\p(i)-\p^\ast(i))^2 \p(i) &= \sum_{i=1}^r (\p(i)-\p^\ast(i))\cdot (\p(i)-\p^\ast(i)) \p(i)  \\
&\leq \sqrt{\sum_{i=1}^r (\p(i)-\p^\ast(i)) ^2 \sum_{i=1}^r \p(i)^2(\p(i)-\p^\ast(i))^2} \\
&\leq \sqrt{\sum_{i=1}^r (\p(i)-\p^\ast(i)) ^2 b \sum_{i=1}^r (\p(i)-\p^\ast(i)) ^2}
 = \sqrt{b} \normtwo{ \p-\p^\ast }^2
\end{align*}
and so 
 \[
 \var[Z] \leq 4 m^3 \sqrt{b} \normtwo{ \p-\p^\ast }^2 + 2 m^2 b.
 \]
 
\noindent For convenience, let $\eta\eqdef\frac{1}{10}$, and write $\rho \eqdef \frac{\normtwo{ \p-\p^\ast }}{\eps}$ -- so that we need to distinguish $\rho \leq 1$ from $\rho \geq 2$. If $\rho \leq 1$, i.e. $\shortexpect[Z] \leq m^2\eps^2$, then 
 \[
    \Pr[ Z > (3-\eta)m^2\eps^2 ] = \Pr[ \lvert Z - \shortexpect[Z] \rvert > m^2\eps^2( ((3-\eta)-\gamma) - \rho^2 ) ]
 \]
 while if $\rho \geq 2$, i.e. $\shortexpect[Z] \geq 4m^2\eps^2$, then
 \[
    \Pr[ Z < (3+\eta)m^2\eps^2 ] = \Pr[ \shortexpect[Z] - Z > m^2( \lVert p-q\rVert_2^2 - (3+\eta)\eps^2 ) ] \leq \Pr[ \lvert Z - \shortexpect[Z] \rvert > m^2\eps^2( \rho^2 - (3+\eta) ) ].
 \]
In both cases, by Chebyshev's inequality, the test will be correct with probability at least~$3/4$ provided $m \geq c\sqrt{b}/\eps^2$ for some suitable choice of $c > 0$, since (where
\begin{align*}
    \Pr[ \lvert Z - \shortexpect[Z] \rvert > m^2\eps^2\lvert \rho^2 - (3\pm\eta) \rvert ] 
    &\leq \frac{\var[Z]}{m^4\eps^4( \rho^2 - (3\pm\eta) )^2} \\
    &\leq \frac{ 4m^3 \sqrt{b} \rho^2\eps^2 + 2 m^2 b }{m^4\eps^4( \rho^2 - (3\pm\eta) )^2} 
    = \frac{\rho^2}{(\rho^2 - (3\pm\eta) )^2}\cdot\frac{ 4\sqrt{b}}{m\eps^2} + \frac{1}{(\rho^2 - (3\pm\eta) )^2}\cdot\frac{2 b }{m^2\eps^4} \\
    &\leq \frac{ 20\sqrt{b}}{m\eps^2} + \frac{5 b }{2m^2\eps^4}  \leq \frac{20}{c}+\frac{5}{2c^2} \leq \frac{1}{3}
\end{align*}
as $\max_{\rho\in[0,1]} \frac{\rho^2}{(\rho^2 - (3\pm\eta) )^2} \leq 5$ and $\max_{\rho\in[0,1]} \frac{1}{(\rho^2 - (3\pm\eta) )^2} \leq \frac{5}{4}$ and the last inequality holds for $c \geq 61$.
\end{proof}

 
\subsection{The Projection Subroutine}\label{sec:projections}
\subsubsection{The projection step for $(n,k)$-SIIRVs}

We can use the proper $\eps$-cover given in~\cite{DKS:15} to find a $(n,k)$-SIIRV near $\p$ by looking at $\fourier{\h}$.

\begin{algorithm}[ht]
  \begin{algorithmic}[1]
    \Require Parameters $n$,$\eps$; the approximate Fourier coefficients $(\fourier{\h}(\xi))_{\xi\in S}$ modulo $M$, of a distribution $\p$ known to be effectively supported on $I$ and to have a Fourier transform effectively supported on $S$ of the form given in Step~\ref{algo:step:dft:computation} of~\cref{algo:ksiirv:tester}, with $\widetilde{\sigma}^2$
and $\widetilde{\mu}$, an approximation to $\shortexpect_{X \sim \p}[X]$ to within half a standard deviation.
	\State \label{step:cover:ksiirv} Compute $\mathcal{C}$, an $\frac{\eps}{5\sqrt{|S|}}$-cover in total variation distance of all $(n,k)$-SIIRVs.
	\For{each $\q \in \mathcal{C}$}
		\If{ the mean $\mu_\q$ and variance  $\sigma_\q$ of $\q$ satisfy $|\widetilde{\mu}-\mu_\q| \leq \widetilde{\sigma}$ and $2(\sigma_\q+1) \geq \widetilde{\sigma}+1 \geq (\sigma_\q+1)/2$} \label{step:cover:moment-test}
			\State Compute $\fourier{\q}(\xi)$ for $\xi \in S$.
			\If{ $\sum_{\xi \in S} |\fourier{\h}-\fourier{\q}|^2 \leq \frac{\eps^2}{5}$}
				\Return \accept
			\EndIf
		\EndIf
	\EndFor
	\State \Return \reject \Comment{we did not return \accept for any $\q \in \mathcal{C}$}
  \end{algorithmic}
  \caption{Algorithm \texttt{Project-k-SIIRV}}\label{algo:ksiirv-easy-projection}
\end{algorithm}

\begin{lemma} If Algorithm \texttt{Project-k-SIIRV} is given inputs that satisfy its assumptions and we have that $\sum_{\xi \in S} |\fourier{\h}-\fourier{\p}|^2 \leq (3\eps/25)^2$, $\totalvardist{\p}{\h} \leq 6\eps/25$, and that if $\p\in\classksiirv[n]{k}$ then $\widetilde{\sigma}^2$ is a factor-$1.5$ approximation to $\var_{X \sim \p}[X]+1$, then it distinguishes between (i) $\p\in\classksiirv[n]{k}$ and (ii) $ \totalvardist{\p}{\classksiirv[n]{k}}>\eps$. The algorithm runs in time $n\left( k/\eps \right)^{O(k\log(k/\epsilon))}$.\end{lemma}
\begin{proof}
By Theorem 3.7 of~\cite{DKS:15}, there is an algorithm that can compute an $\eps$-cover of all $(n,k)$-SIIRVs of size $n\left( k/\eps \right)^{O(k\log(1/\epsilon))}$ that runs in time $n\left( k/\eps \right)^{O(k\log(1/\epsilon))}$. Note the way the cover is given, allows us to compute the Fourier coefficients $\fourier{\q}(\xi)$ for any $\xi$ for each $\q \in \mathcal{C}$ in time $\poly(k/\eps)$.

 Since $\eps/\sqrt{|S|}=1/\poly(k/\eps)$, Step~\ref{step:cover:ksiirv} takes time $n\left( k/\eps \right)^{O(k\log(k/\epsilon))}$ and outputs a cover of size $n\left( k/\eps \right)^{O(k\log(k/\epsilon))}$. As each iteration takes time $|S|$, the whole algorithm takes $n\left( k/\eps \right)^{O(k\log(k/\epsilon))}$ time.

Note that each $\q$ that passes Step~\ref{step:cover:moment-test} is effectively supported on $I$ by (\ref{eq:bennett:application}) and has Fourier transform supported on $S$ by~\cref{claim:ksiirv:fourier:concentrated}.

\begin{itemize}
  \item Suppose that $\p\in\classksiirv[n]{k}$. Then there is a $(n,k)$-SIIRV $\q \in \mathcal{C}$ with $ \totalvardist{\p}{\q} \leq {\eps}/{5\sqrt{|S|}}$.
We need to show that if the algorithm considers $\q$, it accepts. From standard concentration bounds, one gets that the expectations of $\p$ and $\q$ are within $O(\eps\sqrt{\log(1/\eps)})$ standard deviations of $\p$ and the variances of $\p$ and $\q$ are within $O(\eps\log(1/\eps))$ multiplicative error.
Thus $\q$ passes the condition of Step \ref{step:cover:moment-test}. Since $ \totalvardist{\p}{\q} \leq \eps/(5\sqrt{|S|})$, we have that $|\fourier{\p}(\xi)-\fourier{\q}(\xi)| \leq \eps/(5\sqrt{|S|})$ for all $\xi$.
In particular, we have $\sum_{\xi \in S} |\fourier{\h}-\fourier{\q}|^2 \leq \eps^2/25$. Thus by the triangle inequality for $L_2$ norm, we have  $\sum_{\xi \in S} |\fourier{\h}-\fourier{\q}|^2 \leq (\eps/5 + 3\eps/25)^2 \leq (\eps/\sqrt{5})^2$. Thus the algorithm accepts.

  \item Now suppose that the algorithm accepts. We need to show that $\p$ has total variation distance at most $\eps$ from some $(n,k)$-SIIRV. We will show that $ \totalvardist{\p}{\q} \leq \eps$ for the $\q$ which causes the algorithm to accept.  Since the algorithm accepts, $\sum_{\xi \in S} |\fourier{\h}-\fourier{\q}|^2 \leq \eps^2/25$. 
For $x \notin S$, $\fourier{\h}(\xi)=0$ and so $\sum_{\xi \notin S} |\fourier{\h}-\fourier{\q}|^2 = \sum_{\xi \notin S} |\fourier{\q}|^2 \leq \eps^2/100$ by~\cref{claim:ksiirv:fourier:concentrated}. By Plancherel, the distributions $\q'\eqdef \q \bmod M$, $\h'\eqdef \h \bmod M$ satisfy 
\[
  \normtwo{ \q'-\h'}^2 = \frac{1}{M}\sum_{\xi=0}^{M-1} \dabs{ \fourier{\h}-\fourier{\q} }^2 \leq \frac{\eps^2}{20M}.
\] Thus $\totalvardist{\q'}{\h'} \leq \frac{\eps}{4}$. By definition $\h$ has probability $0$ outside $I$ and by (\ref{eq:bennett:application}), $\q$ has at most $\frac{\eps}{5}$ probability outside $I$, Thus $\totalvardist{\q}{\h} \leq \frac{\eps}{4}+\frac{\eps}{5} \leq \frac{\eps}{2}$ and by the triangle inequality $ \totalvardist{\p}{\q} \leq \totalvardist{\q}{\h} +  \totalvardist{\p}{\h} \leq \eps/2 + 6\eps/25 \leq \eps$ as required.
\end{itemize}
\end{proof}

\subsubsection{The case $k=2$}
For the important case of Poisson Binomial distributions, that is $(n,2)$-SIIRVs, we can dispense with using a cover at all.~\cite{DKS:15b} gives an algorithm that can properly learn Poisson binomial distributions in time $(1/\eps)^{O(\log \log 1/\eps)}$. The algorithm works by first learning the Fourier coefficients in $S$, which we have already computed here, and checks if one of many systems of polynomial inequalities has a solution: if the Fourier coefficients are close to those of a $(n,2)$-SIIRV, then there will be a solution to one of these systems. This allows us to test whether or not we are close to a $(n,2)$-SIIRV.

More precisely, we can handle this in two cases: the first, when the variance $s^2$ of $\p$ is relatively small, corresponding to $\widetilde{\sigma} \leq \alpha/\eps^2$ (for some absolute constant $\alpha >0$).

\begin{lemma}
    Let $\p$ be a distribution with variance $O(1/\eps^2)$. Let $\widetilde \mu$ and $\widetilde{\sigma}^2$ be approximations to the mean $\mu$ and variance $s^2$ of $\p$ with $|\widetilde{\mu}-\mu| \leq \widetilde{\sigma}$ and $2(\sigma+1) \geq \widetilde{\sigma}+1 \geq (\sigma+1)/2$. 
Suppose that $\p$ is effectively supported on an interval $I$ and that its DFT modulo $M$ is effectively supported on $S$, the set of integers  $\xi \leq \ell\eqdef O(\log(1/\eps))$.
Let  $\fourier{\h}(\xi)$ be approximations to $\fourier{\p}(\xi)$ for all $\xi \in S$ with $\sum_{\xi \in S} |\fourier{\h}(\xi) - \fourier{\p}(\xi)|^2 \leq \frac{\eps^2}{16}$ .
 There is an algorithm that, given $n$,$\eps$,$\widetilde{\mu}$, $\widetilde{\sigma}$ and $\fourier{\h}(\xi)$, distinguishes between (i) $\p\in\classpbd[n]$ and (ii) $ \totalvardist{\p}{\classpbd[n]}>\eps$, in time at most $(1/\eps)^{O(\log \log 1/\eps)}$.
 \end{lemma}
 \begin{proof}
We use Steps 4 and 5 of Algorithm \texttt{Proper-Learn-PBD} in~\cite{DKS:15b}. Step 5 checks if one of a system of polynomials has a solution. If such a solution is found, it corresponds to an $(n,2)$-SIIRV $\q$ that has $\sum_{|\xi| \leq \ell} |\fourier{\h}(\xi) - \fourier{\q}(\xi)|^2 \leq \eps^2/4$ and so we accept. If no systems have a solution, then there is no such $(n,2)$-SIIRV and so we reject. The conditions of this lemma are enough to satisfy the conditions of Theorem 11 of~\cite{DKS:15b}, though we need that the constant $C'$ used to define $|S|$ is sufficiently large to cover the $\ell=O(\log(1/\eps)$ from that paper. This theorem means that if $\p$ is a $(n,2)$-SIIRV, then we accept.

We need to show that if the algorithm finds a solution, then it is within $\eps$ of a Poisson Binomial distribution.  The system of equations ensures that $\sum_{|\xi| \leq \ell} |\fourier{\h}(\xi) - \fourier{\q}(\xi)|^2 \leq \eps^2/4$. Now the argument is similar to that for $(n,k)$-SIIRVs.
For $x \notin S$, $\fourier{\h}(\xi)=0$ and so $\sum_{\xi \notin S} |\fourier{\h}-\fourier{\q}|^2 = \sum_{\xi \notin S} |\fourier{\q}|^2 \leq \eps^2/100$ by~\cref{claim:ksiirv:fourier:concentrated}. By Plancherel, the distributions $\q'\eqdef \q \bmod M$, $\h'\eqdef \h \bmod M$ satisfy 
\[
  \normtwo{ \q'-\h'}^2 = \frac{1}{M}\sum_{\xi=0}^{M-1} \dabs{ \fourier{\h}-\fourier{\q} }^2 \leq \frac{\eps^2}{20M}.
\] Thus $\totalvardist{\q'}{\h'} \leq \frac{\eps}{4}$. By definition $\h$ has probability $0$ outside $I$ and by (\ref{eq:bennett:application}), $\q$ has at most $\frac{\eps}{5}$ probability outside $I$, Thus $\totalvardist{\q}{\h} \leq \frac{\eps}{4}+\frac{\eps}{5} \leq \frac{\eps}{2}$ and by the triangle inequality $ \totalvardist{\p}{\q} \leq \totalvardist{\q}{\h} +  \totalvardist{\p}{\h} \leq \eps/2 + 6\eps/25 \leq \eps$ as required.
 \end{proof}
 
 If $\widetilde{\sigma} \geq \alpha/\eps^2$ (corresponding to a ``big variance'' $s^2 = \Omega(1/\eps^2)$), then we take an additional $O(|S|/\eps^2)$ samples from $\p$ and use them to learn a shifted binomial using algorithms \texttt{Learn-Poisson} and \texttt{Locate-Binomial} from~\cite{DDS:PBD:15} that is within $O(\eps/\sqrt{|S|})$ total variation distance from $\p$. If these succeed,  we can check if its Fourier coefficients are close using the method in~\cref{algo:ksiirv-easy-projection} (\texttt{Project-k-SIIRV}). As we can compute the Fourier coefficients of a shifted binomial easily, this overall takes time $\poly(1/\eps)$.
 

 
\subsection{The SIIRV Tester}\label{sec:siirv:testing}
We are now ready to describe the algorithm behind~\cref{theo:testing:ksiirv}, and establish the theorem.
\begin{algorithm}[ht]
  \begin{algorithmic}[1]
    \Require sample access to a distribution $\p\in\distribs{\N}$, parameters $n,k\geq 1$ and $\eps\in(0,1]$
    
    \State\Comment{ Let $C,C',C''$ be sufficiently large universal constants }
    
    \State\label{algo:step:estimates:mu:sigma} Draw $O(k)$ samples from $\p$ and compute as in~\cref{claim:estimate:moments}: (a) $\widetilde{\sigma}^2$, a tentative factor-$2$ approximation to $\var_{X \sim \p}[X]+1$,
and (b) $\widetilde{\mu}$, a tentative approximation to $\shortexpect_{X \sim \p}[X]$ to within one standard deviation.
    \If{If $\widetilde{\sigma} > 2k\sqrt{n}$} \label{algo:step:check:stdev}
        \State \Return \reject    \Comment{Blatant violation of $(n,k)$-SIIRV-iness}
    \EndIf
    
    \If{ $\widetilde{\sigma}  \leq 2k\sqrt{\ln\frac{10}{\eps}}$ }\label{algo:step:small:variance}
        \State\label{algo:step:def:interval:smallvariance} Set $M \gets 1+2\clg{15k \ln\frac{10}{\eps}}$, and let $I \gets [\flr{ \widetilde{\mu} }-\frac{M-1}{2},\flr{ \widetilde{\mu} }+\frac{M-1}{2}]$; and $S\gets \modulo{M}$
        \State\label{algo:step:effectivesupport:smallvariance} Draw $O(1/\eps)$ samples from $\p$, to distinguish between $\p(I) \leq 1-\frac{\eps}{4}$ and $\p(I) > 1-\frac{\eps}{5}$.
        If the former is detected, \Return \reject
        \State\label{algo:step:empirical:smallvariance} Take $N=C\left( \frac{\abs{S}}{\eps^2} \right) = \bigO{ \frac{k}{\eps^2} \log\frac{1}{\eps} }$ samples from $\p$ to get an empirical distribution $\h$
    \Else\label{algo:step:big:variance}
        \State\label{algo:step:def:interval} Set $M \gets 1+2\clg{ 4\widetilde{\sigma}\sqrt{\ln(4/\eps)} }$, and let $I \gets [\flr{ \widetilde{\mu} }-\frac{M-1}{2},\flr{ \widetilde{\mu} }+\frac{M-1}{2}]$
        \State\label{algo:step:effectivesupport} Draw $O(1/\eps)$ samples from $\p$, to distinguish between $\p(I) \leq 1-\frac{\eps}{4}$ and $\p(I) > 1-\frac{\eps}{5}$.
        If the former is detected, \Return \reject
        \State\label{algo:step:dft:computation} Let $\delta\gets \frac{\eps}{C''\sqrt{k\log\frac{k}{\eps}}}$, and 
            \[
              S \gets \setOfSuchThat{ \xi \in [M-1] }{ \exists a, b \in \Z, 0 \leq a \leq b < k \textrm{ s.t. } |\xi/M - a/b| \leq  C'\frac{\sqrt{\ln(1/\delta)}}{4\widetilde{\sigma}} } \;.
            \]
        \State\label{algo:step:fourier:support} Simulating sample access to $\p'\eqdef \p\bmod M$, call~\cref{algo:ft:effective:support} on $\p'$ with parameters $M$, $\frac{\eps}{5\sqrt{M}}$, $b=\frac{16k}{\widetilde{\sigma}}$, and $S$. If it outputs \reject, then \Return \reject; otherwise, let $\fourier{\h}=(\fourier{\h}(\xi))_{\xi\in S}$ denote the collection of Fourier coefficients it outputs, and $\h$ their inverse Fourier transform (modulo $M$) \Comment{Do not actually compute $\h$}
    \EndIf
    \State \label{algo:step:cover}\textsf{Projection Step:} Check whether $\totalvardist{\h}{\classksiirv[n]{k}} \leq \frac{\eps}{2}$ (as in~\cref{sec:projections}), and \Return \accept if it is the case. If not, \Return \reject.  \Comment{Mostly computational step}
  \end{algorithmic}
  \caption{Algorithm \texttt{Test-SIIRV}}\label{algo:ksiirv:tester}
\end{algorithm}


\subsubsection{Analyzing the subroutines}

We start with a simple fact, that we will use to bound the running time of our algorithm and which follows immediately from~\cite[Claim 2.4]{DKS:15}:
\begin{fact}\label{fact:bound:size:s}
  For $S$ as defined in Step~\ref{algo:step:dft:computation}, we have
  \[
  \abs{S} \leq Mk^2 \frac{C'}{2\widetilde{\sigma}} \sqrt{\ln\frac{1}{\delta}} \leq 100C' k^2 \sqrt{\ln\frac{4}{\eps}}\sqrt{\ln \frac{k}{\eps} +\log\log \frac{k}{\eps} + \frac{1}{2} \ln(16C'')}
  \leq C'' k^2\log^2\frac{k}{\eps}
  \]
  for a suitably large choice of the constant $C''>0$; from which we get $\delta \leq \frac{1}{4\sqrt{\abs{S}}}$.
\end{fact}

We then argue that with high probability, the estimates obtained in Step~\ref{algo:step:estimates:mu:sigma} will be accurate enough for our purposes. (The somewhat odd statement below, stating two distinct guarantees where the second implies the first, is due to the following: \cref{eq:guarantees:moments} will be the guarantee that (the completeness analysis of) our algorithm relies on, while the second, slightly stronger one, will only be used in the particular implementation of the ``projection step'' (Step~\ref{algo:step:cover}) from~\cref{sec:projections}.)
\begin{claim}[Estimating the first two moments (if $\p$ is a SIIRV)]\label{claim:estimate:moments}
  With probability at least $19/20$ over the $O(k)$ draws from $\p$ in Step~\ref{algo:step:estimates:mu:sigma}, the following holds. If $\p\in\classksiirv[n]{k}$, the estimates $\widetilde{\sigma},\widetilde{\mu}$ defined as the empirical mean and (unbiased) empirical variance meet the guarantees stated in Step~\ref{algo:step:estimates:mu:sigma} of the algorithm, namely
  \begin{equation}\label{eq:guarantees:moments}
      \frac{1}{2}\leq \frac{\widetilde{\sigma}^2}{\var_{X \sim \p}[X]+1} \leq 2, \qquad \abs{ \widetilde{\mu} - \shortexpect_{X \sim \p}[X] } \leq  \sqrt{\var_{X \sim \p}[X]}
  \end{equation}
  We even have a quantitatively slightly stronger guarantee:
  $
      \frac{2}{3}\leq \frac{\widetilde{\sigma}^2}{\var_{X \sim \p}[X]+1} \leq \frac{3}{2}$, and $\abs{ \widetilde{\mu} - \shortexpect_{X \sim \p}[X] } \leq  \frac{1}{2}\sqrt{\var_{X \sim \p}[X]}
  $.
\end{claim}
\begin{proof}We handle the estimation of the mean and variance separately.
\begin{description}
  \item[Estimating the mean.] $\widetilde{\mu}$ will be the usual empirical estimator, namely $\widetilde{\mu} \eqdef \frac{1}{m}\sum_{i=1}^m X_i$ for $X_1,\dots,X_m$ independently drawn from $\p$. Since $\shortexpect[\widetilde{\mu}]=\shortexpect_{X \sim \p}[X]$ and $\var[\widetilde{\mu}]= \frac{1}{m}\var_{X \sim \p}[X]$, Chebyshev's inequality guarantees that
  \[
      \Pr[ \abs{ \widetilde{\mu} - \shortexpect_{X \sim \p}[X] } >  \frac{1}{2}\sqrt{\var_{X \sim \p}[X]}] \leq \frac{4}{m}
  \]
  which can be made at most $1/200$ by choosing $m\geq 800$.
  \item[Estimating the variance.] The variance estimation is exactly the same as in~\cite[Lemma 6]{DDS:PBD:15}, observing that their argument only requires that $\p$ be the distribution of a sum of independent random variables (not necessarily a Poisson Binomial distribution). Namely, they establish that,\footnote{\cite[Lemma 6]{DDS:PBD:15} actually only deals with the case $k=2$; but the bound we state follows immediately from their proof and the simple observation that the excess kurtosis $\kappa$ of an $(n,k)$-SIIRV with variance $s^2$ is at most ${k^2}/{s^2}$.} letting $\widetilde{\sigma}^2\eqdef\frac{1}{m-1}\sum_{i=1}^m (X_i-\frac{1}{m}\sum_{j=1}^m X_j)^2$ be the (unbiased) sample variances, and $s^2\eqdef \var_{X \sim \p}[X]$,
   \[
       \Pr[ \abs{ \widetilde{\sigma}^2 - s^2 } > \alpha(1+s^2) ] \leq \frac{4s^4+k^2s^2}{\alpha^2(1+s^2)^2} \frac{1}{m}
       \leq \frac{4s^4+s^2}{\alpha^2(1+s^2)^2}\cdot  \frac{k^2}{m} \leq \frac{4k^2}{\alpha^2m}
   \]
   which for $\alpha=1/3$ is at most $9/200$ by choosing $m\geq 800k$.
\end{description}
A union bound completes the proof, giving a probability of error at most $\frac{1}{200}+\frac{9}{200}=\frac{1}{20}$.
\end{proof}

\begin{claim}[Checking the effective support]\label{claim:estimate:effectivesupport}
  With probability at least $19/20$ over the draws from $\p$ in Step~\ref{algo:step:effectivesupport}, the following holds.
  \begin{itemize}
    \item if $\p\in\classksiirv[n]{k}$ and~\eqref{eq:guarantees:moments} holds, then $\p(I)\geq 1-\frac{\eps}{5}$ and the algorithm does not output \reject in Step~\ref{algo:step:effectivesupport:smallvariance} nor~\ref{algo:step:effectivesupport};
    \item if $\p$ puts probability mass more than $\frac{\eps}{4}$ outside of $I$, then the algorithm outputs \reject in Step~\ref{algo:step:effectivesupport:smallvariance} or~\ref{algo:step:effectivesupport}.
  \end{itemize}
\end{claim}
\begin{proof}Suppose first $\p\in\classksiirv[n]{k}$ and~\eqref{eq:guarantees:moments} holds, and set $s\eqdef \sqrt{\var_{X\sim\p}[X]}$ and $\mu\eqdef \shortexpect_{X\sim\p}[X]$ as before. By Bennett's inequality applied to $X$, we have
  \begin{equation}\label{eq:bennett:application}
    \Pr[ X > \mu+t ] \leq \exp\left(- \frac{s^2}{k^2} \vartheta\left( \frac{kt}{s^2} \right) \right)
  \end{equation}
  for any $t>0$, where $\vartheta\colon\R_+^\ast\to\R$ is defined by $\vartheta(x) = (1+x)\ln(1+x)-x$.
\begin{itemize}
  \item If the algorithm reaches Step~\ref{algo:step:effectivesupport:smallvariance}, then $s\leq 4k \sqrt{\ln\frac{10}{\eps}}$. Setting $t=\alpha\cdot k \ln\frac{10}{\eps}$ in~\cref{eq:bennett:application} (for $\alpha>0$ to be determined shortly), and $u = \frac{kt}{s^2} = \alpha\frac{k^2}{s^2}\ln\frac{10}{\eps} \geq \frac{\alpha}{16}$, 
  \[
      \frac{s^2}{k^2} \vartheta\left( \frac{kt}{s^2} \right) = \alpha\ln\frac{10}{\eps} \cdot \frac{\vartheta\left( u \right)}{u}
      \geq \left(16\vartheta\left( \frac{\alpha}{16} \right)\right) \ln\frac{10}{\eps} \geq \ln\frac{10}{\eps}
  \]
  since $\frac{\vartheta\left(x\right)}{x} \geq \frac{\vartheta\left(\alpha/16\right)}{\alpha/16}$ for all $x\geq \frac{\alpha}{16}$; the last inequality for $\alpha\geq \alpha^\ast \simeq 2.08$ chosen to be the solution to $16\vartheta\left( \frac{\alpha^\ast}{16} \right)=1$. Thus, $\Pr[X > \mu+t] \leq \frac{\eps}{10}$. Similarly, we have $\Pr[ X < \mu-t ] \leq \frac{\eps}{10}$.
  As $\mu-2t\leq\mu-s\leq \widetilde{\mu} \leq \mu+s \leq \mu +2t$, we get $\Pr[ X \in I ] \geq 1-\frac{\eps}{5}$ as claimed.

  \item If the algorithm reaches Step~\ref{algo:step:effectivesupport}, then $s\geq k \sqrt{\ln\frac{10}{\eps}}$ and $M = 1+2\clg{ 6\widetilde{\sigma}\sqrt{\ln\frac{10}{\eps}}) } \geq 1+2\clg{ 3s\sqrt{\ln\frac{10}{\eps}}) }$. Setting $t=\beta s \sqrt{ \ln\frac{10}{\eps}}$ in~\cref{eq:bennett:application} (for $\beta>0$ to be determined shortly), and $u = \frac{kt}{s^2} = \beta\frac{k}{s}\sqrt{ \ln\frac{10}{\eps}}\leq \beta$, 
  \[
      \frac{s^2}{k^2} \vartheta\left( \frac{kt}{s^2} \right) = \frac{t^2}{s^2} \cdot \frac{\vartheta\left( u \right)}{u^2}
      = \beta^2\ln\frac{10}{\eps}\cdot \frac{\vartheta\left(u\right)}{u^2} \geq \ln\frac{10}{\eps}
  \]
  since $\frac{\vartheta\left(x\right)}{x^2} \geq \frac{\vartheta\left(\beta\right)}{\beta^2}$ for all $x\in(0,\beta]$; the last inequality for $\beta= e-1 \simeq 1.72$ chosen to be the solution to $\vartheta\left( \beta \right)=1$. Thus, $\Pr[X > \mu+t] \leq \frac{\eps}{10}$. Similarly, it holds $\Pr[ X < \mu-t ] \leq \frac{\eps}{10}.$
Now note that $ \lfloor\widetilde{\mu} \rfloor+(M-1)/2 \geq (\mu-s)+\lceil 2s\sqrt{\ln\frac{10}{\eps}}) \rceil \geq \mu + t$
and $\flr{ \widetilde{\mu} } - (M-1)/2 \leq \mu-t$, implying that $X$ is in $[\flr{ \widetilde{\mu} }-(M-1)/2, \flr{ \widetilde{\mu} }+(M-1)/2]$ with probability at least $1- \frac{\eps}{5}$ as desired.
\end{itemize}

To conclude and establish the conclusion of the first item, as well as the second item, recall that distinguishing with probability $19/20$ between the cases $\p(\bar{I})\leq\frac{\eps}{5}$ and $\p(\bar{I})>\frac{\eps}{4}$ can be done with $O(1/\eps)$ samples.
\end{proof}

\begin{claim}[Learning when the effective support is small]\label{claim:learn:empirical}
  If $\p$ satisfies $\p(I)\geq 1-\frac{\eps}{4}$, and the ``If'' statement at Step~\ref{algo:step:small:variance} holds, then with probability at least $19/20$ the empirical distribution $\h$ obtained in Step~\ref{algo:step:empirical:smallvariance} satisfies (i) $\totalvardist{\p}{\h} \leq \frac{\eps}{2}$ and (ii) $\normtwo{\fourier{\p}-\fourier{\h}} \leq \frac{\eps^2}{100}$.
\end{claim}
\begin{proof}The first item, (i), follows from standard bounds on the rate of convergence of the empirical distribution (namely, that $O(r/\eps^2)$ samples suffice for it to approximate an arbitrary distribution over support of size $r$ up to total variation distance $\eps$). Recalling that in this branch of the algorithm, $S=\modulo{M}$ with $M=O(k\log(1/\eps))$, the second item, (ii), is proven by the same argument as in (the first bullet in) the proof of~\cref{theo:ft:effective:support}.
\end{proof}

\begin{claim}[Any $(n,k)$-SIIRV puts near all its Fourier mass in $S$]\label{claim:ksiirv:fourier:concentrated}
  If $\p \in \classksiirv[n]{k}$ and~\eqref{eq:guarantees:moments} holds, then
  $
      \normtwo{\fourier{\p}\indicSet{\bar{S}}}^2 = \sum_{\xi\notin S} \dabs{\fourier{\p}(\xi)}^2 \leq \frac{\eps^2}{100}
  $.
\end{claim}
\begin{proof}Since $\p \in \classksiirv[n]{k}$, our assumptions imply that (with the notations of~\cref{lemma:FourierSupportLem}) the set of large Fourier coefficients satisfies $\setOfSuchThat{\xi\in[M-1]}{ \abs{\fourier{\p}(\xi)} > \delta } \subseteq \mathcal{L}(\delta, M, s)\subseteq S$. Therefore, $\xi\notin S$ implies $\dabs{\fourier{\p}(\xi)} \leq \delta$. We then can conclude as follows: applying~\cref{lemma:FourierSupportLem} (ii) with parameter $\delta 2^{-r-1}$ for each $r \geq 0$, this is at most
\begin{align}
\sum_{r\geq 0} (\delta 2^{-r})^2  \abs{\setOfSuchThat{ \xi }{ |\fourier{\p}(\xi)| > \delta 2^{-r-1} }} 
&\leq  \frac{4Mk \delta^2}{s} \sum_{r\geq 0} 4^{-r}\sqrt{\log(2^{r+2}/\delta)} \notag\\
& \leq \frac{4Mk \delta^2}{s} \sqrt{\log\frac{1}{\delta}} \sum_{r\geq 0} 4^{-r}\sqrt{\log(2^{r+1})} \notag\\
& \leq \frac{12Mk \delta^2}{s} \sqrt{\log\frac{1}{\delta}} = \bigO{\eps^2} \label{eq:dealing:small:coeffs:p}
\end{align}
again at most $\frac{\eps^2}{100}$ for big enough $C''$ in the definition of $\delta$.
\end{proof}


\subsubsection{Putting it together}

In what follows, we implicitly assume that $I$ (as defined in Step~\ref{algo:step:def:interval} of~\cref{algo:ksiirv:tester}) is equal to $\modulo{M}$. This can be done without loss of generality, as this is just a shifting of the interval and all our Fourier arguments are made modulo $M$.

\begin{lemma}[Putting it together: completeness]\label{lemma:completeness}
  If $\p \in \classksiirv[n]{k}$, then the algorithm outputs \accept with probability at least $3/5$.
\end{lemma}
\begin{proof}
Assume $\p \in \classksiirv[n]{k}$. We condition on the estimates obtained in Step~\ref{algo:step:estimates:mu:sigma} to meet their accuracy guarantees, which by~\cref{claim:estimate:moments} holds with probability at least $19/20$: that is, we hereafter assume~\cref{eq:guarantees:moments} holds. Since the variance of any $(n,k)$-SIIRV is at most $s^2 \leq n k^2$, we consequently have $\widetilde{\sigma} \leq 2k\sqrt{n}$ and the algorithm does not output \reject in Step~\ref{algo:step:check:stdev}.

\begin{itemize}
\item \textbf{Case 1:} the branch in Step~\ref{algo:step:small:variance} is taken.
In this case, by~\cref{claim:estimate:effectivesupport} the algorithm does not output \reject in Step~\ref{algo:step:effectivesupport:smallvariance} with probability $19/20$. Since $\p(I)\geq 1-\frac{\eps}{4}$, by~\cref{claim:learn:empirical} we get that with probability at least $19/20$ it is the case that $\totalvardist{\p}{\h}\leq \frac{\eps}{2}$, and therefore the computational check in Step~\ref{algo:step:cover} will succeed, and return \accept. Overall, by a union bound the algorithm is successful with probability at least $1-3/20>3/5$.

\item\textbf{Case 2:} the branch in Step~\ref{algo:step:big:variance} is taken.
In this case, by~\cref{claim:estimate:effectivesupport} the algorithm does not output \reject in Step~\ref{algo:step:effectivesupport} with probability $19/20$. From~\cref{claim:ksiirv:l2:norm}, we know that $\p'$ as defined in Step~\ref{algo:step:fourier:support} satisfies $\normtwo{\p'}^2 \leq \frac{8k}{s}\leq \frac{16k}{\widetilde{\sigma}} = b$.  Moreover,~\cref{claim:ksiirv:fourier:concentrated} guarantees that $\normtwo{\fourier{\p'}\indicSet{\bar{S}}} \leq \frac{\eps}{10\sqrt{M}} = \frac{\eps'}{2}$ (for $\eps' = \frac{\eps}{5\sqrt{M}}$). Since Step~\ref{algo:step:fourier:support} calls~\cref{algo:ft:effective:support} with parameters $M, \eps', b$, and $S$, \cref{theo:ft:effective:support:iii} of~\cref{theo:ft:effective:support} ensures that (with probability at least $7/10$) the algorithm will not output \reject in Step~\ref{algo:step:fourier:support}, but instead return the $S$-sparse Fourier transform of some $\h$ supported on $\modulo{M}$ with $\normtwo{\p'-\h} \leq \frac{6}{5}\eps' = \frac{6\eps}{25\sqrt{M}}$. 

By Cauchy--Schwarz, we then have $\normone{\p'-\h} \leq \sqrt{M}\normtwo{\p'-\h} \leq \frac{6\eps}{25}$, i.e. $\totalvardist{\p'}{\h} \leq \frac{3\eps}{25}$. But since
$\totalvardist{\p}{\p'} \leq \frac{\eps}{4}$, we get $\totalvardist{\p}{\h} \leq \frac{\eps}{4}+\frac{3\eps}{25} < \frac{\eps}{2}$, and the computational check in Step~\ref{algo:step:cover} will succeed, and return \accept.
Overall, by a union bound the algorithm accepts with probability at least $1-(1/20+1/20+3/10)=3/5$.
\end{itemize}
\end{proof}

\begin{lemma}[Putting it together: soundness]\label{lemma:soundness}
  If $ \totalvardist{\p}{\classksiirv[n]{k}} > \eps$, then the algorithm outputs \reject with probability at least $3/5$.
\end{lemma}
\begin{proof}
We will proceed by contrapositive, and show that if the algorithm returns \accept with probability at least $3/5$ then $ \totalvardist{\p}{\classksiirv[n]{k}} \leq \eps$. 
Depending on the branch of the algorithm followed, we assume the samples taken either in
 \begin{itemize}
    \item Steps~\ref{algo:step:estimates:mu:sigma},~\ref{algo:step:effectivesupport:smallvariance},~\ref{algo:step:empirical:smallvariance}, meet the guarantees of~\cref{claim:learn:empirical,claim:estimate:moments,claim:estimate:effectivesupport} (by a union bound, this happens with probability at least $1-3/20 > 2/3$); or
    \item Steps~\ref{algo:step:estimates:mu:sigma},~\ref{algo:step:effectivesupport},~\ref{algo:step:fourier:support} meet the guarantees of~\cref{claim:estimate:moments,claim:estimate:effectivesupport,theo:ft:effective:support} (by a union bound, this happens with probability at least $1-(1/20+1/20+3/10)=3/5$).
 \end{itemize}
In particular, we hereafter assume that $\widetilde{\sigma} \leq 2k\sqrt{n}$.

\begin{itemize}
\item \textbf{Case 1:} the branch in Step~\ref{algo:step:small:variance} is taken.

  By the above discussion, we have $\p(I)\geq 1 - \frac{\eps}{4}$ by~\cref{claim:estimate:effectivesupport} so~\cref{claim:learn:empirical} and our conditioning ensure that the empirical distribution $\h$ is such that $\totalvardist{\p}{\h} \leq \frac{\eps}{2}$. Since the algorithm did not reject in Step~\ref{algo:step:cover}, there exists a $(n,k)$-SIIRV $\p^\ast$ such that $\totalvardist{\h}{\p^\ast} \leq \frac{\eps}{2}$: by the triangle inequality, $ \totalvardist{\p}{\classksiirv[n]{k}} \leq \totalvardist{\p}{\q^\ast} \leq \eps$.


\item \textbf{Case 2:} the branch in Step~\ref{algo:step:big:variance} is taken.

  In this case, we have $\p(I)\geq 1 - \frac{\eps}{4}$ by~\cref{claim:estimate:effectivesupport}. Furthermore, as the algorithm did not output \reject on Step~\ref{algo:step:fourier:support}, by~\cref{theo:ft:effective:support} we know that the inverse Fourier transform (modulo $M$) $\h$ of the $S$-sparse collection of Fourier coefficients $\fourier{\h}$ returned satisfies
  $
       \normtwo{\h-\p'} \leq \frac{6\eps}{25\sqrt{M}}
  $ 
  which by Cauchy--Schwarz implies, as both $\h$ and $\p'$ are supported on $\modulo{M}$, that $\normone{\h-\p'}\leq \frac{6\eps}{25}$, or equivalently $\totalvardist{\h}{\p'}\leq \frac{3\eps}{25}$.
  
  Finally, since the algorithm outputted \accept in Step~\ref{algo:step:cover}, there exists $\p^\ast\in\classksiirv[n]{k}$ (supported on $\modulo{M}$) such that $\totalvardist{\h}{\p^\ast}\leq \frac{\eps}{2}$, and by the triangle inequality
  \[
      \totalvardist{\p}{\p^\ast} \leq \totalvardist{\p}{\p'} + \totalvardist{\h}{\p'} + \totalvardist{\h}{\p^\ast} \leq \frac{\eps}{4} + \frac{3\eps}{25} + \frac{\eps}{2} \leq \eps
  \]
  and thus $ \totalvardist{\p}{\classksiirv[n]{k}} \leq \totalvardist{\p}{\p^\ast} \leq \eps$.
 \end{itemize}
\end{proof}

\begin{lemma}[Putting it together: sample complexity]\label{lemma:sample:complexity}
  The algorithm has sample complexity $O\left( \frac{k n^{1/4}}{\eps^2}\log^{1/4}\frac{1}{\eps} + \frac{k^2}{\eps^2} \log^2\frac{k}{\eps} \right)$.
\end{lemma}
\begin{proof}  \cref{algo:ksiirv:tester} takes samples in Steps~\ref{algo:step:estimates:mu:sigma},~\ref{algo:step:effectivesupport:smallvariance},~\ref{algo:step:effectivesupport}, and~\ref{algo:step:fourier:support}. The sample complexity is dominated by Steps~\ref{algo:step:empirical:smallvariance} and~\ref{algo:step:fourier:support}, which take respectively $N$ and 
  \begin{align*}
    O\left( \frac{\sqrt{b}}{(\eps/\sqrt{M})^2} + \frac{\abs{S}}{M(\eps/\sqrt{M})^2} + \sqrt{M} \right) 
    &= O\left( \frac{\sqrt{k \widetilde{\sigma}}}{\eps^2}\sqrt[4]{\log\frac{1}{\eps}} 
        + \frac{\abs{S}}{\eps^2}
        + \sqrt{\widetilde{\sigma}}\sqrt[4]{\log\frac{1}{\eps}} \right)
    \\
    &= O\left( \frac{k n^{1/4}}{\eps^2}\log^{1/4}\frac{1}{\eps} + \frac{k^2}{\eps^2}\log^2\frac{k}{\eps} \right)
  \end{align*}
  samples; recalling that Step~\ref{algo:step:check:stdev} ensured that $\widetilde{\sigma} \leq 2k\sqrt{n}$ and that $\abs{S}=\bigO{k^2\log^2\frac{k}{\eps}}$ by~\cref{fact:bound:size:s}.
\end{proof}


\begin{lemma}[Putting it together: time complexity]\label{lemma:running:time}
  The algorithm runs in time $\bigO{ \frac{k^4n^{1/4}}{\eps^2}\log^{4}\frac{k}{\eps} } + T(n,k,\eps)$, where $T(n,k,\eps)=n(k/\eps)^{\bigO{k\log(k/\eps)}}$ is the running time of the projection subroutine of Step~\ref{algo:step:cover}.
\end{lemma}
\begin{proof}  The running time, depending on the branch taken, is either $O(N)+T(n,k,\eps)$ for the first or $O\left( \abs{S}\left(\frac{k n^{1/4}}{\eps^2}\log^{1/4}\frac{1}{\eps} + \frac{k^2}{\eps^2}\log^2\frac{k}{\eps}\right) \right)+T(n,k,\eps)$ for the second (the latter from the running time of~\cref{algo:ft:effective:support}). Recalling that $\abs{S}=\bigO{k^2\log^2\frac{k}{\eps}}$ by~\cref{fact:bound:size:s} yields the claimed running time.
\end{proof}

 
\subsection{The General Tester}\label{sec:general:testing}
In this section, we abstract the ideas underlying the $(n,k)$-SIIRV from~\cref{sec:siirv:testing}, to provide a general testing framework. In more detail, our theorem (\cref{theo:testing:general}) has the following flavor: if $\property$ is a property of distributions such that every $\p\in\property$ has both (i) small effective support and (ii) sparse effective Fourier support, then one can test membership to $\property$ with $O(\sqrt{sM}/\eps^2+s/\eps^2)$ samples (where $M$ and $s$ are the bounds on the effective support and effective Fourier support, respectively). As a caveat, we do require that the sparse effective Fourier support $S$ be independent of $\p\in\property$, i.e., is a characteristic of the class $\property$ itself.

The high-level idea is then quite simple: the algorithm proceeds in three stages, namely the \emph{effective support test}, the \emph{Fourier effective support test}, and the \emph{projection step}. In the first, it takes some samples from $\p$ to identify what should be the effective support $I$ of $\p$, if $\p$ did have the property: and then checks that indeed $\abs{I}\leq M$ (as it should) and that $\p$ puts probability mass $1-O(\eps)$ on $I$.

In the second stage, it invokes the Fourier testing algorithm of~\cref{sec:fourier:support:testing} to verify that $\fourier{\p}$ indeed puts very little Fourier mass outside of $S$; and, having verified this, learns very accurately the set of Fourier coefficients of $\p$ on this set $S$, in $L_2$ distance.

At this point, either the algorithm has detected that $\p$ violates some required characteristic of the distributions in $\property$, in which case it has rejected already; or is guaranteed to have \emph{learned} a good approximation $\h$ of $\p$, by the Fourier learning performed in the second stage. It only remains to perform the third stage, which ``projects'' this good approximation $\h$ of $\p$ onto $\property$ to verify that $\h$ is close to some distribution $\p^\ast\in\property$ (as it should if indeed $\p\in\property$).

\begin{algorithm}[ht]
\algblockdefx[EFFECTIVE]{StartEffectiveSupport}{EndEffectiveSupport}{\textsf{Effective Support}}{}
\algblockdefx[FOURIER]{StartFourierSupport}{EndFourierSupport}{\textsf{Fourier Effective Support}}{}
\algblockdefx[PROJECT]{StartProjection}{EndProjection}{\textsf{Projection Step}}{}
  \begin{algorithmic}[1]
    \Require sample access to a distribution $\p\in\distribs{\N}$, parameter $\eps\in(0,1]$, $b\in(0,1]$, functions $S\colon(0,1]\to 2^{\N}$, $M\colon (0,1]\to\N$, $q_I\colon (0,1]\to\N$, and procedure $\textsc{Project}_{\property}$ as in~\cref{theo:testing:general}    
    \StartEffectiveSupport
      \State\label{algo:general:step:def:interval} Take $q_I(\eps)$ samples from $\p$ to identify a ``candidate set'' $I$.
            \Comment{Guaranteed to work w.p. $19/20$ if $\p\in\property$.}
      \State\label{algo:general:step:effectivesupport} Draw $O(1/\eps)$ samples from $\p$, to distinguish between $\p(I) \geq 1- \frac{\eps}{5}$ and $\p(I) < 1 - \frac{\eps}{4}$.
            \Comment{Correct w.p. $19/20$.}
      \If{$\abs{I} > M(\eps)$ or we detected that $\p(I) > \frac{\eps}{4}$}
        \State\label{algo:general:step:effectivesupportcheck} \Return \reject
      \EndIf
    \EndEffectiveSupport
    \StartFourierSupport
      \State\label{algo:general:step:fourier:support} Simulating sample access to $\p'\eqdef \p\bmod M(\eps)$, call~\cref{algo:ft:effective:support} on $\p'$ with parameters $M(\eps)$, $\frac{\eps}{5\sqrt{M(\eps)}}$, $b$, and $S(\eps)$.
      \If{\cref{algo:ft:effective:support} returned \reject}
        \State \Return \reject
      \EndIf
        \State Let $\fourier{\h}=(\fourier{\h}(\xi))_{\xi\in S(\eps)}$ denote the collection of Fourier coefficients it outputs, and $\h$ their inverse Fourier transform (modulo $M(\eps)$) \Comment{Do not actually compute $\h$ here.}
    \EndFourierSupport
    \StartProjection
      \State \label{algo:general:step:project} Call $\textsc{Project}_{\property}$ on parameters $\eps$ and $\h$, and \Return \accept if it does, \reject otherwise. 
    \EndProjection
  \end{algorithmic}
  \caption{Algorithm \texttt{Test-Fourier-Sparse-Class}}\label{algo:general:tester}
\end{algorithm}

\begin{theorem}[General Testing Statement]\label{theo:testing:general}
Assume $\property\subseteq \distribs{\N}$ is a property of distributions satisfying the following. There exist $S\colon(0,1]\to 2^{\N}$, $M\colon (0,1]\to\N$, and $q_I\colon (0,1]\to\N$ such that, for every $\eps\in(0,1]$,
\begin{enumerate}
  \item\label{condition:1} \textsf{Fourier sparsity:} for all $\p\in\property$, the Fourier transform (modulo $M(\eps)$) of $\p$ is concentrated on $S(\eps)$: namely,
  $
      \normtwo{\fourier{\p}\indicSet{\overline{S(\eps)}}}^2 \leq \frac{\eps^2}{100}
  $.
  \item\label{condition:2} \textsf{Support sparsity:} for all $\p\in\property$, there exists an interval $I(\p)\subseteq \N$ with $\abs{I(\p)} \leq M(\eps)$ such that (i) $\p$ is concentrated on $I(\p)$: namely,
  $
      \p(I(\p)) \geq 1-\frac{\eps}{5}
  $ and (ii) $I(\p)$ can be identified with probability at least $19/20$ from $q_I(\eps)$ samples from $\p$.
  \item\label{condition:3} \textsf{Projection:} there exists a procedure $\textsc{Project}_{\property}$ which, on input $\eps\in(0,1]$ and the explicit description of a distribution $\h\in\distribs{\N}$, runs in time $T(\eps)$; and outputs $\accept$ if $\totalvardist{\h}{\property}\leq \frac{2\eps}{5}$, and $\reject$ if $\totalvardist{\h}{\property} > \frac{\eps}{2}$ (and can answer either otherwise).
  \item\label{condition:4} \textsf{(Optional) $L_2$-norm bound:} there exists $b\in(0,1]$ such that, for all $\p\in\property$, $\normtwo{\p}^2 \leq b$.
\end{enumerate}
Then, there exists a testing algorithm for $\property$, in the usual standard sense: it outputs either \accept or \reject, and satisfies the following.
    \begin{enumerate}
        \item if $\p \in \property$, then it outputs \accept with probability at least $3/5$;
        \item if $ \totalvardist{\p}{\property} > \eps$, then it outputs \reject with probability at least $3/5$.
    \end{enumerate}
The algorithm takes 
\[
    \bigO{ \frac{\sqrt{\abs{S(\eps)}M(\eps)}}{\eps^2} + \frac{\abs{S(\eps)}}{\eps^2} + q_I(\eps)  }
\] samples from $\p$ (if~\cref{condition:4} holds, one can replace the above bound by
$
    \bigO{ \frac{\sqrt{b}M(\eps)}{\eps^2} + \frac{\abs{S(\eps)}}{\eps^2} + q_I(\eps)  }
$); and runs in time $\bigO{m\abs{S} + T(\eps)}$, where $m$ is the sample complexity.

Moreover, whenever the algorithm outputs \accept, it also \emph{learns} $\p$; that is, it provides a hypothesis $\h$ such that $\totalvardist{\p}{\h} \leq \eps$ with probability at least $3/5$.
\end{theorem}
We remark that the statement of~\cref{theo:testing:general} can be made slightly more general; specifically, one can allow the procedure $\textsc{Project}_{\property}$ to have sample access to $\p$ and err with small probability, and further provide it with the Fourier coefficients $\fourier{\h}$ learnt in the previous step.

\begin{proof}[Proof of~\cref{theo:testing:general}]
  For convenience, we hereafter write $S$ and $M$ instead of $S(\eps)$ and $M(\eps)$, respectively. Before establishing the theorem, which will be a generalization of (the second branch of)~\cref{algo:ksiirv:tester}, we note that it is sufficient to prove the version including~\cref{condition:4}. This is because, if no bound $b$ is provided, one can fall back to setting $b\eqdef\frac{\abs{S}+1}{M}$: indeed, for any $\p\in\property$,
  \begin{equation}\label{eq:default:bound:l2:norm}
      \normtwo{\p}^2 = \normtwo{\fourier{\p}}^2 = \normtwo{\fourier{\p}\indicSet{S}}^2+\normtwo{\fourier{\p}\indicSet{\bar{S}}}^2
      = \frac{1}{M}\sum_{\xi\in S}\dabs{\fourier{\p}(\xi)}^2 + \normtwo{\fourier{\p}\indicSet{\bar{S}}}^2
      \leq \frac{\abs{S}}{M} + \frac{\eps^2}{100M} = \frac{\abs{S}+\frac{\eps^2}{100}}{M}
  \end{equation}
  from~\cref{condition:1} and the fact that $\dabs{\fourier{\p}(\xi)}\leq 1$ for any $\xi\in\modulo{M}$. Then, we have $\sqrt{b}M \leq \sqrt{2\frac{\abs{S}}{M}}M = \sqrt{2\abs{S}M}$, concluding the remark. 
  
  The algorithm is given in~\cref{algo:general:tester}. Its sample complexity and running time are immediate from the assumptions on the input parameters, and its description; we thus focus on establishing its correctness.
  
  \begin{itemize}
      \item Completeness: suppose $\p\in\property$. Then, by definition of $q_I$ and $M$ (\cref{condition:2} of the theorem), we have that with probability at least $19/20$ the interval $I$ identified in Step~\ref{algo:general:step:def:interval} satisfies $\p(I)\geq 1-\frac{\eps}{5}$ and $\abs{I}\leq M$. In this case, also with probability at least $19/20$ the check in Step~\ref{algo:general:step:effectivesupport} succeeds, and the algorithm does not output \reject there.
      
      The call to~\cref{algo:ft:effective:support} in Step~\ref{algo:general:step:fourier:support} then, with probability at least 7/10, does not output \reject, but instead Fourier coefficients $\fourier{H}$ (supported on $S$) of some $\h$ such that $\h'=\h \bmod M$ satisfies $\normtwo{\h'-\p'} \leq \frac{6}{5}\cdot \frac{\eps}{5\sqrt{M}} = \frac{6\eps}{25\sqrt{M}}$ (this is because of the definition of $b$ and~\cref{condition:1}, which ensure the assumptions of~\cref{theo:ft:effective:support} are met). Thus $\normone{\h'-\p'} \leq \sqrt{M}\normtwo{\h'-\p'} \leq \frac{6\eps}{25}$. Since $\normtwo{\p-\p'} \leq 2\cdot\frac{\eps}{4}$ (as $\p(I)\geq 1-\frac{\eps}{4}$ and $\p'=\p\bmod M$), by the triangle inequality
      \[
        \totalvardist{\p}{\h'} = \frac{1}{2}\normone{\h'-\p'} \leq \frac{3\eps}{25}+\frac{\eps}{4} < \frac{2\eps}{5}
      \]
      and the algorithm returns \accept in Step~\ref{algo:general:step:project} (as promised by~\cref{condition:3}).
      
      Overall, by a union bound the algorithm is correct with probability at least $1-(\frac{1}{20}+\frac{1}{20}+\frac{3}{10}) \geq \frac{3}{5}$.
      
      \item Soundness: we proceed by contrapositive, and show that if the algorithm returns \accept with probability at least $3/5$ then $ \totalvardist{\p}{\property} \leq \eps$. 
We hereafter assume the guarantees of Steps~\ref{algo:general:step:def:interval},~\ref{algo:general:step:effectivesupport}, and~\ref{algo:general:step:fourier:support} hold, which by a union bound is the case with probability at least $1-(\frac{1}{20}+\frac{1}{20}+\frac{3}{10}) \geq \frac{3}{5}$.

Since the algorithm passed Step~\ref{algo:general:step:effectivesupportcheck}, we have $\p(I)\geq 1 - \frac{\eps}{4}$ and $\abs{I}\leq M$. Furthermore, as the algorithm did not output \reject on Step~\ref{algo:general:step:fourier:support}, by~\cref{theo:ft:effective:support} we know that the inverse Fourier transform (modulo $M$) $\h$ of the $S$-sparse collection of Fourier coefficients $\fourier{\h}$ returned satisfies, for $\h'\eqdef \h \bmod M$,
  \[
       \normtwo{\h'-\p'} \leq \frac{6\eps}{25\sqrt{M}}
  \]
  which by Cauchy--Schwarz implies that $\normone{\h-\p'}\leq \frac{6\eps}{25}$, or equivalently $\totalvardist{\h}{\p'}\leq \frac{3\eps}{25}$.
  
  Finally, since the algorithm outputted \accept in Step~\ref{algo:general:step:project}, there exists $\p^\ast\in\property$ (supported on $\modulo{M}$) such that $\totalvardist{\h}{\p^\ast}\leq \frac{\eps}{2}$, and by the triangle inequality
  \[
      \totalvardist{\p}{\p^\ast} \leq \totalvardist{\p}{\p'} + \totalvardist{\h}{\p'} + \totalvardist{\h}{\p^\ast} \leq \frac{\eps}{4} + \frac{3\eps}{25} + \frac{\eps}{2} \leq \eps
  \]
  and thus $ \totalvardist{\p}{\property} \leq \totalvardist{\p}{\p^\ast} \leq \eps$.
      
  \end{itemize}
\end{proof}

 
\subsection{The PMD Tester}\label{sec:pmd:testing}
In this section, we generalize our Fourier testing approach to higher dimensions, and leverage it to design a testing algorithm for the class of Poisson Multinomial distributions -- thus establishing~\cref{theo:testing:pmd} (restated below).

\begin{theorem}[Testing PMDs]
    Given parameters $k,n\in\N$, $\eps\in(0,1]$, and sample access to a distribution $\p$ over $\N$, there exists an algorithm (\cref{algo:pmd:tester}) which outputs either \accept or \reject, and satisfies the following.
    \begin{enumerate}
        \item if $\p \in \classpmd[n]{k}$, then it outputs \accept with probability at least $3/5$;
        \item if $\totalvardist{\p}{\classpmd[n]{k}} > \eps$, then it outputs \reject with probability at least $3/5$.
    \end{enumerate}
    Moreover, the algorithm takes $\bigO{\frac{n^{(k-1)/4} k^{2k} \log(k/\eps)^k}{\eps^2}}$ samples from $\p$, and runs in time $n^{O(k^3)} \cdot (1/\eps)^{O(k^3\frac{\log(k/\eps)}{\log\log(k/\eps)})^{k-1}}$ or alternatively in time $n^{O(k)} \cdot  2^{O(k^{5k} \log(1/\eps)^{k+2}}$.
\end{theorem}
The reason for the two different running times is that, for the projection step, one can use either the cover given by~\cite{DKS:15b} or that given by~\cite{DDKT:16}, which yield the two statements.
In contrast to~\cref{sec:siirv:testing} and~\cref{sec:general:testing}, for PMDs we will have to use a \emph{multidimensional} Fourier transform, which is a little more complicated -- and we define next.


Let $M \in \Z^{k \times k}$ be an integer $k \times k$ matrix.
We consider the integer lattice
$L  = L(M) =  M \Z^k \eqdef \{ p \in \Z^k \mid p = M q,  q \in \Z^k \}$, and its dual lattice
 $L^{\ast} = L^{\ast}(M)   \eqdef \setOfSuchThat{ \xi \in \R^k }{ \xi \cdot x \in \Z \textrm{ for all } x \in L }.$
 Note that  $L^{\ast} = (M^T)^{-1} \Z^k,$ and that $L^{\ast}$ is not necessarily integral.
The quotient $ \Z^k \slash L$ is the set of equivalence classes of points in $\Z^k$ such that two points $x, y \in \Z^k$
are in the same equivalence class if, and only if, $x - y \in L$.
Similarly, the quotient $L^{\ast} \slash \Z^k$ is the set of equivalence
classes of points in $L^{\ast}$ such that any two points $x, y \in L^{\ast}$ are in the same equivalence class if, and only if, $x -y \in \Z^k$.

The \emph{Discrete Fourier Transform (DFT) modulo $M$}, $M \in \Z^{k \times k}$, of a function
$F\colon \Z^k \to \C$ is  the function $\widehat{F}_M\colon L^{\ast} \slash \Z^k  \to \C$
defined as $\widehat{F}_M(\xi)\eqdef\sum_{x \in  \Z^k} e(\xi \cdot x) F(x).$ 
(We will omit the subscript $M$ when it is clear from the context.)
Similarly, for the case that $F$ is a probability mass function, we can equivalently write
$\widehat{F}(\xi)= \shortexpect_{X \sim F} \left[ e(\xi \cdot X) \right].$ The \emph{inverse DFT} of a function $\widehat{G}\colon L^{\ast} \slash \Z^k  \to \C$
is the function $G\colon A \to \C$ defined on a \emph{fundamental domain} $A$ of $L(M)$ as follows:
$G(x) = \frac{1}{|\det(M)|} \sum_{\xi \in L^{\ast} \slash \Z^k} \widehat{G}(x) e(- \xi \cdot x).$
Note that these operations are inverse of each other,
namely for any function $F\colon A \to \C$, the inverse DFT of $\widehat{F}$ is identified with  $F$.

With this in hand, \cref{algo:ft:effective:support} easily generalizes to high dimension:
\begin{algorithm}
  \begin{algorithmic}[1]
    \Require parameters, a $k \times k$ matrix $M$, $b,\eps\in(0,1]$; a fundamental domain $A$ of $L(M)$; sample access to distribution $\q$ over $A$
    \State\label{algo:ft:pmd:step:choosemprime} Set $m\gets \clg{C(\frac{\sqrt{b}}{\eps^2}+ \sqrt{\det(M)})}$ \Comment{$C>0$ is an absolute constant; $C=2000$ works.}
    \State Draw $m'\gets \poisson{m}$; if $m'>2m$, \Return \reject
    \State\label{algo:ft:pmd:step:empr} Draw $m'$ samples from $\q$, and let $\q'$ be the corresponding empirical distribution over $\modulo{M}$
    \State\label{algo:ft:pmd:step:norm} Compute $\normtwo{\q'}^2$, $\fourier{\q'}(\xi)$ for every $\xi\in S$, and $\normtwo{\fourier{\q'}\indicSet{S}}^2$ \Comment{Takes time $\bigO{m\abs{S}}$}
    \If{ $m'^2\normtwo{\q'}^2 - m' > \frac{3}{2}bm^2$ }\label{algo:ft:pmd:step:norm:check} \Return \reject
    \ElsIf{ $\normtwo{\q'}^2 - \normtwo{\fourier{\q'}\indicSet{S}}^2 \geq 3\eps^2+\frac{1}{m'}$ } \Return \reject
    \Else
      \State \Return $(\fourier{\q'}(\xi))_{\xi\in S}$
    \EndIf
  \end{algorithmic}
  \caption{Testing the Fourier Transform Effective Support in high dimension}\label{algo:ft:pmd:effective:support:high-dim}
\end{algorithm}

Crucially, we observe that the proof of~\cref{theo:ft:effective:support} nowhere requires that $\modulo{M}$ be a set of $M$ consecutive integers, but only that it is a fundamental domain of the lattice used in the DFT. Consequently,~\cref{theo:ft:effective:support} also applies in this high dimensional setting, with appropriate notation. Note that the size of any fundamental domain is $\det(M)$ which appears in place of $M$ in the sample complexity.

\begin{algorithm}[ht]
  \begin{algorithmic}[1]
    \Require sample access to a distribution $\p\in\distribs{\N^k}$, parameters $n,k\geq 1$ and $\eps\in(0,1]$
    
    \State\Comment{ Let $C,C',C''$ be sufficiently large universal constants }
    
    \State\label{algo:step:estimates:mu:sigma:pmd} Draw {$m_0 = O(k^4)$} samples from $X$, and 
let $\widehat{\mu}$ be the sample mean and $\widehat{\Sigma}$ the sample covariance matrix.
	
	\State Compute an approximate spectral decomposition of $\widehat{\Sigma}$, i.e., 
an orthonormal eigenbasis $v_i$ with corresponding eigenvalues $\lambda_i$, $i \in [k].$


	\State Set $M \in \Z^{k \times k}$  to be the matrix whose $i^{th}$ column 
is the closest integer point to the vector $C \left(\sqrt{k \log(k/\eps)\lambda_i+k^2\log^2(k/\eps)}\right)v_i.$
	
	\State Set $I \gets \Z^k\cap (\widehat{\mu} + M \cdot (-1/2,1/2]^k)$

	\State\label{algo:step:effectivesupport:pmd} Draw $O(1/\eps)$ samples from $\p$, and \Return \reject if any falls outside of $I$

	\State\label{algo:step:dft:computation:pmd} Let $S \subseteq (\R/\Z)^k$ to be the set of points $\xi = (\xi_1, \ldots, \xi_k)$ 
of the form $\xi = (M^T)^{-1} \cdot v {+\Z^k},$ for some $v\in \Z^k$ with $\|v\|_2 \leq C^2 k^2 \log(k/\eps).$

	\State Define $\p\bmod M$ to be the distribution obtained by sampling $X$ from $\p$ and if it lies outside in $I$, returning $X$, else returning $X + M b$ for the uniwue $b \in \Z^k$ such that $X + M b \in I$.

	\State\label{algo:step:fourier:support:pmd} Simulating sample access to $\p'\eqdef \p\bmod M$, call~\cref{algo:ft:pmd:effective:support:high-dim} on $\p'$ with parameters $M$, $\frac{\eps}{5\sqrt{\det(M)}}$, $b=\frac{|S|+1}{\det(M)}$, and $S$. If it outputs \reject, then \Return \reject; otherwise, let $\fourier{\h}=(\fourier{\h}(\xi))_{\xi\in S}$ denote the collection of Fourier coefficients it outputs, and $\h$ their inverse Fourier transform (modulo $M$) onto $I$. \Comment{Do not actually compute $\h$} 

    \State \label{algo:step:cover:pmd} Compute a proper $\eps/6\sqrt{|S|}$-cover $\mathcal{C}$ of all PMDs using the algorithm from~\cite{DKS:15c}.
	
	\For{each $\q \in \mathcal{C}$}
		\If{ the mean $\mu_\q$ and covariance matrix $\Sigma_\q$ satisfy $(\widehat{\mu}-\mu_\q)^T(\Sigma+I)^{-1}(\widehat{\mu}-\mu_\q) \leq 1$ and $2(\Sigma_\q+I) \geq \widehat{\Sigma}+I \geq (\Sigma_\q+I)/2.$}
			\State Compute $\fourier{\q}(\xi)$ for $\xi \in S$.
			\If{ $\sum_{\xi \in S} |\fourier{\h}-\fourier{\q}|^2 \leq \eps^2/16$}
				\Return \accept
			\EndIf
		\EndIf
	\EndFor
	\State \Return \reject if we do not \accept for any $\q \in \mathcal{C}$.
  \end{algorithmic}
  \caption{Algorithm \texttt{Test-PMD}}\label{algo:pmd:tester}
\end{algorithm}

The proof of correctness of~\cref{algo:pmd:tester} is very similar to that of~\cref{algo:ksiirv:tester}, except that we need results from the proof of correctness of the PMD Fourier learning algorithm of~\cite{DKS:15c}; we will only sketch these ingredients here. That $I$ is an effective support of a PMD whose mean and covariance matrix we have estimated to within approprate error with high probability follows from Lemmas 3.3--3.6 of~\cite{DKS:15c}, the last of which gives that the probability mass outside of $I$ is at most $\eps/10$, smaller than that claimed for $I$ in the $(n,k)$-SIIRV algorithm. Lemma 3.3 gives, if $\p$ is a PMD, that the mean and covariance satisfy $(\widehat{\mu}-\mu)^T(\Sigma+I)^{-1}(\widehat{\mu}-\mu) = O(1)$ and $2(\Sigma_\q+I) \geq \widehat{\Sigma}+I \geq (\Sigma_q+I)/2.$ Again, with more samples, we can strengthen this to $(\widehat{\mu}-\mu)^T(\Sigma+I)^{-1}(\widehat{\mu}-\mu) = \frac{1}{2}$ and $(3/2)(\Sigma+I) \geq \widehat{\Sigma}+I \geq (\Sigma+I)/(3/2)$ with $O(k^4)$ samples.

\noindent The effective support of the Fourier transform of a PMD is given by the following proposition:
\begin{proposition}[Proposition 2.4 of~\cite{DKS:15c}] \label{prop:ft-effective-support}
Let $S$ be as in the algorithm. With probability at least $99/100$, the Fourier coefficients of $\p$ outside $S$ satisfy
$\sum_{\xi \in (L^{\ast}/\Z^k)  \setminus S }|\widehat{\p}(\xi)| < \eps/10.$

This holds not just for $\p$, but any $(n,k)$-PMD $\q$ whose mean $\mu_\q$ and covariance matrix $\Sigma_\q$ satisfy $(\widehat{\mu}-\mu_\q)^T(\Sigma+I)^{-1}(\widehat{\mu}-\mu) = O(1)$ and $2(\Sigma_\q+I) \geq \widehat{\Sigma}+I \geq (\Sigma_\q+I)/2.$
\end{proposition}

We need to show that this $L_1$ bound is stronger than the $L_2$ bound we need. Since every individual $\xi \notin S$ has $|\widehat{\p}(\xi)| < \eps/10$, we have
$$\sum_{\xi \in (L^{\ast}/\Z^k)  \setminus S }|\widehat{\p}(\xi)|^2 \leq \sum_{\xi \in (L^{\ast}/\Z^k)  \setminus S } \eps |\widehat{\p}(\xi)|/10 \leq \eps^2/100$$
and so $S$ is an effective support of the DFT modulo $M$.


To show that the value of $b$ is indeed a bound on $\normtwo{\p}^2$, we can use (\ref{eq:default:bound:l2:norm}), yielding that $\normtwo{\p}^2 \leq (|S|+1)/\det(M) = b $, where $\det(M)$ here is indeed the size of $I$.

The proof of correctness of the algorithm and the projection step is now very similar to the $(n,k)$-SIIRV case. We need to get bounds on the sample and time complexity.
We can bound the size of $S$ using
\begin{align*}
|S| & \leq \abs{\setOfSuchThat{ v\in \Z^k }{ \|v\|_2 \leq  C^2 k^2 \log(k/\eps) }} 
\leq \abs{\setOfSuchThat{ v\in \Z^k }{ \|v\|_\infty \leq  C^2 k^2 \log(k/\eps) }} \\
&= \left(1+2\lfloor C^2 k^2 \log(k/\eps)\rfloor \right)^k  
 = O(k^2\log(k/\eps))^k
\end{align*}
We can bound $\det(M)$ in terms of the $L_2$ norms of its columns using Hadamard's inequality
\[
  \det(M) \leq \prod_{i=1}^k \normtwo{M_i} \leq \prod_{i=1}^k \left( C \left(\sqrt{k \log(k/\eps)\lambda_i+k^2\log^2(k/\eps)}\right) + \sqrt{k} \right)
\]
recalling that $\lambda_i$ are the eigenvalues of $\widehat{\Sigma}$ which satisfies $2(\Sigma_\q+I) \geq \widehat{\Sigma}+I$.
 We need a bound on $\|\Sigma\|_2$. Each individual summand $k$-CRV (categorical random variable) is supported on unit vectors, the distance between any two of which is $\sqrt{2}$. Therefore we have that $\|\Sigma\|_2 \leq 2n$. Then $\lambda_i \leq 4n+1$ for every $1\leq i\leq k$; moreover, since the $k$ coordinates must sum to $n$, $\widehat{\Sigma}$ has rank at most $k-1$ and so at least one of the $\lambda_i$'s is zero. Combining these observations, we obtain
\[
    \det(M) \leq \sqrt{k^2\log^2\frac{k}{\eps}}\cdot \left(C^2 k(4n+2) \log\frac{k}{\eps} + k^2\log^2\frac{k}{\eps}\right)^{\frac{k-1}{2}} = k\log\frac{k}{\eps}\cdot \bigO{nk^2 \log\frac{k}{\eps}}^{\frac{k-1}{2}} \; .
\]
With high constant probability, the number of samples we need is then
 \begin{align*}
 & \bigO{ \frac{\sqrt{\abs{S}\det{M}}}{\eps^2} + \frac{\abs{S}}{\eps^2} + q_I(\eps)  } = 
  \frac{1}{\eps^2}\sqrt{k\log\frac{k}{\eps}}\cdot \bigO{nk^2 \log\frac{k}{\eps}}^{\frac{k-1}{4}} + \frac{O(k^2\log(k/\eps))^{k}}{\eps^2} + O(k^4) \\
 &= O(n^{(k-1)/4} k^{2k} \log(k/\eps)^k/\eps^2)
 \end{align*}
 The time complexity of the algorithm is dominated by the projection step. By Proposition 4.9 and Corollary 4.12 of~\cite{DKS:15c}, we can produce a proper
 $\eps$-cover of $\classpmd[n]{k}$ of size $n^{O(k^3)} \cdot (1/\eps)^{O(k^3\frac{\log(k/\eps)}{\log\log(k/\eps)})^{k-1}}$ in time also $n^{O(k^3)} \cdot (1/\eps)^{O(k^3\frac{\log(k/\eps)}{\log\log(k/\eps)})^{k-1}}$. Note that producing an $(\eps/6\sqrt{|S|})$-cover, as $=\eps/O(k^2\log(k/\eps))^{k/2}$, takes time $n^{O(k^3)} \cdot (1/\eps)^{O(k^3\frac{\log(k/\eps)}{\log\log(k/\eps)})^{k-1}}$ (which is also the size of the resulting cover). Hence the running time of the algorithm is at most $n^{O(k^3)} \cdot (1/\eps)^{O(k^3\frac{\log(k/\eps)}{\log\log(k/\eps)})^{k-1}}$.

 Alternatively,~\cite{DDKT:16} gives an $\eps$-cover of size $n^{O(k)} \cdot \min{2^{\poly(k/\eps)}, 2^{O(k^{5k} \log(1/\eps)^{k+2}}}$ that can also be constructed in polynomial time. By using this result, one needs to take time $n|S|\poly(\log(1/\eps))$ to compute the Fourier coefficients. Applying this to get an $\eps/O(k^2\log(k/\eps))^{k/2}$-cover means that unfortunately we are always doubly exponential in $k$. In this case, the running time of the algorithm is $n^{O(k)} \cdot  2^{O(k^{5k} \log(1/\eps)^{k+2}}$.
 
 

 
\subsection{The Discrete Log-Concavity Tester}\label{sec:log:concaves}
\begin{theorem}[Testing Log-Concavity]
Given parameters $n\in\N$, $\eps\in(0,1]$, and sample access to a distribution $\p$ over $\Z$, there exists an algorithm which outputs either \accept or \reject, and satisfies the following.
    \begin{enumerate}
        \item if $\p \in \classlogconcave[n]$, then it outputs \accept with probability at least $3/5$;
        \item if $\totalvardist{\p}{\classlogconcave[n]} > \eps$, then it outputs \reject with probability at least $3/5$.
    \end{enumerate}
where $\classlogconcave[n]$ denotes the class of (discrete) log-concave distributions over $\modulo{n}$. Moreover, the algorithm takes $O(\sqrt{n}/\eps^2) + \tildeO{(\log(n/\eps)/\eps)^{5/2}}$ samples from $\p$; and runs in time $O(\sqrt{n} \cdot \poly(1/\eps))$.
\end{theorem}

We will sketch the proof and algorithm here. We first remark that the Maximum Likelihood Estimator (MLE) for log-concave distributions can be formulated as a convex program~\cite{DR:11}, 
which can be solved in sample polynomial time. One advantage of the MLE for log-concave distributions is that it properly learns log-concave distributions (over support size $M$) 
to within Hellinger distance $\eps$ using $\tildeO{(\log M)/\eps^{5/2}}$ samples\footnote{We note that a similar, slightly stronger result is already known for \emph{continuous} log-concave distributions, which can be learned to Hellinger distance $\eps$ from only $O(\eps^{-5/2})$ samples~\cite{KS:16}. The proof of this result, however, does not seem to generalize to discrete log-concave distributions, which is our focus here; thus, we establish in~\cref{appendix:log:concave} the learning result we require, namely an upper bound on the sample complexity of the MLE estimator for learning the class of log-concave distributions over $\modulo{M}$ in Hellinger distance (\cref{theo:mle:logconcave}).}. Note that the squared Hellinger distance satisfies:
\[
\hellinger{\p}{\q}^2 = \sum_x (\sqrt(\p(x)-\sqrt{\q(x)})^2 = \sum_x \frac{(\p(x)-\q(x))^2}{(\sqrt{\p} +\sqrt{\q})^2} \geq \frac{\normtwo{\p-\q}}{2\max\{\p(x),\q(x)\}} \;.
\]
Further, it is known that a log-concave distribution with variance $\sigma^2$ is effectively supported in an interval of length $M=O(\log(1/\eps) \sigma)$ centered at the mean, 
and that its maximum probability is $O(1/\sigma)$ (See~\cref{fact:log-concave-standard}). Thus, by learning a log-concave distribution properly to within $\eps/\log(1/\eps)$ Hellinger distance, one also learns it to within $\frac{\eps}{\sqrt{M}}$ $L_2$-distance.

A log-concave distribution $\p$ has $L_2$ norm bounded by $\normtwo{\p}^2 \leq \max_x \p(x) \leq O(1/\sigma)$. 
It is easy to show using standard concentration bounds that $\p \bmod M$ also has $L_2$ norm $O(1/\sqrt{\sigma})$. 
We will prove in~\cref{prop:log-concave-sparse-FT} that its DFT modulo $M$ is effectively 
supported on a known set $S$ of size $|S|=O(\log(1/\eps)^2/\eps^2)$.

Thus our algorithm will work as follows: First we estimate the mean and variance under the assumption of log-concavity. 
We construct an interval $I$ of length $M=O(\log(1/\eps) \sigma)$ which would be containing the effective support if we were log-concave; 
and reject if it is not the case, i.e., too much probability mass falls outside $I$. Then we properly learn $\p$ to within $\eps/\log(1/\eps)$ Hellinger distance 
using the MLE of $\tildeO{(\log M)/\eps^{5/2}}$ samples,\footnote{Note that we here invoke the MLE estimator not on the full domain, but on the effective support, which contains at least $1-O(\eps^2)$ probability mass. This conditioning overall does not affect the sample complexity nor the distances, as it can only cause $O(\eps^2)$ error in total variation (and thus $O(\eps)$ in Hellinger distance).}\ giving a hypothesis $\h$. At this point, we reject if our estimates for the mean and variance 
are far from that of $\h$. Then we run an $L_2$ identity tester between $\p$ and $\h$, i.e., test whether the empirical distribution $\q$ of $O(M/\sigma\eps^2)$ samples is large. 
To do this efficiently, we compute $\normtwo{\q}^2-\normtwo{\fourier{\q}\indicSet{S}}^2/M + \normtwo{\fourier{\q}\indicSet{S}-\fourier{\h}\indicSet{S}}^2/M$ 
(since we know $\fourier{\h}$ is supported on $S$).

To do this in time $O(\sqrt{n} \cdot \poly(1/\eps)$, we need to compute the Fourier coefficients efficiently. The MLE for log-concave distributions 
is a piecewise exponential distribution with a number of pieces at most the number of samples~\cite{DR:11}, 
which is $\tildeO{(\log M)/\eps^{5/2}}$ in this case. Using the expression for the integral of an exponential function 
gives a simple closed-form expression for $\h(\xi)$ that we can compute in time $\tildeO{(\log M)/\eps^{5/2}}$.


\begin{proposition} \label{prop:log-concave-sparse-FT} 
Let $\p$ be a discrete log-concave distribution with variance $\sigma^2$ and  $M = O(\log(1/\eps) \sigma)$ be the size of its effective support. 
Then its Discrete Fourier transform is effectively supported on a known set $S$ of size $|S|=O(\log(1/\eps)^2/\eps^2)$. 
\end{proposition}
\begin{proof}
First we show that for any unimodal distribution, we can relate the maximum probability to the size of the effective support.

\begin{lemma} 
Let $\p$ be a unimodal distribution supported on $\Z$ such that the probability of the mode is $\p_{\max}$. 
Then the DFT modulo $M$ of $\p$ at $\xi \in [-M/2,M/2)$ has $\fourier{\p}(\xi)=O(\p_{\max} M/|\xi|)$.
\end{lemma}
\begin{proof}
Let $m$ be the mode of $\p$. Then we have 
\[
  \fourier{\p}(\xi)=\sum_{j=-\infty}^{m-1} \p(j) \exp\!\left(-2\pi i \frac{\xi j}{M}\right) + \sum_{j=m}^\infty \p(j) \exp\!\left(-2\pi i \frac{\xi j}{M}\right)\,.
\]
We will apply summation by parts to these two series. 
Let $g(x) = \sum_{j=m+1}^x \exp(-2\pi i \xi j/M)$ and $g(m)=0$. 
By a standard result on geometric series, we have $g(x)= -\frac{\exp(-2\pi i \xi (x+1)/M) - \exp(-2\pi i \xi (m+1)/M)}{1- \exp(-2\pi i \xi/M)}$.
\begin{claim} 
$|g(x)| = O(M/\xi)$ for all integers $x \geq m$. 
\end{claim}
\begin{proof}
The modulus of the numerator $|\exp(-2\pi i \xi (x+1)/M) - \exp(-2\pi i \xi (m+1)/M)|$ is at most $2$. 
We thus only need to find a lower bound for $|1- \exp(-2\pi i \xi/M|$.
\begin{align*}
|1- \exp(-2\pi i \xi/M)|^2 & = (1-\cos(2\pi\xi/M))^2 + \sin(2\pi\xi/M)^2 
 = 2 - 2 \cos(2\pi\xi/M) 
 = \Omega((\xi/M)^2) \;,
\end{align*}
and so $|g(x)| \leq 2/\sqrt{\Omega((\xi/M)^2)}=O(M/|\xi|)$.
\end{proof}
\noindent Now consider the following, for any $n > m$:
\[
  \sum_{j=m+1}^n \p(j) (g(j)-g(j-1)) + \sum_{j=m+1}^n g(j) (\p(j+1)-\p(j))  = \p(n+1)g(n)-\p(m+1)g(m) \; .
\]
Now $g(m)=0$ and $\p(n+1) \rightarrow 0$ as $n \rightarrow \infty$ while $g(n+1)$ is bounded for all $n$. 
Hence, the RHS tends to $0$ as $n \rightarrow \infty$ and we have:
\begin{align*} 
|\sum_{j=m+1}^\infty \p(j) \exp(-2\pi i \xi j/M)| 
&= |\sum_{j=m+1}^\infty \p(j) (g(j)-g(j-1))| 
= |\sum_{j=m+1}^\infty g(j) (\p(j+1)-\p(j)) | \\
& \leq O(M/\xi) \cdot \sum_{j=m+1}^\infty \left( \p(j)- \p(j+1) \right)
 = O(\p_{\max}M/\xi) \;.
\end{align*}
Similarly, we can show that $\sum_{j=-\infty}^{m-1} \p(j) \exp(-2\pi i \xi j/M)= O(\p_{\max}M/\xi)$ since $\p$ is monotone there as well.
\end{proof}
\noindent Then we can get a bound on the size of the effective support:
\begin{lemma} \label{lem:S-unimodal}
Let $\p$ be a unimodal distribution supported on $\Z$ such that the probability of the mode is $\p_{\max}$ and let $\eps \leq 1/M$. 
Then the DFT modulo $M$ of $\p$ has $\sum_{|\xi| > \ell} |\fourier{\p}|^2 \leq \eps^2/100$, where $\ell= \bigTheta{\p_{\max}^2M^2/\eps^2}$.
\end{lemma}
\begin{proof}
\begin{align*}
\sum_{|\xi| > \ell} |\fourier{P}|^2 & \leq 2\sum_{\xi=\ell+1}^{M/2} O(\p_{\max}M/\xi)^2 
 \leq O(\p_{\max}M)^2 \sum_{\xi=\ell+1}^\infty 1/\xi^2 
 \leq O(\p_{\max}^2M^2/\ell) \leq \frac{\eps^2}{100} \;. 
\end{align*}
\end{proof}
For log-concave distributions, we can relate $\p_{\max}$ and $M$ as follows,
\begin{fact} \label{fact:log-concave-standard} 
Let $\p$ be a discrete log-concave distribution with mean $\mu$ and variance $\sigma^2$. 
Then 
\begin{itemize}
\item $\p$ is unimodal;
\item its probability mass function satisfies $\p(x)=\exp(-O((x-\mu)/\sigma))/\sigma$; and
\item $\Pr[|X-\mu| \geq \Omega(\sigma \log(1/\eps))] \leq \eps$.
\end{itemize}
\end{fact}
Since $\p_{\max}=O(1/\sigma)$, we can take $M=O(\sigma \log(1/\eps)))=O(\log(1/\eps)/\p_{\max})$. 
Substituting this into Lemma \ref{lem:S-unimodal} completes the proof of the proposition.
\end{proof}
 
\subsection{Lower Bound for PMD Testing}\label{sec:lower:bounds}
In this section, we obtain a lower bound to complement our upper bound for testing Poisson Multinomial Distributions. Namely, we prove the following:
\begin{theorem}\label{theo:lb:pmd}
  There exists an absolute constant $c\in(0,1)$ such that the following holds. For any $k\leq n^c$, any testing algorithm for the class of $\classpmd[n]{k}$ must have sample complexity
  $
    \bigOmega{ \left(\frac{4\pi}{k}\right)^{{k}/{4}}\frac{n^{{(k-1)}/{4}}}{\eps^2} }
  $.
\end{theorem}
The proof will rely on the lower bound framework of~\cite{CDGR:16}, reducing testing $\classpmd[n]{k}$ to testing identity to some suitable hard distribution $\p^\ast\in\classpmd[n]{k}$. To do so, we need to (a) choose a convenient $\p^\ast\in\classpmd[n]{k}$; (b) prove that testing identity to $\p^\ast$ requires that many samples (we shall do so by invoking the~\cite{VV:14} instance-by-instance lower bound method); (c) provide an agnostic learning algorithm for $\classpmd[n]{k}$ with small enough sample complexity, for the reduction to go through. Invoking~\cite[Theorem 18]{CDGR:16} with these ingredients will then conclude the argument.
\begin{proof}[Proof of~\cref{theo:lb:pmd}]
In what follows, we choose our ``hard instance'' $\p^\ast\in\classpmd[n]{k}$ to be the PMD obtained by summing $n$ i.i.d. random variables, all uniformly distributed on $\{e_1,\dots,e_k\}$. This takes care of point (a) above.

To show (b), we will rely on a result of Valiant and Valiant, which showed in~\cite{VV:14} that testing identity to any discrete distribution $\p$ required $\bigOmega{\norm{\p^{-\max}_{-\eps}}_{2/3}/\eps^2}$ samples, where $\p^{-\max}_{-\eps}$ is the vector obtained by zeroing out the largest entry of $\p$, as well as a cumulative $\eps$ mass of the smallest entries. Since $\norm{\p^{-\max}_{-\eps}}_{2/3}$ is rather cumbersome to analyze, we shall instead use a slightly looser bound, considering $\normtwo{\p}$ as a proxy.
\begin{fact}\label{fact:23:2:holder}
For any discrete distribution $\p$, we have $\norm{\p}_{2/3} \geq \frac{1}{\normtwo{\p}}$. More generally, for any vector $x$ we have $\norm{x}_{2/3} \geq \frac{\normone{x}^2}{\normtwo{x}}$.
\end{fact}
\begin{proof}
It is sufficient to prove the second statement, which implies the first. This is in turn a straightforward application of H\"older's inequality, with parameters $(4,\frac{4}{3})$:
$
    \normone{x} = \sum_{i} \abs{x}_i^{1/2}\abs{x}_i^{1/2} \leq \left( \sum_{i} \abs{x}_i^{2}\right) ^{1/4} \left( \abs{x}_i^{2/3}\right)^{3/4}
$. Squaring both sides yields the claim.
\end{proof}
\begin{fact}
For our distribution $\p^\ast$, we have $\normtwo{\p^\ast} = \bigTheta{ \frac{k^{k/4}}{ (4\pi n)^{(k-1)/4}} }$.
\end{fact}
\begin{proof}
It is not hard to see that, from any $\mathbf{n}=(n_1,\dots,n_k)\in\N^k$ such that $\sum_{i=1}^k n_i = n$, $\p^\ast( \mathbf{n} ) = \frac{1}{k^n} \binom{n}{n_1,\dots,n_k}$ (where $\binom{n}{n_1,\dots,n_k}$ denotes the multinomial coefficient). From there, we have
\[
    \normtwo{\p^\ast}^2 = \frac{1}{k^{2n}}\sum_{n_1+\dots+n_k=n} \binom{n}{n_1,\dots,n_k}^2 \operatorname*{\sim}_{n\to\infty} \frac{1}{k^{2n}} \cdot k^{2n}\frac{k^{k/2}}{ (4\pi n)^{(k-1)/2}}
\]
where the equivalent is due to Richmond and Shallit~\cite{RS:08:numbertheory}.
\end{proof}
\noindent However, from~\cref{fact:23:2:holder} we want to get a hold on $\normtwo{{\p^\ast}^{-\max}_{-\eps}}$, not $\normtwo{\p^\ast}$ (since $\normone{{\p^\ast}^{-\max}_{-\eps}}^2 \geq 1-\Omega(\eps)$, we then will have our lower bound on $\norm{{\p^\ast}^{-\max}_{-\eps}}_{2/3}$). Fortunately, the two are related: namely, $\normtwo{{\p^\ast}^{-\max}_{-\eps}}\leq \normtwo{\p^\ast}$, so
$
    \frac{1}{\normtwo{{\p^\ast}^{-\max}_{-\eps}}} \geq \frac{1}{\normtwo{\p^\ast}}
$ which is the direction we need.

\noindent Combining the three facts above establishes (b), providing a lower bound of $q_{\rm hard}(n,k,\eps) = \bigOmega{ \frac{ (4\pi n)^{(k-1)/4}}{k^{k/4}\eps^2} }$ for testing identity to $\p^\ast$. It only remains to establish (c):

\begin{lemma}
There exists a (not necessarily efficient) agnostic learner for $\classpmd[n]{k}$, with sample complexity $q_{\rm agn}(n,k,\eps) = \frac{1}{\eps^2}\left( O(k^2\log n)+ \bigO{ \frac{k\log(k/\eps)}{\log\log(k/\eps)} }^k \right)$.
\end{lemma}
\begin{proof}
This is implied by a result of~\cite{DKS:15c}, which establishes the existence of a (proper) $\eps$-cover $\mathcal{M}_{n,k,\eps}$ of $\classpmd[n]{k}$ such that
$
    \abs{ \mathcal{M}_{n,k,\eps} } \leq n^{O(k^2)}\cdot (1/\eps)^{\bigO{ \frac{k\log(k/\eps)}{\log\log(k/\eps)} }^{k-1}}
$. By standard arguments, this yields information-theoretically an agnostic learner with sample complexity $\bigO{\frac{\log\abs{ \mathcal{M}_{n,k,\eps} } }{\eps^2}}$.
\end{proof}

Having (a), (b), and (c), an application of~\cite[Theorem 18]{CDGR:16} yields that, as long as 
$
  q_{\rm agn}(n,k,\eps) = o( q_{\rm hard}(n,k,\eps) )
$ then testing membership to $\classpmd[n]{k}$ requires $\bigOmega{q_{\rm hard}(n,k,\eps)}$ samples as well. This in particular holds for $k = o(n^{c})$ (where e.g. $c<1/9$) and $\eps = 1/2^{O(n)}$.

\end{proof}

\subsection{Learning Discrete Log-Concave Distributions in Hellinger Distance}\label{appendix:log:concave}

Recall that the Hellinger distance between two probability distributions over a domain $\domain$ is defined as 
\[
\hellinger{p}{q} \eqdef \frac{1}{\sqrt{2}}\normtwo{\sqrt{p}-\sqrt{q}}
\]
where the 2-norm is to be interpreted as either the $\lp[2]$ distance or $L^2$ distance between the pmf or pdf's of $p,q$, depending on whether $\domain$ is $\Z$ or $\R$. In particular, one can extend this metric to the set of \emph{pseudo}-distributions over $\domain$, relaxing the requirement that the measures sum to one. We let $\mathcal{F}_{\domain}$ denote the set of pseudo-distributions over $\domain$. The \emph{bracketing entropy} of a family of functions $\mathcal{G}\subseteq\R^{\domain}$ with respect to the Hellinger distance (for parameter $\eps$) if then the minimum cardinality of a collection $\class$ of pairs $(g_L,g_U)\in \mathcal{F}_{\domain}^2$ such that every $f\in\mathcal{G}$ is ``bracketed'' between the elements of some pair in $\class$:
\[
    \bracketing{\eps}{\mathcal{G}} \eqdef \min\setOfSuchThat{ N\in\N }{ \exists \class \subseteq \mathcal{F}_{\domain}^2,\ \abs{\class}=N,\ \forall f\in\mathcal{G}, \exists (g_L,g_U)\in\class\text{ s.t. } g_L\leq f\leq g_U \text{ and } \hellinger{g_L}{g_H}\leq \eps }
\]

\begin{theorem}\label{theo:mle:logconcave}
Let $\hat{p}_m$ denote the maximum likelihood estimator (MLE) for discrete log-concave distributions on a sample of size $m$. Then, the minimax supremum risk satisfies
\[
    \sup_{p\in\classlogconcave[n]}\shortexpect_p[ \hellinger{\hat{p}_m}{p}^2 ] = \bigO{ \frac{\log^{4/5} (mn)}{m^{4/5}} }.
\]
\end{theorem}

Note that it is known that for \emph{continuous} log-concave distributions over $\R$, the rate of the MLE is $O(m^{-4/5})$~\cite{KS:16}; this result, however, does not generalize to discrete log-concavity, as it crucially relies on a scaling argument which does not work in the discrete case. On the other hand, one can derive a rate of convergence to learn discrete log-concave distributions in \emph{total variation distance} (using another estimator than the MLE), getting again $O(m^{-4/5})$ in that case~\cite{DKS:16}. However, due to the loose upper bound relating total variation and Hellinger distance, this latter result only implies an $O(m^{-2/5})$ convergence rate in Hellinger distance, which is quadratically worse than what we would hope for.

Thus, the result above, while involving a logarithmic dependence on the support size, has the advantage of getting the ``right'' rate of convergence. (While this additional dependence does not matter for our purposes, we believe a modification of our techniques would allow one to get rid of it, obtaining a rate of $\tildeO{m^{-4/5}}$ instead.) We however conjecture that the tight rate of convergence should be $O(m^{-4/5})$, as in the continuous case (i.e., without the dependence on the domain size $n$ nor the extra logarithmic factors in $m$).

In order to prove~\cref{theo:mle:logconcave}, we obtain along the way several interesting results on discrete (and continuous) log-concave distributions, namely a bound on their bracketing entropy (\cref{theo:bracketing:hellinger}) and an approximation result (\cref{theo:approx:hellinger}), which we believe are of independent interest.\medskip

In what follows, $\domain$ will denote either $\R$ or $\Z$; we let $\classlogconcave[\domain]$ denote the set of log-concave distributions over $\domain$, and $\classlogconcave[n]\subseteq \classlogconcave[\Z]$ be the subset of log-concave distributions supported on $\modulo{n}$.
\begin{theorem}\label{theo:bracketing:hellinger}
  For every $\eps\in(0,1)$,
  \[
      \bracketing{\eps}{\classlogconcave[n]} \leq \left(\frac{n}{\eps}\right)^{O(1/\sqrt{\eps})}
  \]  
\end{theorem}

A crucial element in to establish~\cref{theo:bracketing:hellinger} will be the following theorem, which shows that log-concave distributions are well-approximated (in Hellinger distance) by piecewise-constant pseudo-distributions with few pieces:
 \begin{theorem}\label{theo:approx:hellinger}
  Let $\domain$ be either $\R$ or $\Z$. For every $p\in\classlogconcave[\domain]$ and $\eps\in(0,1)$, there exists a pseudo-distribution $g$ such that (i) $g$ is piecewise-linear with $\bigO{1/\sqrt{\eps}}$ pieces; (ii) $g$ is supported on an interval $[a,b]$ with $p(\domain\setminus[a,b]) = O(\eps^2)$; and (iii) $\hellinger{p}{g}\leq \eps$. (Moreover, one can choose to enforce $g\leq p$, or $p\leq g$, on $[a,b]$).
\end{theorem}

The proof of~\cref{theo:approx:hellinger} will be very similar to that of~\cite[Theorem 12]{DKS:16}; specifically, we will use the following (reformulation of a) lemma due to Diakonikolas, Kane, and Stewart:
\begin{lemma}[{\cite[Lemma 14]{DKS:16}, rephrased}]\label{lemma:logconcave:pointwise:approx}
  Let $\domain$ be either $\R$ or $\Z$. Let $f$ be a log-concave function defined on an interval $I\subseteq \domain$, and suppose that $f(I)\subseteq [a,2a]$ for some constant $a>0$. Furthermore, suppose that the logarithmic derivative of $f$ (or, if $\domain=\Z$, the log-finite difference of $f$) varies by at most $1/\abs{I}$ on $I$; then, for any $\eps\in(0,1)$ there exists two piecewise linear functions $g^\ell,g^u\colon I\mapsto \R$ with $\bigO{1/\sqrt{\eps}}$ pieces such that
  \begin{equation}\label{eq:logconcave:pointwise:approx}
      \abs{f(x)-g^j(x)} = \bigO{\eps}f(x), \qquad j\in\{\ell,u\}
  \end{equation}
  for all $x\in I$, and with $g^\ell\leq f\leq g^u$.
\end{lemma}
\begin{proof}
  Observe that it suffices to establish~\cref{eq:logconcave:pointwise:approx} for a piecewise linear function $g\colon I\mapsto \R$ with $\bigO{1/\sqrt{\eps}}$ pieces; indeed, then in order to obtain $g^\ell,g^u$ from $g$, it will be sufficient to scale it by respectively $(1+\alpha\eps)^{-1}$ and  $(1+\alpha\eps)$ (for a suitably big absolute constant $\alpha>0$), thus ensuring both~\cref{eq:logconcave:pointwise:approx} and $g^\ell\leq f\leq g^u$. We therefore focus hereafter on obtaining such a pseudo-distribution $g$.

  For ease of notation, we write $h$ for the logarithmic derivative (or log-finite difference) of $f$ (e.g., in the continuous case, $h=(\ln f)'$). By rescaling $f$, we may assume without loss of generality that $a=1$. Note that $h$ is then bounded on $I$, i.e. $\abs{h}\leq {c}/{\abs{I}}$ for some absolute constant $c>0$.  We now partition $I$ into subintervals $J_1,J_2,\ldots,J_\ell$ so that (i) each $J_i$ has length at most $\eps^{1/2}\abs{I}$, and (ii) $h$ varies by at most $\eps^{1/2}/\abs{I}$ on each $J_i$. This can be achieved with $\ell=\bigO{1/\sqrt{\eps}}$ by placing an interval boundary every $\eps^{1/2}\abs{I}$ distance as well as every time $h$ passes a multiple of $\eps^{1/2}/\abs{I}$.

We now claim that on each interval $J_i$ there exists a linear function $g_i$ so that $\abs{g_i(x)-f(x)} = O(\eps)f(x)$ for all $x\in J_i$. Letting $g$ be $g_i$ on $J_i$ will complete the proof. Fix any $i$, and write $J_i=[s_i,t_i]$. Letting $\alpha_0\in h(J_i)$ be an arbitrary value in the range spanned by $h$ on $J_i$, observe that for any $x\in J_i$ there exists $\alpha_x\in h(J_i)$ such that 
\[
  f(x) = f(s_i) e^{\alpha_x(x-s_i)}
\]
from which we have
\begin{align*}
  f(x) &= f(s_i) e^{\alpha_0(x-s_i)+(\alpha_x-\alpha_0)(x-s_i)}
  = f(s_i) e^{\alpha_0(x-s_i)}e^{(\alpha_x-\alpha_0)(x-s_i)}\\
  &= f(s_i) \left(1+\alpha_0(x-s_i)+O(\eps)\right)(1+O((\alpha_x-\alpha_0)(x-s_i)))\\
  &= f(s_i) \left(1+\alpha_0(x-s_i)+O(\eps)\right)(1+O(\eps))\\
  &= f(s_i)+\alpha_0f(s_i)(x-s_i)+O(\eps)
\end{align*}
recalling that $\abs{\alpha_0},\abs{\alpha_x}=O(1/\abs{I})$, $\abs{x-s_i}\leq \eps^{1/2}\abs{I}$, and $\abs{\alpha_x-\alpha_0}\leq \eps^{1/2}/\abs{I}$, so that $\abs{\alpha_0(x-s_i)} = O(\eps^{1/2})$ and $\abs{(\alpha_x-\alpha_0)(x-s_i)} = O(\eps)$. This motivates defining the affine function $g_i$ as
\[
    g_i(x) \eqdef f(s_i)+\alpha_0f(s_i)(x-s_i), \qquad x\in J_i
\]
from which
\begin{align*}
    \abs{\frac{f(x)-g_i(x)}{f(x)}} 
    &= \abs{1-\frac{f(s_i)+\alpha_0f(s_i)(x-s_i)}{f(s_i) e^{\alpha_x(x-s_i)}}}
    = \abs{1-\frac{1+\alpha_0(x-s_i)}{e^{\alpha_x(x-s_i)}}} \\
    &= \abs{1-\frac{1+\alpha_0(x-s_i)}{1+\alpha_x(x-s_i)+O(\eps)}} 
    = \abs{1-\left( 1+\alpha_0(x-s_i)\right)\left( 1-\alpha_x(x-s_i)+O(\eps)\right)} \\
    &= \abs{(\alpha_x-\alpha_0)(x-s_i)+O(\eps)} = O(\eps)
\end{align*}
as claimed. This concludes the proof.
\end{proof}

\noindent We will also rely on the following proposition, from the same paper:
\begin{proposition}[{\cite[Proposition 15]{DKS:16}}]\label{proposition:logconcave:interval:partition}
Let $f$ be a log-concave distribution on $\domain$ (as before, either $\R$ or $\Z$). Then there exists a partition of $\domain$ into disjoint intervals $I_1, I_2,\ldots$ and a constant $C>0$ such that
\begin{itemize}
\item $f$ satisfies the hypotheses of~\cref{lemma:logconcave:pointwise:approx} on each $I_i$.
\item For each $m$, there are most $Cm$ values of $i$ so that $f(I_i) > 2^{-m}$.
\end{itemize}
(Moreover, $f$ is monotone on each $I_i$.)
\end{proposition}


\noindent We are now ready to prove~\cref{theo:approx:hellinger}:
\begin{proofof}{\cref{theo:approx:hellinger}}
Fix any $\eps\in(0,1)$, and $p\in\classlogconcave[\domain]$. We divide $\domain$ into intervals as described in~\cref{proposition:logconcave:interval:partition}. Call these intervals $I_1,I_2,\ldots$ sorted so that $p(I_i)$ is decreasing in $i$. Therefore, we have that $p(I_m) \leq 2^{-m/C}$.

For $1 \leq m \leq M\eqdef 2C\log(1/\eps)$, let $\eps_m\eqdef \eps 2^{m/(3C)}$; we use~\cref{lemma:logconcave:pointwise:approx} to approximate $p$ in $I_m$ by two piecewise linear functions $g^{\ell}_m, g^u_m$ so that (i) $g^j_m$ has at most $O(1/\sqrt{\eps_m})$ pieces and (ii) $p$ and $g^j_m$ are, on $I_m$, within a multiplicative $(1\pm O(\eps_m))$ factor with $g^{\ell}_m \leq p\leq g^u_m$. For $j\in\{\ell,u\}$, let $g^j$ be the piecewise linear function that is $g^j_m$ on $I_m$ for $1\leq m\leq M$, and $0$ elsewhere. $g^j$ is then piecewise linear on
\[
\sum_{m=1}^{M} O(\eps_m^{-1/2}) = \sum_{m=1}^{M} \bigO{ \eps^{-1/2} 2^{-\frac{m}{6C}} } = O(\eps^{-1/2})
\]
intervals.

Let $I$ be defined as the smallest interval such that $\bigcup_{m=1}^M I_m\subseteq I$. By definition, $g$ is $0$ outside of $I$, and moreover the total mass of $p$ there is
\[
  \sum_{m=M+1}^\infty p(I_m) \leq \sum_{m=M+1}^\infty \frac{1}{2^{m/C}} = \bigO{2^{-M/C}} = \bigO{\eps^2}
\]
By replacing $g^j$ by $\max(g^j,0)$, we may ensure that it is non-negative (while at most doubling the number of pieces without increasing the distance from $p$). This establishes the first two items of the theorem; we now turn to the third.

The Hellinger distance between $p$ and $g^j$ satisfies, letting $J\eqdef \bigcup_{m=1}^M I_m$,
\begin{align*}
  2\hellinger{p}{g^j}^2 &= \normtwo{\sqrt{p}-\sqrt{g^j}}^2
    = \int_{\domain}\left(\sqrt{p(x)}-\sqrt{g^j(x)}\right)^2 \mu(dx) \\
    &= \int_{\domain\setminus J}\left(\sqrt{p(x)}-\sqrt{g^j(x)}\right)^2 \mu(dx)+\int_{J}\left(\sqrt{p(x)}-\sqrt{g^j(x)}\right)^2 \mu(dx)\\
    &= \int_{\domain\setminus J}p(x) \mu(dx)+\sum_{m=1}^M\int_{I_m}p(x)\left(1-\sqrt{1\pm O(\eps_m)}\right)^2 \mu(dx)\\
    &\leq O(\eps^2) + \sum_{m=1}^M\int_{I_m}p(x)\left(1-\sqrt{1\pm O(\eps_m)}\right)^2 \mu(dx) \\
    &= O(\eps^2) + \sum_{m=1}^M\int_{I_m}p(x)O(\eps_m^2) \mu(dx) 
    = O(\eps^2) + \sum_{m=1}^M \bigO{\eps_m^2 p(I_m)} \\
    &= O(\eps^2) + \sum_{m=1}^M \bigO{\eps^2 2^{\frac{2m}{3C}}2^{\frac{-m}{C}}} 
    = O(\eps^2) + \sum_{m=1}^M \bigO{\eps^2 2^{\frac{-m}{3C}}} \\
    &= O(\eps^2) + O(\eps^2) = O(\eps^2)
\end{align*}
establishing the third item. (By dividing $\eps$ by a sufficiently big absolute constant before applying the above, one gets (i), (ii), and (iii) with $\hellinger{p}{g^j} \leq \eps$ as desired.) For technical reasons (that we will need in the proof of~\cref{theo:bracketing:hellinger}), instead of defining $[a,b]$ to be our interval $I$, we choose $[a,b]$ to be $I$ augmented with up to two of the remaining $I_m$'s (those directly on the left and right of $I$, defining $g^\ell_m, g^u_m$ on these two additional pieces as before by~\cref{lemma:logconcave:pointwise:approx}). This does not change the fact that the piecewise linear function obtained on $[a,b]$ has $O(\eps^{-1/2})$ pieces (we only added $o(\eps^{-1/2})$ pieces), and $p(\domain\setminus [a,b])\leq p(\domain\setminus I) = O(\eps^2)$. Finally, it is easy to see that this only changes, as per the computation above, the Hellinger distance by $O(\eps^2)$ as well. (The advantage of this technicality is that now, the two end intervals in the union constituting  $[a,b]$ have each total probability mass $O(\eps^2)$ under $p$, which will come in handy later.) It then only remains to choose $g$ to be either $g^\ell$ or $g^u$, depending on whether one wants a lower- or upperbound on $f$ (on $[a,b]$).
\end{proofof}

\noindent We can finally prove~\cref{theo:bracketing:hellinger}:
\begin{proofof}{\cref{theo:bracketing:hellinger}}
We can slightly strengthen the proof of~\cref{theo:approx:hellinger} for the case of $\classlogconcave[n]$, by imposing some restriction on the form of the `approximating distributions'' $g$. Namely, for any $\eps\in(0,1)$, fix any $p\in\classlogconcave[n]$ and consider the construction of $g^\ell,g^u$ as in the proof of~\cref{theo:approx:hellinger}. Clearly, we can assume $[a,b]\subseteq\modulo{n}$. 

Now, we modify $g^j$ as follows (for $j\in\{\ell,u\}$): for $1\leq m\leq M$, consider the interval $I_m=[a_m,b_m]$, and the corresponding ``piece'' $g^j_m$ of $g$ on $I_m$. We let $\tilde{g}^j_m$ be the pseudo-distribution defined from $g^j_m$ as follows: it is affine on $I_m$, with
\[
\tilde{g}^u_m(a_m) \eqdef \clg{g^u(a_m)\frac{M\abs{I_m}}{2\eps^2}}\frac{2\eps^2}{M\abs{I_m}}, \qquad \tilde{g}^u_m(a_m) \eqdef \clg{g^u(b_m)\frac{M\abs{I_m}}{2\eps^2}}\frac{2\eps^2}{M\abs{I_m}}
\]
and
\[
\tilde{g}^\ell_m(a_m) \eqdef \flr{g^\ell(a_m)\frac{M\abs{I_m}}{2\eps^2}}\frac{2\eps^2}{M\abs{I_m}}, \qquad \tilde{g}^\ell_m(a_m) \eqdef \flr{g^\ell(b_m)\frac{M\abs{I_m}}{2\eps^2}}\frac{2\eps^2}{M\abs{I_m}}
\]
i.e. $g^j_m$ is $g$ ``rounded up'' (resp. down) to the near multiple of $\frac{\eps^2}{M\abs{I_m}}$ on the endpoints. We then let $\tilde{g}^j$ be the correspond piecewise-affine pseudo-distribution defined by piecing together the $\tilde{g}^j_m$'s. Clearly, by construction $\tilde{g}^\ell$ and $\tilde{g}^u$ still satisfies (i) and (ii) of~\cref{theo:approx:hellinger}, and $\tilde{g}^\ell\leq p\leq\tilde{g}^u$. As for (iii), observe that at all $1\leq m\leq M$ and $k\in I_m$ we have $\abs{\tilde{g}^j(k)-g^j(k)} \leq \frac{2\eps^2}{M\abs{I_m}}$, from which
\[
  \hellinger{p}{\tilde{g}^j} \leq \hellinger{p}{g^j} + \hellinger{g}{\tilde{g}^j}
  \leq \eps + \sqrt{\totalvardist{g^j}{\tilde{g}^j}}
  \leq \eps + \sqrt{\frac{1}{2}\sum_{m=1}^M \abs{I_m}\cdot\frac{2\eps^2}{M\abs{I_m}}}
  = 2\eps
\]
showing that we get (up to a constant factor loss in the distance) (iii) as well. Given this, we get that specifying $(\tilde{g}^\ell,\tilde{g}^u)$ can be done by the list of the $O(1/\sqrt{\eps})$ endpoints along with the value of each $\tilde{g}^j$ for all of these endpoints. Now, given the two endpoints, one gets the size of the corresponding interval $I_m$ (which is at most $n$), and the two values to specify are a multiple of $\eps^2/(M\abs{I_m})$ in $[0,1]$. (If we were to stop here, we would get the existence of an $\eps$-cover $\class'_\eps$ of $\classlogconcave[n]$ in Hellinger distance of size $(n/\eps)^{O(1/\sqrt{\eps})}$.)


\paragraph{One last step: outside $[a,b]$} To get the bracketing bound we seek, we need to do one last modification to our pair $(\tilde{g}^\ell,\tilde{g}^u)$. Specifically, in the above we have one issue when approximating $p$: namely, that outside of their common support $\{a,\dots,b\}$, both $\tilde{g}^j$'s are $0$. While this is fine for the lower bound $\tilde{g}^\ell$, this is not for $\tilde{g}^u$, as it needs to dominate $p$ outside of $\{a,\dots,b\}$ as well, where $p$ may have $O(\eps^2)$  probability mass. Thus, we need to adapt the construction above, as follows (we treat the setting of $\tilde{g}^u$ on $\{b+1,\dots,n\}$, the definition on $\modulo{a}$ is similar).

First, observe if $p(b+1)=0$, we are done, as then by monotonicity we must have $(k)=0$ for all $k\geq b+1$, and so setting $\tilde{g}^u=0$ on $\{b+1,\dots,n\}$ suffices. Thus, we hereafter assume $p(b+1)>0$; and, for $b+1\leq k\leq n$, set
\[
    \tilde{g}^u(k) \eqdef \alpha e^{\beta(k-(b+1))}
\]
where $\alpha \eqdef \clg{p(b+1)\frac{n}{2\eps^2}}\frac{2\eps^2}{n}$ and $\beta \eqdef \clg{\frac{n}{\eps}\ln \frac{p(b+2)}{p(b+1)}}\frac{\eps}{n}$ (so that $\beta \leq 0$). Then $\tilde{g}^u(b+1)\geq p(b+1)$, and for $b+1<k\leq n$
\[
    \frac{\tilde{g}^u(k)}{\tilde{g}^u(k-1)} = e^\beta \geq \frac{p(b+2)}{p(b+1)} \geq \frac{p(k)}{p(k-1)}
\]
(the last inequality due to the log-concavity of $p$). This implies $\tilde{g}^u\geq p$ on $\{b+1,\dots,n\}$ as desired; and, thanks to the rounding, there are only $O(n/\eps^2)$ different possibilities for the tail of $\tilde{g}^u$. 
In view of bounding the Hellinger distance between $p$ and $\tilde{g}^u$ added by this modification, which is upper bounded by the (square root) of the total variation distance this added, recall that $p(\{b+1,\dots,n\})=O(\eps^2)$ by construction, and that
\[
    \tilde{g}^u(\{b+1,\dots,n\}) = \sum_{k=b+1}^n \alpha e^{\beta(k-(b+1))} = \frac{\alpha}{1-e^\beta}.
\]
Thus, the Hellinger distance incurred on $\{b+1,\dots,n\}$ is at most $\sqrt{O(\eps^2)+\frac{\alpha}{1-e^\beta}}$; and to conclude, it only remains to show that $\frac{\alpha}{1-e^\beta} = O(\eps^2)$.

To show this last point, let $I_m=[c,b]$ be the rightmost interval in the decomposition from~\cref{proposition:logconcave:interval:partition}. Recall that we are guaranteed that $p$ is non-increasing on $I_m$; further, by inspection of the proof of~\cite[Proposition 15]{DKS:16}, we also have that $I_m$ is \emph{maximal}, in the sense that $b$ is the rightmost point $k$ such that $[c,k]$ satisfies the assumptions of~\cref{lemma:logconcave:pointwise:approx}. Using first the monotoncity, we have
\[
    p(b+1)\leq p(b)\leq \frac{p(I_m)}{b-c} \leq \frac{O(\eps^2)}{b-c}
\]
that last inequality by construction (from the technicality we enforced in the end of the proof of~\cref{theo:approx:hellinger}); and therefore $\alpha \leq \frac{O(\eps^2)}{b-c} + \frac{\eps^2}{n} = \frac{O(\eps^2)}{b-c}$.

\noindent In order to obtain an upper bound on $\beta$, we rely on the maximality of $I_m$, leading to two cases to consider:
\begin{itemize}
  \item The first is that $p(b+1) < \frac{p(c)}{2}$; in which case $p(b+2) \leq p(b+1) < \frac{p(c)}{2}$; which implies that
\[
    \frac{1}{2} > \frac{p(b+2)}{p(c)} = \frac{p(b+2)}{p(b+1)}\cdot\frac{p(b+1)}{p(b)}\cdots\frac{p(c+1)}{p(c)} \geq \left(\frac{p(b+2)}{p(b+1)}\right)^{b-c+2}
\]
the last inequality by log-concavity. In turn, we get
\[
    \beta \leq \ln\frac{p(b+2)}{p(b+1)} + \frac{\eps}{n} \leq -\frac{\ln 2}{b-c+2}+ \frac{\eps}{n}.
\]
  \item The second is that $\ln\frac{p(c+1)}{p(c)} - \ln\frac{p(b+1)}{p(b)} > \frac{1}{b-c+1}$. In this case,
  \[
      \ln \frac{p(b+2)}{p(b+1)} \leq \ln \frac{p(b+1)}{p(b)} < \ln\frac{p(c+1)}{p(c)} - \frac{1}{b-c+1} \leq - \frac{1}{b-c+1} < -\frac{\ln 2}{b-c+2}
  \]
  (the last inequality as $b-c \geq 0$) and therefore $\beta\leq -\frac{\ln 2}{b-c+2}+ \frac{\eps}{n}$ as in the first case.
\end{itemize}
Combining these two bounds, we obtain
\[
    \frac{\alpha}{1-e^\beta} \leq \frac{O(\eps^2)}{b-c} \cdot \frac{1}{1-e^{\frac{\eps}{n}}e^{-\frac{\ln 2}{b-c+2}}} = O(\eps^2)
\]
the last inequality for $\eps < \frac{\ln 2}{2}$ (using the fact that $1\leq b-c\leq n$). This concludes the proof: as discussed, we then have that our setting of $\bar{g}^u$ outside of $[a,b]$ only causes an addition Hellinger distance of $\sqrt{O(\eps^2)+\frac{\alpha}{1-e^\beta}} = \sqrt{O(\eps^2)}=O(\eps)$.

\end{proofof}

We are, at last, ready to prove our main theorem:
\begin{proofof}{\cref{theo:mle:logconcave}}
    Recall the following theorem, due to Wong and Shen~\cite{WS:95} (see also~\cite[Theorem 7.4]{vdG:00},~\cite[Theorem 17]{KS:16}):
    \begin{theorem}[{\cite[Theorem 2]{WS:95}}]
        There exist positive constants $\tau_1,\tau_2,\tau_3,\tau_4>0$ such that, for all $\eps\in(0,1)$, if
        \begin{equation}\label{eq:mle:logconcave:condition}
            \int_{\eps^2/2^8}^{\sqrt{2}\eps} \sqrt{\bracketing{u/\tau_1}{\mathcal{G}}}\, du \leq \tau_2 m^{1/2}\eps^2
        \end{equation}
        and $\tilde{p}_n$ is an estimator that approximates $\hat{p}_m$ within error $\eta$ (i.e., solves the maximization problem within additive error $\eta$) with $\eta \leq \tau_3\eps^2$, then
        \[
            \probaOf{ \hellinger{\tilde{p}_m}{p} \geq \eps } \leq 5\exp(-\tau_4 m \eps^2).
        \]
    \end{theorem}
    To apply this theorem, define the function $J_n\colon(0,1)\to\R$ by $J(x) \eqdef \int_{x^2}^x \sqrt{\ln\frac{n}{u}}u^{-1/4}\, du$. By (tedious) computations, one can verify that $J_n(x) \sim_{x\to 0} \frac{4}{3}x^{3/4}\sqrt{\ln\frac{n}{x}}$; this, combined with the bound of~\cref{theo:bracketing:hellinger}, yields that for any $\eps\in(0,1)$
    \[
        \int_{\eps^2/2^8}^{\sqrt{2}\eps} \sqrt{\bracketing{u/\tau_1}{\classlogconcave[n]}}\, du = \bigO{\eps^{3/4}\sqrt{\ln\frac{n}{\eps}}}.
    \]
    Thus, setting, for $m\geq 1$, $\eps_m \eqdef C m^{-2/5}(\ln(mn))^{2/5}$ for a sufficiently big absolute constant $C>0$ ensures that $\eps_m$ satisfies~\eqref{eq:mle:logconcave:condition}. Let $\rho_{m} \eqdef 1/\eps_m$. It follows that any estimator which, on a sample of size $m$, approximates the log-concave MLE to within an additive $\eta_m\eqdef \tau_3\eps_m^2$ has minimax error
    \begin{align*}
        \rho^2_{m}\sup_{p\in\classlogconcave[n]}\shortexpect_p[ \hellinger{\tilde{p}_m}{p}^2 ] &
        = \sup_{p\in\classlogconcave[n]} \int_0^\infty {  \probaOf{ \rho^2_{m} \hellinger{\tilde{p}_n}{p}^2 \geq t } }\, dt \\
        &= \sup_{p\in\classlogconcave[n]} \int_0^\infty {  \probaOf{  \hellinger{\tilde{p}_n}{p} \geq \sqrt{t}\rho^{-1}_{m} } }\, dt \\
        &\leq 1+ \sup_{p\in\classlogconcave[n]} \int_1^\infty {  \probaOf{  \hellinger{\tilde{p}_n}{p} \geq \sqrt{t}\rho^{-1}_{m} } }\, dt \\
        &= 1+ \sup_{p\in\classlogconcave[n]} \int_1^\infty {  \probaOf{  \hellinger{\tilde{p}_n}{p} \geq \sqrt{t}\eps_{m} } }\, dt \\
        &\leq = 1+ 5\sup_{p\in\classlogconcave[n]} \int_1^\infty \exp(-\tau_4 m t\eps_m^2)\, dt \\
        &= 1+ 5\sup_{p\in\classlogconcave[n]} \int_1^\infty \exp(-\tau_4 Cm^{1/2}\ln(mn) t)\, dt \\
        &= O(1)
    \end{align*}
    where we used the fact that if $\eps_t > \eps_m$, then $\eps_t$ satisfies~\eqref{eq:mle:logconcave:condition} as well (and applied it to $\eps_t = \sqrt{t}\eps_{m}$). This concludes the proof.
\end{proofof}


\chapter*{Conclusion} % Not a numbered chapter
\addcontentsline{toc}{chapter}{Conclusion} % Puts your conclusion in your table of contents even though we have used the asterisk in the \chapter command above.

\epigraph{For the Snark \emph{was} a Boojum, you see.}{Lewis Carroll, \textit{The Hunting of the Snark}}


%%%%%%%%%%%%%%%%%%%%%%%%%%%%%%%%%%%%%%%%%%%%%%%%%%%%%%%%%%%%%%%%%%%%%%%%%%%%%%%%%%%%%%%%%%%%%%%%%%%% 
%%%%%%%%%%%%%%%%%%%%%%%%%%%%%%%%%%%%%%%%%%%%%%%%%%%%%%%%%%%%%%%%%%%%%%%%%%%%%%%%%%%%%%%%%%%%%%%%%%%% 
% Bibliography
\backmatter
\setlength{\bibitemsep}{\baselineskip} % Skip lines between bibliography entries. Columbia requires that you skip a line between entries.
\SingleSpacing % Start single-spacing text before you start the bibliography. We used \bibitemsep earlier in this document to keep bibliography items separated by one line of blank space, but we need to keep the entries themselves single-spaced.
\printbibliography % Print the bibliography.



\end{document}
